\part{Модуль 1}

\chapter{Лекция №1 16.02.24}

\section{Политология как наука. Учебная дисциплина}

\allocation{Рассмотрим 3 вопроса:}
\begin{itemize}
      \item Формирование и институализация политологии
      \item Объект и предмет политологии
      \item Система, категории, методы и функции политологии
\end{itemize}

\subsection{Формирование и институализация политологии}

Политическая наука в России развивалась в рамках политико-правовой мысли. В 17-18 веках
в Российской политической мысли особое внимание уделяется поиску наиболее оптимальных
форм государственной власти, взаимоотношений церковной власти и власти царя,
общественно-политического развития страны. Политико-правовая мысль эволюционирует
в политическую науку, которой считают наукой нравственной. Она признаётся как наука
политическая, поскольку законы восприниматься как важнейший инструмент государственной
политики.

\begin{itemize}
      \item \allocation{1703 г.} - открывается Нарышкинское училище, в котором начинают
            преподавать \textbf{этику и политическую}
      \item \allocation{1715 г.} - Петру 1 был представлен проект \textbf{об учреждении
                  в России академии политики} (для пользы государственной канцелярии)
      \item \allocation{1724-1725 г.} - В Санкт-Петербургской академии наук создаётся
            кафедра \textbf{права, публичной политики и этики}
      \item \allocation{Середина 18 века} - на юридическом факультете Московского
            университета создаётся кафедра \textbf{политики} (международных отношений
            и права).
      \item Юридический факультет Московского университета \allocation{в начале 19 века}
            преобразовывается в отделение \textbf{нравственных и политических наук}.
            В его составе создаются кафедры права, дипломатии и политической экономии
      \item \allocation{В 1841 году} в Санкт-Петербургской академии открывается
            направление \textbf{Истории и политических наук}
      \item \allocation{На рубеже 80-90 годов 20 века} в России складываются условия
            для легитимации политической науки, а разрозненные знания о политике
            сводятся в единую науку политологию. Постановление государственного комитета
            по науке и технике СССР \textbf{№386 от 4 ноября 1988 года} утверждается
            номенклатура специальностей научных работников под общим названием
            Политические науки
\end{itemize}

\subsubsection{Необходимость изучения политологии обусловлена следующими
      обстоятельствами:}
\begin{enumerate}
      \item \allocation{Политическое образование} позволяет овладеть системой знаний
            о политике
      \item \allocation{Политическое образование} это один из способов политическое
            социализации
      \item \allocation{Политическое образование} способствует превращению мышления
            и объекта манипуляции властей и оппозиции в коллективный субъект политики
\end{enumerate}

\subsection{Объект и предмет политологии}

\subsubsection{Сферы общества:}

\begin{itemize}
      \item Экономическая
      \item Политическая
      \item Духовная
      \item Социальная
\end{itemize}

\allocation{Объектом изучения политологии} является политическая сфера жизнедеятельности
общества - все явления и политические процессы происходящие в ней.

\allocation{Предметом политологии} является закономерности и тенденции развития
политической жизни общества, теоретической и прикладной деятельности субъектов политики
по поводу власти.

\subsubsection{Политологию интересуют такие вопросы как:}

\begin{itemize}
      \item Что находиться в основе возникновения, функционирования и развития
            политических мотивов, интересов, взглядов, концепций и теорий?
      \item Что предопределяет возникновение и развитие различных политических процессов,
            конкретные виды политических отношений и формы политической деятельности?
      \item Каковы причинно-следственные связи, закономерности и тенденции формирования
            и развития конкретных типов политической власти, а так же какова роль
            случайного в этом?
\end{itemize}

\allocation{Политология} - это наука об общих и специфических закономерностях и
тенденциях возникновения, функционирования и эволюции политических интересов, взглядов,
теорий и отношений, становления и развития политической власти - теоретической и
прикладной деятельности субъектов политики.

\subsection{Система, категории, методы и функции политологии}

Центральными базовыми понятиями политологии является: политика, власть, политические
интересы, политические отношения, политическая деятельность, политический процесс,
политическая система, политическая культура, политическая элита, политический лидер,
политическая партия и т.д.

\subsubsection{Функции политологии:}

\begin{enumerate}
      \item \allocation{Насеологическая (теоретико-познавательная)} - оснащает людей
            знаниями о политике
      \item \allocation{Методологическая} - раскрывает закономерности развития
            политических явлений, разрабатывает теорию и методологию, исследования
            политических отношений и институтов
      \item \allocation{Управленческая} - поставляет субъектам власти информацию,
            необходимую для эффективного принятия решения, руководство и управление
      \item \allocation{Прогнатическая} - выступает в качестве предупреждения и
            эффективного разрешения возникающих конфликтов и противоречий
      \item \allocation{Идеологическая} - выражается в обосновании политических идеалов
            и ценностей
      \item \allocation{Воспитательная} - служит задачей воспитания личности гражданина
            (включает в себя гражданское, патриотическое, идеологическое, духовное и
            политическое воспитание)
\end{enumerate}

\chapter{Лекция №2 01.03.24}

\section{Теория политики: сущность, содержание, функции}

\allocation{Рассмотрим 2 вопроса:}
\begin{itemize}
      \item Сущность, структуры и функции политики
      \item Классификации ресурсов политики
\end{itemize}

Наша цель разобраться что такое политика, выяснить её роль и узнать её структуру.

\subsection{Сущность структуры и функции политики}

Весь спектр теоретических подходов к пониманию политики можно свести к 2 неравнозначным
структурам неравнозначным группам:

\begin{enumerate}
      \item \allocation{Анархическая теория политики} - фактически отрицает политику в
            общепринятом её понимании \\
            Сторонники данной теории утверждают, что политика есть наука о свободе, что
            власть человека над человеком, какую бы форму она не принимала, есть
            угнетение

      \item \allocation{Традиционная} - фактически объединяет все остальные теории
            политики \\
            Объединяющим началом в сущностном понимании политики является то, что всё
            вращается вокруг 2 базовых понятий: \textbf{власть} и \textbf{государство}
\end{enumerate}

\textbf{Расширительное понимание политики} - это отождествление её со всей социальной
действительностью. Так, например, один из западных политологов Макс Вебер отмечал, что
политика имеет чрезвычайно широкие рамки и охватывает все виды деятельности по
самостоятельному руководству.

По Ленину политика есть отношение между классами

\subsubsection{Представление о политике в историческом контексте}

\begin{itemize}
      \item \allocation{Директивное (авторитарное)} - понимание политики связано с
            представлением о том, что общественные отношения всегда носят силовой
            характер и это закон
      \item \allocation{Функциональное} - его связывают с платоном, потому что именно
            ему принадлежит идея разделения обязанностей и полномочий участников
            политического сообщества в соответствии с их знаниями и умениями (правители
            философы, войны, земледельцы, ремесленники)
      \item \allocation{Коммуникативное} - принадлежит Аристотелю: \textit{"Аристотелю
                  и семья, и население, и полис это общение, причём полис это вполне
                  завершённое и совершенное"}
\end{itemize}

\textbf{Политика} - это деятельность в сфере отношений, между большими социальными
группами (классами, нациями, государствами) по поводу выявления и согласования их
интересов, установление и функционирование на этой основе политической власти

\subsection{Содержание политики}

\begin{itemize}
      \item Субъекты политики
      \item Интересы, установки и цели субъектов политики
      \item Практическая деятельность субъектов политики
      \item Методы и средства политики
      \item Функции политики
      \item Наука и искусство о политике
\end{itemize}

\subsubsection{Субъекты политики}

К субъектам политики относят:
\begin{enumerate}
      \item \allocation{Политическую элиту и бюрократию} - узкие группы лиц, способные
            концентрировать могущество власти
      \item \allocation{Элективные участники} - крупные соц группы и общности: класс,
            сословие, межклассовые группы
      \item \allocation{Непосредственные участник} - государство, партии, общественные
            организации и движения и т.д.
\end{enumerate}

Реализация политических целей и установок осуществляется политическими методами и
средствами

\subsubsection{Методы и средства политики:}

\begin{enumerate}
      \item \allocation{Метод системного комплексного подхода} - политика должна
            опираться на понимание общества как системы
      \item \allocation{Метод убеждения консенсуса и согласования} - так например по
            Платону: \textit{"Политика это искусство жить вместе"}
      \item \allocation{Метод интеграции} - учёт интересов всех социальных групп
            субъектов политики
      \item \allocation{Насильственный метод} - должен быть свойственен только главному
            субъекту политики - государству
      \item \allocation{Метод прогноза и перспективных оценок} - анализ прошлого опыта,
            изучение данных статистики, учёт общественного мнения, экспертных оценок,
            составление альтернативы вариантов общественного развития
\end{enumerate}

\subsubsection{Функции политики:}

\begin{enumerate}
      \item Выражение и реализация политических интересов социальных групп и слоёв
            общества, их признаний на государственном уровне
      \item Обеспечение преемственности и инновационности социального развития
      \item Рационализация конфликтов и придания межгрупповым отношениям цивилизованного
            характера
      \item Распределение и перераспределение общественных ресурсов с учётом приоритетов
            развития общества и интересов господствующих социально-политических сил
      \item Управление и руководство солильными процессами в интересах тех или иных
            групп и общества в целом
      \item Социализация личности и включение её в политическую жизнь
      \item Интеграция общества и обеспечение целостности общественных систем,
            стабильности общества и порядка
      \item Обеспечение социально-экономической коммуникации
\end{enumerate}

\subsection{Классификации ресурсов политики}

\subsubsection{Основания классификации:}

\begin{enumerate}
      \item По сферам жизнедеятельности:
            \begin{enumerate}
                  \item Внутреннее
                  \item Внешнее
            \end{enumerate}

      \item По масштабности и долговременности целей:
            \begin{enumerate}
                  \item Тактическое (узкое)
                  \item Стратегическое (широкое)
            \end{enumerate}

      \item По полноте охвата сфер общественное жизни:
            \begin{enumerate}
                  \item Научно-техническая (узкое)
                  \item Военная (широкое)
                  \item Экономическая
                  \item Социализация
            \end{enumerate}

      \item По субъектам политики:
            \begin{enumerate}
                  \item Политика партий
                  \item Политика общественных движений и организаций
                  \item Государственная политика
            \end{enumerate}
\end{enumerate}

\subsubsection{Ресурсы политики:}

\begin{enumerate}
      \item \allocation{Материальные} - силовой и интеллектуальный потенциал
      \item \allocation{Людские} - человеческий и демографический фактор
      \item \allocation{Моральные} - множество ценностей, обеспечивающих концентрацию
            власти в руках господствующих лиц
      \item \allocation{Пропагандистские} - идеология, объединяющую индустрию
            общественного мнения
      \item \allocation{Геополитические} - реально контролируемые жизненные пространства
\end{enumerate}

\chapter{Лекция №3 15.03.24}

\section{Политическая власть в управлении обществом}

\allocation{Рассмотрим 2 вопроса:}
\begin{itemize}
      \item Власть, как общественное явление
      \item Понятие политической и государственной власти
\end{itemize}

\subsection{Власть, как общественное явление}

\subsubsection{Власть это:}

\begin{itemize}
      \item Способность, право или возможность распоряжаться кем либо или чем либо;
            оказывает решающее воздействие на судьбы, поведение и деятельность людей с
            помощью различного рода средств - \allocation{право авторитета, воли,
                  убеждения, принуждения}
      \item Политическое господство над людьми
      \item Система органов управления
      \item Лица или органы, обличённые соответствующими административными полномочиями
\end{itemize}

Специфическим признаком власти является доминирование властной воли.

По широко распространёнными представлениям российских политологов суть власти можно
выразить в виде: \allocation{А имеет власть на Б, если А определяет поведение Б}.

\subsubsection{Источники власти:}

\begin{enumerate}
      \item Неоднородность положения людей в любом социальном образовании,
            дифиринцированность их ролевых функций
      \item Сила, как в физическом, так и в социальном смысле
      \item Богатство, то есть владение материальными ценностями, деньгами,
            собственностью или средств производства
      \item Знания, информация
      \item Человек, как универсальный и многофункциональный источник власти, создающий
            все её другие источники
\end{enumerate}

Любая власть это субъектно0объектные отношения, поэтому её структурными элементами
является субъект и объект.

Властвующий субъект воплощает в себе активное, направляет в начало. Им может быть как
отдельный человек, так и общность людей.

\subsubsection{Субъект власти должен обладать рядом качеств:}

\begin{itemize}
      \item Желание властвовать
      \item Воля к власти
      \item Готовность к ответственности за проученное дело
      \item Компетентностью
      \item Умение руководить подменёнными и обладать авторитетом
\end{itemize}

Власть невозможна без доминирования воли субъекта и без подчинения объекта. Если
отсутствует подчинение и ответные действия со стороны объекта, то нет и власти.

Структура власти и властных отношений определяется целым комплексом различного рода
средств осуществления власти, которые в современной литературе называют ресурсами власти.

\textbf{Ресурсы власти} - средства, использование которых обеспечивает влияние субъекта
власти на объект в соответствии с его целями.

\subsubsection{Виды ресурсов:}
\begin{itemize}
      \item \allocation{Утилитарные} - материальные и другие социальные блага, которые
            связаны с интересами людей
      \item \allocation{Принудительный} -  мера административного воздействия со стороны
            субъекта власти на объект
      \item \allocation{Нормативный} - средство воздействия на внутренний мир, ценностные
            ориентации и нормы поведения человека
\end{itemize}

\subsection{Понятие политической и государственной власти}

\textbf{Политическая власть} - специфическая форма общественных отношений между большими
группами людей, реальная способность определённого класса, социальной группы или индивида
проводить жизнь свою волю выраженную в политике.

\subsubsection{Отличительные признаки политической власти:}

\begin{itemize}
      \item \allocation{Верховенство} - обязательность её решений для всякой другой
            власти
      \item \allocation{Публичность} - всеобщность и безличностный характер
      \item \allocation{Моноцентричность} - наличие единого центра принятия решений
      \item \allocation{Многообразие ресурсов}
\end{itemize}

\subsubsection{Функции политической власти:}

\begin{itemize}
      \item Формирование политической системы общества
      \item Выработка стратегии управления обществом
      \item Контроль политических и других отношений и создание в конечном счёте
            определённого типа правления, политического режима и государственного строя
      \item Обеспечение законных прав граждан и их конституционных свобод
\end{itemize}

\textbf{Государственная власть} - это форма политической власти, её ядро,
характеризующиеся способностью влиять на характер, направление деятельностью и поведение
людей посредством экономических, социальных, политических, духовных и
организационно-правовых механизмов в целях обеспечения нормального функционирования
общества.

\subsubsection{Отличительные признаки государственной власти:}
\begin{itemize}
      \item Выступает в качестве силы, концентрировано выражающей и символизирующей
            общество в целом
      \item Монополия на легальное использование силы и физического принуждения с помощью
            аппарата насилия
      \item Наличие специального аппарата (механизма) управления всем обществом
      \item Исключительное право на нормирование жизни всего общества, право на издание
            законов и норм, обязательных для всего населения
      \item Право на взымание налогов и различных сборов
\end{itemize}

\subsubsection{Функции государственной власти:}
\begin{itemize}
      \item Внутренние функции государственной власти: экономическая, социальная,
            политическая, образовательная, культурно0воспитательная, правовая,
            организаторская
      \item Внешние функции государственной власти: защита границ и территории страны,
            поддержание и развитие межгосударственных отношений
\end{itemize}
