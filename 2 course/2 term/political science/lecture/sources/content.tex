\part{Модуль 1}

\chapter{Лекция №1 16.02.24}

\section{Политология как наука. Учебная дисциплина}

\allocation{Рассмотрим 3 вопроса:}
\begin{itemize}
      \item Формирование и институализация политологии
      \item Объект и предмет политологии
      \item Система, категории, методы и функции политологии
\end{itemize}

\subsection{Формирование и институализация политологии}

Политическая наука в России развивалась в рамках политико-правовой мысли. В 17-18 веках
в Российской политической мысли особое внимание уделяется поиску наиболее оптимальных
форм государственной власти, взаимоотношений церковной власти и власти царя, 
общественно-политического развития страны. Политико-правовая мысль эволюционирует
в политическую науку, которой считают наукой нравственной. Она признаётся как наука
политическая, поскольку законы восприниматься как важнейший инструмент государственной
политики.

\begin{itemize}
      \item \allocation{1703 г.} - открывается Нарышкинское училище, в котором начинают
            преподавать \textbf{этику и политическую}
      \item \allocation{1715 г.} - Петру 1 был представлен проект \textbf{об учреждении
            в России академии политики} (для пользы государственной канцелярии)
      \item \allocation{1724-1725 г.} - В Санкт-Петербургской академии наук создаётся 
            кафедра \textbf{права, публичной политики и этики}
      \item \allocation{Середина 18 века} - на юридическом факультете Московского 
            университета создаётся кафедра \textbf{политики} (международных отношений
            и права).
      \item Юридический факультет Московского университета \allocation{в начале 19 века}
            преобразовывается в отделение \textbf{нравственных и политических наук}.
            В его составе создаются кафедры права, дипломатии и политической экономии
      \item \allocation{В 1841 году} в Санкт-Петербургской академии открывается
            направление \textbf{Истории и политических наук}
      \item \allocation{На рубеже 80-90 годов 20 века} в России складываются условия 
            для легитимации политической науки, а разрозненные знания о политике 
            сводятся в единую науку политологию. Постановление государственного комитета
            по науке и технике СССР \textbf{№386 от 4 ноября 1988 года} утверждается
            номенклатура специальностей научных работников под общим названием Политические науки
\end{itemize}

\subsubsection{Необходимость изучения политологии обусловлена следующими
      обстоятельствами:}
\begin{enumerate}
      \item \allocation{Политическое образование} позволяет овладеть системой знаний
            о политике
      \item \allocation{Политическое образование} это один из способов политическое
            социализации
      \item \allocation{Политическое образование} способствует превращению мышления
            и объекта манипуляции властей и оппозиции в коллективный субъект политики
\end{enumerate}

\subsection{Объект и предмет политологии}

\subsubsection{Сферы общества:}
\begin{itemize}
      \item Экономическая
      \item Политическая
      \item Духовная
      \item Социальная
\end{itemize}

\allocation{Объектом изучения политологии} является политическая сфера жизнедеятельности
общества - все явления и политические процессы происходящие в ней.

\allocation{Предметом политологии} является закономерности и тенденции развития 
политической жизни общества, теоретической и прикладной деятельности субъектов политики
по поводу власти.

\subsubsection{Политологию интересуют такие вопросы как:}
\begin{itemize}
      \item Что находиться в основе возникновения, функционирования и развития
            политических мотивов, интересов, взглядов, концепций и теорий?
      \item Что предопределяет возникновение и развитие различных политических процессов,
            конкретные виды политических отношений и формы политической деятельности?
      \item Каковы причинно-следственные связи, закономерности и тенденции формирования
            и развития конкретных типов политической власти, а так же какова роль
            случайного в этом?
\end{itemize}

\allocation{Политология} - это наука об общих и специфических закономерностях и
тенденциях возникновения, функционирования и эволюции политических интересов, взглядов,
теорий и отношений, становления и развития политической власти - теоретической и
прикладной деятельности субъектов политики.

\subsection{Система, категории, методы и функции политологии}

Центральными базовыми понятиями политологии является: политика, власть, политические
интересы, политические отношения, политическая деятельность, политический процесс,
политическая система, политическая культура, политическая элита, политический лидер,
политическая партия и т.д.

\subsubsection{Функции политологии:}
\begin{enumerate}
      \item \allocation{Насеологическая (теоретико-познавательная)} - оснащает людей
            знаниями о политике
      \item \allocation{Методологическая} - раскрывает закономерности развития
            политических явлений, разрабатывает теорию и методологию, исследования 
            политических отношений и институтов
      \item \allocation{Управленческая} - поставляет субъектам власти информацию,
            необходимую для эффективного принятия решения, руководство и управление
      \item \allocation{Прогнатическая} - выступает в качестве предупреждения и
            эффективного разрешения возникающих конфликтов и противоречий
      \item \allocation{Идеологическая} - выражается в обосновании политических идеалов
            и ценностей
      \item \allocation{Воспитательная} - служит задачей воспитания личности гражданина
            (включает в себя гражданское, патриотическое, идеологическое, духовное и 
            политическое воспитание)
\end{enumerate}
