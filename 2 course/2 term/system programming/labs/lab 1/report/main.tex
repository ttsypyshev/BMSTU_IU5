\documentclass[bachelor,subf,12pt]{disser}
\usepackage[
a4paper, mag=1000, includefoot,
left=2cm, right=1cm, top=2cm, bottom=1.5cm, headsep=1cm, footskip=1cm
]{geometry}
\setcounter{tocdepth}{2}
\linespread{1} % интервал между строк

\usepackage[T2A]{fontenc}
\usepackage[utf8x]{inputenc}
\usepackage[english,russian]{babel}
\usepackage{cmap} % поиск в pdf
% \usepackage{multirow}
% \usepackage{longtable}

\usepackage{amsmath} % математика
\usepackage{tikz}
% \usepackage{pict2e}
% \usepackage{wasysym}

\ifpdf\usepackage[pdftex]{graphicx}\else\usepackage{graphicx}\fi
\ifpdf\usepackage{epstopdf}\usepackage{pdfpages}\fi
\graphicspath{ {images/} } % папка с фото
\renewcommand{\thechapterfont}{\normalsize\bfseries} % Номер главы полужирным
\renewcommand{\prethechapter}{} % Убираем слово "глава"
\renewcommand{\postthechapter}{.~} % ставим точку и пробел после номер
\renewcommand{\appendixfont}{\normalsize\bfseries}
\renewcommand{\chapterfont}{\normalsize\bfseries}
\renewcommand{\sectionfont}{\normalsize\bfseries}
\renewcommand{\subsectionfont}{\normalsize\bfseries}
\renewcommand{\subsubsectionfont}{\normalsize\bfseries}
\renewcommand{\tocprethechapter}{} % в оглавлении убираем слово "Глава"
\renewcommand{\sectionindent}{1cm}
\renewcommand{\subsectionindent}{1cm}
\renewcommand{\aftersection}{6pt plus .1pt}
\renewcommand{\aftersubsection}{3pt plus .1pt}

\usepackage{enumitem} % форматирование списков
\setlist{noitemsep}
% \setlist[1]{\labelindent=\parindent} % < Usually a good idea
\setlist[itemize]{leftmargin=*}
\setlist[itemize,1]{label=$\triangleleft$}
\setlist[enumerate]{labelsep=*, leftmargin=1.6pc}
\setlist[enumerate,1]{label=\arabic*., ref=\arabic*}
\setlist[enumerate,2]{label=\emph{\alph*}.,ref=\theenumi.\emph{\alph*}}
\setlist[enumerate,3]{label=\roman*., ref=\theenumii.\roman*}
\setlist[description]{font=\sffamily\bfseries}

\tikzset{baseline, inner sep=2pt, minimum height=12pt, rounded corners=2pt} % выделение текста (` `)
\newcommand{\code}[1]{\mbox{\ttfamily \tikz \node[anchor=base,fill=black!12]{#1};}}

\begin{document}

\begin{titlepage}
    \definecolor{color_29791}{rgb}{0,0,0}
    \definecolor{color_274846}{rgb}{1,0,0}
    \begin{tikzpicture}[overlay]\path(0pt,0pt);\end{tikzpicture}
    \begin{picture}(-5,0)(2.5,0)
        \put(32.225,-36.20001){\fontsize{12}{1}\usefont{T2A}{cmr}{b}{n}\selectfont\color{color_29791} Московский государственный технический университет  им. Н.Э.Баумана}
    \end{picture}
    \begin{tikzpicture}[overlay]
        \path(0pt,0pt);
        \filldraw[color_29791][even odd rule]
        (30pt, -38.79999pt)
        -- (428.00pt, -38.79999pt)
        -- (428.00pt, -38.79999pt)
        -- (428.00pt, -37.59998pt)
        -- (428.00pt, -37.59998pt)
        -- (30pt, -37.59998pt) -- cycle;
    \end{tikzpicture}
    \begin{picture}(-5,0)(2.5,0)
        \put(0,-91.22998){\fontsize{12}{1}\usefont{T2A}{cmr}{b}{n}\selectfont\color{color_29791} Защищено: }
        \put(0,-105.03){\fontsize{12}{1}\usefont{T2A}{cmr}{m}{n}\selectfont\color{color_29791} Большаков С.А. }
        \put(0,-132.62){\fontsize{12}{1}\usefont{T2A}{cmr}{m}{n}\selectfont\color{color_29791} "13" \_ февраля \_ 2024 г.}
    \end{picture}
    \begin{tikzpicture}[overlay]
        \path(0pt,0pt);
        \filldraw[color_29791][even odd rule]
        (3pt, -135pt) -- (17.00pt, -135pt)
        (22pt, -135pt) -- (93.00pt, -135pt);
    \end{tikzpicture}
    \begin{picture}(-5,0)(2.5,0)
        \put(280.00,-91.22998){\fontsize{12}{1}\usefont{T2A}{cmr}{b}{n}\selectfont\color{color_29791} Демонстрация ЛР: }
        \put(280.00,-105.03){\fontsize{12}{1}\usefont{T2A}{cmr}{m}{n}\selectfont\color{color_29791} Большаков С.А. }
        \put(280.00,-132.62){\fontsize{12}{1}\usefont{T2A}{cmr}{m}{n}\selectfont\color{color_29791} "13" \_ февраля \_ 2024 г.}
    \end{picture}
    \begin{tikzpicture}[overlay]
        \path(0pt,0pt);
        \filldraw[color_29791][even odd rule]
        (283pt, -135pt) -- (297.00pt, -135pt)
        (302pt, -135pt) -- (373.00pt, -135pt);
    \end{tikzpicture}
    \begin{picture}(-5,0)(2.5,0)
        \put(75,-279.28){\fontsize{16}{1}\usefont{T2A}{cmr}{b}{n}\selectfont\color{color_29791} Отчет по лабораторной работе № 1 по курсу }
        \put(120.45,-297.68){\fontsize{16}{1}\usefont{T2A}{cmr}{b}{n}\selectfont\color{color_29791} Системное программирование }
        \put(110.00,-328.08){\fontsize{14}{1}\usefont{T2A}{cmr}{b}{n}\selectfont\color{color_29791} "Изучение электронных справочников }
        \put(90.00,-346.08){\fontsize{14}{1}\usefont{T2A}{cmr}{b}{n}\selectfont\color{color_29791} системного программиста и эмуляторов ОС" }
        \put(90.00,-373.08){\fontsize{14}{1}\usefont{T2A}{cmr}{m}{n}\selectfont\color{color_29791} (есть ли дополнительные требования - ДА/HET) }
        \put(180.00,-414.5){\fontsize{12}{1}\usefont{T2A}{cmr}{m}{n}\selectfont\color{color_29791} (количество листов) }
        \put(195.00,-428.3){\fontsize{12}{1}\usefont{T2A}{cmr}{m}{n}\selectfont\color{color_29791} Вариант №  20 }
    \end{picture}
    \begin{tikzpicture}[overlay]
        \path(0pt,0pt);
        \filldraw[color_29791][even odd rule]
        (255.00pt, -430pt) -- (272.00pt, -430pt);
    \end{tikzpicture}
    \begin{picture}(-5,0)(2.5,0)
        \put(15.00,-455.9){\fontsize{12}{1}\usefont{T2A}{cmr}{m}{n}\selectfont\color{color_29791} 1. Команда ОС –       SORT }
        \put(15.00,-469.7){\fontsize{12}{1}\usefont{T2A}{cmr}{m}{n}\selectfont\color{color_29791} 2. Блок  ОС –       MCB }
        \put(15.00,-483.5){\fontsize{12}{1}\usefont{T2A}{cmr}{m}{n}\selectfont\color{color_29791} 3. Прерывание ОС –       33-9 }
        \put(160.00,-538.73){\fontsize{12}{1}\usefont{T2A}{cmr}{m}{n}\selectfont\color{color_29791} ИСПОЛНИТЕЛЬ:   }
        \put(160.00,-566.33){\fontsize{12}{1}\usefont{T2A}{cmr}{b}{n}\selectfont\color{color_29791} студент группы ИУ5-41}
        \put(345.00,-580.12){\fontsize{12}{1}\usefont{T2A}{cmr}{m}{n}\selectfont\color{color_29791} (подпись) }
    \end{picture}
    \begin{tikzpicture}[overlay]
        \path(0pt,0pt);
        \filldraw[color_29791][even odd rule]
        (325pt, -566.33pt) -- (460.00pt, -566.33pt);
    \end{tikzpicture}
    \begin{picture}(-5,0)(2.5,0)
        \put(160.00,-594.12){\fontsize{12}{1}\usefont{T2A}{cmr}{b}{n}\selectfont\color{color_29791} Цыпышев Т.А. }
        \put(325.00,-607.73){\fontsize{12}{1}\usefont{T2A}{cmr}{m}{n}\selectfont\color{color_29791} "13" \_ февраля \_ 2024 г. }
    \end{picture}
    \begin{tikzpicture}[overlay]
        \path(0pt,0pt);
        \filldraw[color_29791][even odd rule]
        (328pt, -610pt) -- (342.00pt, -610pt)
        (347pt, -610pt) -- (418.00pt, -610pt);
    \end{tikzpicture}
    \begin{picture}(-5,0)(2.5,0)
        \put(170.00,-711.775){\fontsize{12}{1}\usefont{T2A}{cmr}{m}{n}\selectfont\color{color_29791} Москва, МГТУ   -  2024 }
    \end{picture}
    \begin{tikzpicture}[overlay]
        \path(0pt,0pt);
        \filldraw[color_29791][even odd rule]
        (0pt, -725.575pt) -- (480.00pt, -725.575pt);
    \end{tikzpicture}
\end{titlepage}

\tableofcontents

\section{Цель выполнения лабораторной работы}
Целью выполнения лабораторной работы №1 является знакомство с специальными
электронными справочниками системного программиста и изучение принципов поиска
них информации по операционным системам, предназначенной для системного программиста.

\section{Порядок и условия проведения работы}
\begin{enumerate}
    \item Изучить в общем пособии (разделы 1,5,7) [6 – см. на сайте] разделы по: работе в режиме
          командной строки (КС) и по работе с файл-менеджерами (ФМ).
    \item В режиме КС запустить команды: DIR, HELP, DATE и SET. Продемонстрировать полученные
          навыки преподавателю ЛР.
    \item В программах FAR или VC (архив TASM3.ZIP – где 3 ЛР – на сайте) проверить: переключение
          каталогов, поиск файлов, создание и редактирование простого текстового файла, копирование
          и перемещение файлов, навигацию по меню. Продемонстрировать полученные навыки преподавателю ЛР.
    \item Скачать и развернуть справочники под эмулятором ОС (DOSBox v 7.4 – если на своем компьютере
          он не установлен, то скачать с сайта, установить, русифицировать и смонтировать виртуальный диск V:
          - см. ниже ) или в КС под CMD.EXE.
    \item Ответить устно на все контрольные вопросы ЛР.
    \item Изучить таблицу заданий для своего варианта
    \item Найти свою информацию по своему варианту и зафиксировать в отчете и изучить.
          СП 2024 год 2 курс ИУ5- 4-й сем. и 3-й курс ГУИМЦ 6-й семестр Большаков С.А.
    \item Изучить контрольные вопросы к ЛР и ответить на них.
    \item Показать ее преподавателю найденную информацию (демонстрация - отмечается в журнале)
    \item Оформить и распечатать отчет по своему варианту (шаблон в архиве этой ЛР).
    \item Защитить ЛР у преподавателя по контрольным вопросам (защита - отмечается в журнале
          и на титульном листе отчета).
\end{enumerate}

\section{Краткая инструкция по работе со справочником Help6}
 (Вставляются результаты поиска в виде копии текста из окна командной строки или снятых с
 экрана скриншотов. Даются пояснения и примеры.)

\section{Результаты поиска команды ОС}


\section{Результаты поиска прерывания ОС}
(Краткое назначение прерывания ОС Вставляются результаты поиска в виде копии текста из
окна командной строки или снятых с экрана скриншотов(без черного!!!). Даются пояснения и
примеры.)

\section{Результаты поиска управляющего блока ОС}
(Краткое назначение управляющего блока ОС Вставляются результаты поиска в виде копии
текста из окна командной строки или снятых с экрана скриншотов(без черного!!!). Даются
пояснения и примеры.)

\section{Выводы по ЛР}
 (Формулируются выводы, которые были сформулированы при выполнении работы.)

\end{document}