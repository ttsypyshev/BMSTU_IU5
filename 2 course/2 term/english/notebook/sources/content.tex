\part{Модуль №1}

\chapter{Семинар №1 08.02.24}

\section{Text 10A}

\subsection*{Light Beam at the Service of Humanity}
\allocation{(1)} Lasers often remind us of science fiction films and novels. Long ago
science fiction writers predicted the appearance of a mysterious fiery sword, which
would become an invincible weapon. The idea of using lasers as death rays has also
been employed by creators of such blockbusters as X-Men and Star Wars. And though the
ray laser gun still remains science fiction, putting a light beam at the service of
humanity is embodied in myriads5 of other uses based on laser technology.

\allocation{(2)} The word "laser" stands for “light amplification by stimulated
emission of radiation”. A laser, an optical device that strengthens light waves and
generates very intense beams of light, represents a powerful light source. The
difference between ordinary light and laser light is like the difference between the
ripples in your bathtub and huge waves on the sea. Until the invention of the laser,
the available light sources were generally neither monochromatic nor coherent and were
of relatively low intensity. The laser produces a well-directed, very intense beam
which is monochromatic, directional and coherent. Monochromatic means that all of the
light produced by the laser is of a single wavelength. Directional means that the beam
of light has a very low divergence. Light from conventional sources, such as a light
bulb or the sun, diverges, spreading in all directions. The intensity may be large at
the source, but it decreases rapidly as the observer moves away from the source. In
contrast, the laser output has a very small divergence and can maintain high beam
intensities over long ranges. Thus, relatively low power lasers are able to project
more energy at a single wavelength within a narrow beam than can be obtained from
much more powerful conventional light sources. Coherent means that the waves of
light are in phase with each other. A light bulb produces many wavelengths, that is
why its light is incoherent.

\allocation{(3)} The first discoveries that eventually brought us lasers were made at
the dawn of the 20th century. In 1917, Einstein laid the foundation for the laser when
he introduced the concept of stimulated emission. In 1954, Russian physicists
Nikolay Basov and Alexander Prokhorov working on the quantum oscillator created
the first microwave generator, laser’s predecessor, and described the theory of
its operation. At the same time, the idea how to generate stimulated emission at
microwave frequencies was also developed independently by American physicist
Charles Townes. He showed how this device, which was named a maser, could work.
A decade later, in 1964, all three were awarded with the Nobel Prize in physics for
their discoveries. In 1960, physicist from California Theodore Maiman demonstrated
the first ruby laser, which was considered the first successful light laser.
Other types of laser quickly followed: a gas laser and a semiconductor injection13
laser.

\allocation{(4)} Due to their remarkable properties lasers turned out to have all sorts
of useful applications in different fields from communications to medicine. In
science they are a great help in spectroscopy. They allow gigabytes of information
to be recorded. They can be used to focus relatively low wattage power to such high
intensity that it can be used to cut, heat or vaporise material. They have numerous
applications aboard spacecraft. Laser beams allow us to measure distances with
much greater accuracy than ever before. Laser-sighting devices are fitted to
military and police rifles to help soldiers hit their targets. Lasers can be used
as a defence against nuclear missiles and they may also be of use in thermonuclear
fusion reactors. Medicine and surgery have been transformed thanks to highly
accurate laser scalpels and laser diagnostics. In the arts, lasers can provide
fantastic displays of light.

\allocation{(5)} We are currently living in an era of intense development of lasers.
New types of lasers(chemical, excimer, semiconductor, free electron) are introduced
almost every year. New applications of lasers are constantly emerging. For example,
not long ago archaeologists uncovered a new vast network of cities and roads in the
thick jungles around the ancient Cambodian temple complex of Angkor Wat, implementing
an aerial survey using Lidar (light detection and ranging). Lidar might also prove
crucial in helping autonomous vehicles navigate. Lasers could have a huge impact
on the world of computing. For example, a silicon laser computer chip promises
faster data transfers. Laser developers say it could enable us to see people behind
walls, detect underground infrastructure without digging holes, and develop
navigation systems that do not rely on GPS.

\section{Exercise №1}
\subsection*{Match the words (1-6) with their definitions (a-f).
      Use a dictionary if necessary.}
\begin{enumerate}
      \item stimulated
      \item radiation
      \item acronym
      \item emission
      \item beam
      \item amplification \\
\end{enumerate}

\begin{enumerate}
      \item[a.] energy in the form of heat or light that you cannot see and
            which can be very harmful
      \item[b.] a word formed from the initial letters of other words
      \item[c.] the increase in volume of a signal
      \item[d.] a line of radiation or particles flowing in one direction
      \item[f.] the act of sending out gases or other substances
      \item[e.] made stronger or more active
\end{enumerate}

\subsection*{Solution}
\begin{enumerate}
      \item f
      \item a
      \item b
      \item e
      \item d
      \item c
\end{enumerate}

\section{Exercise №2}
\subsection*{In groups answer the questions.}
\begin{enumerate}
      \item What is a laser?
            \begin{enumerate}
                  \item a device which produces a very narrow beam of light useful in
                        many technologies
                  \item a process of optical amplification of light based on
                        radiation emission
                  \item both a and b
            \end{enumerate}
      \item What kind of word is the word ‘laser’?
            \begin{enumerate}
                  \item acronym
                  \item shortening
                  \item contraction
            \end{enumerate}
      \item Can you decode the word ‘laser’? (use the words from task 1)
            \begin{enumerate}
                  \item[] L... A... by Stimulated E... of R... .
            \end{enumerate}
\end{enumerate}

\subsection*{Solution}
\begin{enumerate}
      \item a
      \item a
      \item Light Amplification by Stimulated Emission of Radiation
\end{enumerate}

\section{Exercise №3}
\subsection*{Study the pictures below. Which of the following words and phrases
      refer to ordinary light (1) and which to laser light (2)?}
Coherent; its intensity decreases with distance; highly monochromatic;
it is not strictly monochromatic; organized; less intense; travels in one direction;
incoherent; highly intense; concentrated; travels in all directions; disorganized.

\subsection*{Solution}
\allocation{Ordinary light:}
\begin{itemize}
      \item disorganized
      \item its intensity decreases with distance
      \item it is not strictly monochromatic
      \item less intense, incoherent
      \item travels in all directions
\end{itemize}

\allocation{Laser light:}
\begin{itemize}
      \item organized
      \item coherent
      \item highly monochromatic
      \item travels in one direction
      \item highly intense
      \item concentrated
\end{itemize}

\section{Exercise №6}
\allocation{[Устно]}

\subsection*{Read the text again and answer the following questions.}
\begin{enumerate}
      \item Why can we say that lasers were predicted long before their invention?
      \item What is a laser? What does the word ‘laser’ mean?
      \item What kind of beam do lasers have?
      \item What do we mean by the words ‘monochromatic, directional, and coherent’
            when we refer to laser light?
      \item Why is the light from the laser so concentrated?
      \item Who proposed the theoretical possibility of the process that made lasers
            possible?
      \item Who created the first microwave generator?
      \item Who demonstrated the first successful light laser?
      \item What laser types are mentioned in the text?
      \item Do you agree with the author’s opinion that lasers have found myriads
            of useful applications? What examples do you think best prove this point?
      \item While reading this text, which uses of lasers surprised you the most?
      \item Can you think of an example of a laser device or technology that you
            have used or are using?
\end{enumerate}

\section{Exercise №7}
\allocation{[Устно]}

\subsection*{Read the statements and decide which of them are true (T) and
      which are false (F) according to text 10A. Explain why.}
\begin{enumerate}
      \item The word ‘laser’ means microwave amplification by stimulated emission
            of radiation.
      \item Laser was invented at the dawn of the 20th century.
      \item Albert Einstein was the first inventor of a laser.
      \item Laser came into existence only in the second half of the 20-th century.
      \item Unfortunately most of the applications of a laser proved to be unattainable
            in the real world.
      \item The use of lasers in thermonuclear fusion reactors may be the key
            to the future.
      \item Laser weapons are widely used by the military.
      \item In medicine lasers can be used for various surgical procedures.
      \item Very few inventions can match the impact of the laser’s invention.
      \item Laser technology has a promising future.
\end{enumerate}

\chapter{Домашнее задание №1 15.02.24}

\section{Exercise №5}
\subsection*{Find the words and phrases in the text which have the following meanings.}
\begin{enumerate}
      \item[§] \allocation{1}
            \begin{enumerate}
                  \item[1.] \textbf{a verb:} to make someone remember something
                  \item[2.] \textbf{a verb:} to use a particular idea or method
                  \item[3.] \textbf{a verb:} to continue to be in the same state or condition
                  \item[4.] \textbf{a verb:} to express clearly or show the importance of an
                        idea or principle
            \end{enumerate}
      \item[§] \allocation{2}
            \begin{enumerate}
                  \item[5.] \textbf{a noun:} the product of making larger or greater in amount
                        or intensity
                  \item[6.] \textbf{a noun:} the result of sending something out (e.g. gas or
                        heat)
                  \item[7.] \textbf{a verb:} to make stronger
                  \item[8.] \textbf{a noun:} the point from which something begins
                  \item[9.] \textbf{an adverb:} in relation to something else
                  \item[10.] \textbf{a noun:} a shining line of light
                  \item[11.] \textbf{an -ing form of a verb:} covering a large area
                  \item[12.] \textbf{a verb:} to go down to a lower level
                  \item[13.] a phrase used when you are comparing objects or situations
                        and saying that they are completely different
                  \item[14.] the amount of something (energy, work, information)
                        produced by a machine
            \end{enumerate}
      \item[§] \allocation{3}
            \begin{enumerate}
                  \item[15.] \textbf{an adverb:} after a long time
                  \item[16.] \textbf{a verb phrase:} to provide something (idea, principle) from
                        which another thing can develop
                  \item[17.] \textbf{a verb:} to give someone a prize for something they have
                        done
            \end{enumerate}
      \item[§] \allocation{4, 5}
            \begin{enumerate}
                  \item[18.] \textbf{a prepositional phrase:} because of or thanks to
                  \item[19.] \textbf{an adjective:} unusual or surprising and therefore
                        deserving attention
                  \item[20.] \textbf{a verb:} to have a particular result, especially one that
                        you didn’t expect
                  \item[21.] \textbf{a verb:} to write something (e.g. information) down
                  \item[22.] \textbf{a verb:} to change into a vapour
                  \item[23.] \textbf{a verb:} to find the size, length or amount of something
                  \item[24.] \textbf{a noun:} the quality of being correct and true
                  \item[25.] \textbf{a verb:} to carry out
                  \item[26.] \textbf{a verb phrase:} to be of primary importance
            \end{enumerate}
\end{enumerate}

\subsection*{Solution}
\begin{enumerate}
      \item remind
      \item employ
      \item remain
      \item embody
      \item amplification
      \item emission
      \item strengthen
      \item source
      \item ?
      \item laser
      \item spreading
      \item decrease
      \item the difference between
      \item intensity
      \item eventually
      \item -
      \item award
      \item due to
      \item remarkable
      \item -
      \item record
      \item -
      \item measure
      \item accuracy
      \item -
\end{enumerate}

\section{Exercise №8}
\subsection*{Complete the sentences using the information from the text without
      looking into the text.}
\begin{enumerate}
      \item The word laser is an acronym standing for \underline{\hspace{2cm}}.
      \item Laser light differs from ordinary light due to its \underline{\hspace{2cm}}.
      \item Russian physicists Nikolay Basov and Alexander Prokhorov created
            \underline{\hspace{2cm}} while working on \underline{\hspace{2cm}}.
      \item In 1960, physicist from California Theodore Maiman demonstrated
            \underline{\hspace{2cm}}.
      \item Lasers turned out to have myriads of uses, from \underline{\hspace{2cm}}
            to \underline{\hspace{2cm}}.
      \item In science lasers provide great assistance with \underline{\hspace{2cm}}.
      \item Laser-sighting devices are fitted to \underline{\hspace{2cm}} to help
            soldiers \underline{\hspace{2cm}}.
      \item Today new applications of lasers are \underline{\hspace{2cm}}.
      \item Not long ago archaeologists uncovered \underline{\hspace{2cm}} using Lidar.
      \item In computing lasers could have \underline{\hspace{2cm}}.
\end{enumerate}

\subsection*{Solution}
\begin{enumerate}
      \item Laser stands for "light amplification by stimulated emission of radiation." \\
            "The word 'laser' stands for 'light amplification by stimulated emission
            of radiation'." \allocation{(Paragraph 2)}
      \item Laser light is different from ordinary light because it's monochromatic,
            directional, and coherent. \\
            "The laser produces a well-directed, very intense beam which is
            monochromatic, directional and coherent." \allocation{(Paragraph 2)}
      \item Basov and Prokhorov created the precursor to the laser while studying the
            quantum oscillator. \\
            "In 1954, Russian physicists Nikolay Basov and Alexander Prokhorov working
            on the quantum oscillator created the first microwave generator, laser’s
            predecessor." \allocation{(Paragraph 3)}
      \item In 1960, Maiman demonstrated the first ruby laser. \\
            "In 1960, physicist from California Theodore Maiman demonstrated the first
            ruby laser." \allocation{(Paragraph 3)}
      \item Lasers have diverse applications, from medicine to communications. \\
            "Due to their remarkable properties, lasers turned out to have all sorts
            of useful applications in different fields from communications to medicine."
            \allocation{(Paragraph 4)}
      \item Lasers assist greatly in scientific spectroscopy. \\
            "In science they are a great help in spectroscopy."
            \allocation{(Paragraph 4)}
      \item Laser-sighting aid devices soldiers in hitting targets. \\
            "Laser-sighting devices are fitted to military and police rifles to help
            soldiers hit their targets." \allocation{(Paragraph 4)}
      \item New laser applications are continually emerging. \\
            "New applications of lasers are constantly emerging."
            \allocation{(Paragraph 5)}
      \item Archaeologists found ancient structures using Lidar. \\
            "Not long ago archaeologists uncovered a new vast network of cities and
            roads in the thick jungles around the ancient Cambodian temple complex
            of Angkor Wat, implementing an aerial survey using Lidar."
            \allocation{(Paragraph 5)}
      \item Lasers in computing could greatly improve data transfer speeds. \\
            "For example, a silicon laser computer chip promises faster data
            transfers." \allocation{(Paragraph 5)}
\end{enumerate}

\chapter{Семинар №2 15.02.24}

\subsection*{Vocabulary}
\allocation{Text 10 C}

confirm (v)
consumer (n)
controversial (n)
cure (v) diseases
far-reaching (adj)
invisible (adj)
lack (v, n)
make (v) sense (n)
particle (n)
photonics (n)
underpin (v)

\allocation{Выпилнять -}
\begin{enumerate}
      \item carry out
      \item implement
      \item do
      \item make
      \item created
      \item turn out
      \item produce/manufacture
      \item run
      \item execute
\end{enumerate}

\section{Exercise №21}
\subsection*{Fill in the gaps with the words from Exercise 20 in the right form. The
      first letters are given. Translate the sentences into Russian.}
\allocation{Example:}A d\underline{\hspace{2cm}} microphone is the one that picks up
sound from a specific area. → A directional microphone is the one that picks up sound
from a specific area.

1. All our laboratories are f\underline{\hspace{2cm}} with computers and high-speed
internet access. 2. Some people think that electromagnetic r\underline{\hspace{2cm}}
from our mobiles is harmful. 3.Climatologists say that the e\underline{\hspace{2cm}}
of greenhouse gases contributes to global warming. 4. Melatonin, a hormone involved in
controlling our sleep, is s\underline{\hspace{2cm}} by darkness. 5. The sky cleared up
and a b\underline{\hspace{2cm}} of sunlight shone in through the window. 6. If we don’t
modernise, the o\underline{\hspace{2cm}} from the factory will decrease. 7. Today it is
r\underline{\hspace{2cm}} easy to find any information thanks to the Internet. 8. The
20th century was r\underline{\hspace{2cm}} for its inventions. 9. The Nobel Prizes are
a\underline{\hspace{2cm}} annually from a fund created for that purpose by the Swedish
inventor and industrialist Alfred Bernhard Nobel. 10. A school’s success can be
m\underline{\hspace{2cm}} in terms of the number of pupils who got into university.
11. Scientists need to be very careful about the a\underline{\hspace{2cm}} of their
research results. 12. Reforms should be i\underline{\hspace{2cm}} that will allow the
company to stay competitive. 13. Our students’ ideas are e\underline{\hspace{2cm}} in
new classroom rules. 14. Exercising regularly is the best way to s\underline{\hspace{2cm}}
your immune system. 15. D\underline{\hspace{2cm}} to the large volume of letters he is
unable to answer personally. 16. Sometimes things don't t\underline{\hspace{2cm}} out
the way we think they're going to.

\subsection*{Solution}
\begin{enumerate}
      \item All our laboratories are \textbf{fitted} with computers and high-speed
            internet access.
      \item Some people think that electromagnetic \textbf{radiation} from our
            mobiles is harmful.
      \item Climatologists say that the \textbf{emission} of greenhouse gases
            contributes to global warming.
      \item Melatonin, a hormone involved in controlling our sleep, is
            \textbf{stimulated} by darkness.
      \item The sky cleared up and a \textbf{beam} of sunlight shone in through
            the window.
      \item If we don’t modernise, the \textbf{output} from the factory will
            decrease.
      \item Today it is \textbf{relatively} easy to find any information thanks
            to the Internet.
      \item The 20th century was \textbf{remarkable} for its inventions.
      \item The Nobel Prizes are \textbf{awarded} annually from a fund created
            for that purpose by the Swedish inventor and industrialist Alfred Bernhard
            Nobel.
      \item A school’s success can be \textbf{measured} in terms of the number
            of pupils who got into university.
      \item Scientists need to be very careful about the \textbf{accuracy} of
            their research results.
      \item Reforms should be \textbf{implemented} that will allow the company
            to stay competitive.
      \item Our students’ ideas are \textbf{embodied} in new classroom rules.
      \item Exercising regularly is the best way to \textbf{strengthen} your
            immune system.
      \item \textbf{Due} to the large volume of letters he is unable to answer
            personally.
      \item Sometimes things don't \textbf{turn} out the way we think they're
            going to.
\end{enumerate}

\section{Exercise №22}
\subsection*{Guess the word by its definition. Use text 10A word list to help you.}
\begin{enumerate}
      \item If two or more waves have the same phase we call this light
            c\underline{\hspace{2cm}}.
      \item When a liquid changes into gas we can say that it v\underline{\hspace{2cm}}.
      \item M\underline{\hspace{2cm}} colour refers to a colour scheme that is
            comprised of variations of one colour.
      \item If one thing is in c\underline{\hspace{2cm}} to another, it is very
            different from it.
      \item If something e\underline{\hspace{2cm}} heat, light or gas, it produces it
            and sends out by means of a physical or chemical process.
      \item If someone r\underline{\hspace{2cm}} you of a fact or event that you
            already know about, they say something which makes you think about it.
      \item If someone or something r\underline{\hspace{2cm}} in a particular state
            or condition, they stay in that state or condition and do not change.
      \item You use the conjunction n\underline{\hspace{2cm}} n\_\_ when you are
            talking about two or more things that are not true or that do not happen.
      \item Laser light is very d\underline{\hspace{2cm}} which means that it is
            extremely narrow and is emitted in one direction.
      \item A l\underline{\hspace{2cm}} is a device that emits light through a process
            of optical amplification based on the stimulated emission of
            electromagnetic radiation
\end{enumerate}

\subsection*{Solution}
\begin{enumerate}
      \item If two or more waves have the same phase we call this light
            \textbf{coherent}.
      \item When a liquid changes into gas we can say that it \textbf{vaporizes}.
      \item \textbf{Monochromatic} colour refers to a colour scheme that is
            comprised of variations of one colour.
      \item If one thing is in \textbf{contrast} to another, it is very different
            from it.
      \item If something \textbf{emits} heat, light, or gas, it produces it and
            sends out by means of a physical or chemical process.
      \item If someone \textbf{reminds} you of a fact or event that you already
            know about, they say something which makes you think about it.
      \item If someone or something \textbf{remains} in a particular state or
            condition, they stay in that state or condition and do not change.
      \item You use the conjunction \textbf{neither nor} when you are talking about
            two or more things that are not true or that do not happen.
      \item Laser light is very \textbf{directional} which means that it is
            extremely narrow and is emitted in one direction.
      \item A \textbf{laser} is a device that emits light through a process of
            optical amplification based on the stimulated emission of electromagnetic
            radiation.
\end{enumerate}

\section{Exercise №23}
\subsection*{Match the words with numbers (1-10) with the words with letters (a-j) to
      make up word collocations. Explain the meaning of these expressions and try to
      recall how they were used in text 10A.}
\allocation{Example:}to lay + the foundation for something means ‘to provide conditions
that will make something possible’, e.g. Einstein laid the foundation for the laser.

\begin{enumerate}
      \item to lay
      \item to prove
      \item to measure
      \item light
      \item stimulated
      \item to decrease
      \item conventional
      \item to spread
      \item remarkable
      \item to vaporise \\
\end{enumerate}

\begin{enumerate}
      \item[a.] crucial
      \item[b.] amplification
      \item[c.] emission
      \item[d.] source
      \item[e.] properties
      \item[f.] the foundation
      \item[g.] distances
      \item[h.] material
      \item[i.] rapidly
      \item[j.] in all directions
\end{enumerate}

\subsection*{Solution}
\begin{enumerate}
      \item f
      \item a
      \item g
      \item c
      \item b
      \item i
      \item h
      \item j
      \item e
      \item d
\end{enumerate}

\section{Exercise №24}
\subsection*{Complete each sentence with the correct word to make up a word collocation
      from Exercise 23.}
1. Buying the works of his contemporary artists, Pavel Tretiakov laid the
\underline{\hspace{2cm}} for one of the world’s greatest collections of Russian
paintings. 2. Learning the facts about how COVID-19 emerged may \underline{\hspace{2cm}}
crucial for preventing future outbreaks. 3.Before electricity was invented the
\underline{\hspace{2cm}} sources of light were candles or oil lamps. 4. The use of
lasers to \underline{\hspace{2cm}} distances is based on the principle of reflection of
a laser beam. 5. One of the problems the inventors of a laser faced was how to create
conditions for light \underline{\hspace{2cm}}. 6. Stimulated \underline{\hspace{2cm}}
of radiation is the first and necessary condition for laser light generation, but it is
not the only one. 7. Marketers know that the value of data \underline{\hspace{2cm}}
rapidly over time. 8. The fire was spreading out in all \underline{\hspace{2cm}} because
of the hot weather and strong wind. 9.The number of articles about new materials with
some remarkable \underline{\hspace{2cm}} has increased in the last years. 10. Processing
materials with a laser beam allows engineers to cut, drill, weld, and even
\underline{\hspace{2cm}} different materials.

\subsection*{Solution}
\begin{enumerate}
      \item Buying the works of his contemporary artists, Pavel Tretiakov laid the
            \textbf{foundation} for one of the world’s greatest collections of
            Russian paintings.
      \item Learning the facts about how COVID-19 emerged may \textbf{prove} crucial
            for preventing future outbreaks.
      \item Before electricity was invented the \textbf{primary} sources of light
            were candles or oil lamps.
      \item The use of lasers to \textbf{measure} distances is based on the
            principle of reflection of a laser beam.
      \item One of the problems the inventors of a laser faced was how to create
            conditions for light \textbf{amplification}.
      \item Stimulated \textbf{emission} of radiation is the first and necessary
            condition for laser light generation, but it is not the only one.
      \item Marketers know that the value of data \textbf{decreases} rapidly over
            time.
      \item The fire was spreading out in all \textbf{directions} because of the
            hot weather and strong wind.
      \item The number of articles about new materials with some remarkable
            \textbf{properties} has increased in the last years.
      \item Processing materials with a laser beam allows engineers to cut, drill,
            weld, and even \textbf{vaporise} different materials.
\end{enumerate}

\section{Exercise №25}
\subsection*{Match the words with the correct definition or synonym of each word as it
      is used in text 10B.}

\begin{enumerate}
      \item photon
      \item a partial (mirror)
      \item back and forth
      \item power source
      \item to emit
      \item to reflect
      \item to absorb
      \item to bounce
      \item concentrated
      \item hence
      \item to inject \\
\end{enumerate}

\begin{enumerate}
      \item[a.] to introduce (e.g. a fluid) into something forcefully
      \item[b.] a unit of energy that carries light and has zero mass
      \item[c.] the device that supplies energy
      \item[d.] to return or throw back (e.g. light or sound)
      \item[e.] so, thus
      \item[f.] to move away from a surface
      \item[g.] not complete, limited
      \item[h.] to send out (e.g. light or gas)
      \item[i.] to take a liquid in
      \item[j.] focused
      \item[k.] moving first in one direction and then in the opposite one
\end{enumerate}

\subsection*{Solution}
\begin{enumerate}
      \item b
      \item g
      \item k
      \item c
      \item h
      \item d
      \item i
      \item f
      \item j
      \item e
      \item a
\end{enumerate}

\section{Exercise №27}
\subsection*{Find the opposites. Match the words in column A with their opposites in
      column B.}
\allocation{Example:}\allocation{to evolve} is the opposite of \allocation{to decrease,
      worsen}.

\allocation{ A. }
\begin{enumerate}
      \item to increase
      \item to absorb
      \item stimulated emission
      \item inside
      \item output
      \item to get excited
      \item to flash on
      \item to inject (energy)
      \item coherent
      \item organised
      \item to strengthen
      \item to implement \\
\end{enumerate}

\allocation{ B. }
\begin{enumerate}
      \item[a.] input
      \item[b.] to emit (energy)
      \item[c.] disorganised
      \item[d.] to decrease
      \item[e.] outside
      \item[f.] incoherent
      \item[g.] to reflect
      \item[h.] to calm down
      \item[i.] to weaken
      \item[j.] spontaneous emission
      \item[k.] to prevent, delay
      \item[l.] to flash off
\end{enumerate}

\subsection*{Solution}
\begin{enumerate}
      \item d
      \item g
      \item j
      \item e
      \item a
      \item h
      \item l
      \item k
      \item f
      \item c
      \item i
      \item l
\end{enumerate}

\chapter{Домашнее задание №2 22.02.24}

\section{Exercise №28}
\subsection*{Rewrite each sentence replacing the words in italics by their opposites.
      Use the words in brackets so that the new sentence has the meaning opposite to
      the first sentence.}
\allocation{Example:}The production efficiency is the result of good work. (bad). → The
production inefficiency is the result of poor work.

1. Black surfaces absorb more light than other colours. \textbf{(white)} 2. In spring wild birds
increase in number in Moscow region. \textbf{(in autumn)} 3. Spontaneous emission takes place
without interaction with other photons. \textbf{(when photon emission is triggered by other
      photons)} 4. It feels really warm inside on a winter morning. \textbf{(cold)} 5. A mouse and a
keyboard are the examples of input devices. \textbf{(amonitor and a printer)} 6. For the system
(such as an atom or a molecule) to calm down, you need to make its energy level lower.
\textbf{(higher than the ground state)} 7. If you want to take a picture when it is dark you
should choose a ‘flash on’ mode. \textbf{(in daylight)} 8. Ordinary light unlike laser light is
incoherent and disorganized. \textbf{(laser light)} 9. The committee agreed that it was necessary
to implement the changes recommended in the report. 10. Our attention is weakened by
stress. \textbf{(mindfulness)}

\subsection*{Solution}
\begin{enumerate}
      \item White surfaces reflect more light than other colors. \textbf{(black)}
      \item In autumn wild birds decrease in number in Moscow region. \textbf{(in spring)}
      \item Spontaneous absorption takes place without interaction with other photons.
            \textbf{(when photon absorption is triggered by other photons)}
      \item It feels really cold inside on a winter morning. \textbf{(warm)}
      \item A monitor and a printer are the examples of output devices. \textbf{(a mouse and
                  a keyboard)}
      \item For the system (such as an atom or a molecule) to excite, you need to make
            its energy level higher. \textbf{(lower than the ground state)}
      \item If you want to take a picture when it is bright you should choose a ‘flash
            off’ mode. \textbf{(in darkness)}
      \item Laser light, unlike ordinary light, is coherent and organized. \textbf{(ordinary
                  light)}
      \item The committee agreed that it was unnecessary to implement the changes
            recommended in the report.
      \item Our attention is strengthened by mindfulness.
\end{enumerate}

\section{Exercise №29}
\subsection*{Use the word given in brackets to form a word which fits in the gap.}
1. The name ‘laser’ stands for Light \underline{\hspace{2cm}} by stimulated emission of
radiation. (amplify) 2. Many enjoy the mental \underline{\hspace{2cm}} of a challenging
job. (stimulate) 3.Words \underline{\hspace{2cm}} thoughts and feelings.( embodiment)
4. Difficulties \underline{\hspace{2cm}} the mind, as labour does the body. (strong)
5. Laws controlling the \underline{\hspace{2cm}} of greenhouse gases should be
introduced. (emit) 6. Truth is the \underline{\hspace{2cm}} of all knowledge. (found)
7. A cloud is a mass of \underline{\hspace{2cm}} in the sky. (vaporise) 8. A graphical
\underline{\hspace{2cm}} of the experiment results is required. (represent) 9. Do you
think mobile phones emit \underline{\hspace{2cm}} ? (radiate) 10. If a text is
\underline{\hspace{2cm}}, it means that it is well planned, clear and logical.
(coherence)

\subsection*{Solution}
\begin{enumerate}
      \item The name ‘laser’ stands for Light Amplification by stimulated emission of
            radiation.
      \item Many enjoy the mental stimulation of a challenging job.
      \item Words embody thoughts and feelings.
      \item Difficulties strengthen the mind, as labour does the body.
      \item Laws controlling the emission of greenhouse gases should be introduced.
      \item Truth is the foundation of all knowledge.
      \item A cloud is a mass of vapor in the sky.
      \item A graphical representation of the experiment results is required.
      \item Do you think mobile phones emit radiation?
      \item If a text is coherent, it means that it is well planned, clear and logical.
\end{enumerate}

\section{Exercise №30}
\subsection*{Read the text and fill in the gaps with the following words in the
      appropriate form.}
concentrated, coherence, weapon, monochromatic, stands for, emission, beam, to encode
and transmit, sophisticated, represents, hence, to vaporise

In «The War of the Worlds» written before the turn of the last century, H. Wells told
a fantastic story of how Martians almost invaded our Earth. Their \allocation{1}
\underline{\hspace{2cm}} was a mysterious «sword of heat». Today Wells’ sword of heat
has come to reality in the laser. The name \allocation{2}\underline{\hspace{2cm}} light
amplification by stimulated \allocation{3}\underline{\hspace{2cm}} of radiation. Laser,
one of the most \allocation{4} \underline{\hspace{2cm}} inventions of man, produces an
intensive \allocation{5}\underline{\hspace{2cm}} of light of a very pure single colour.
It \allocation{6}\underline{\hspace{2cm}} the fulfillment of one of the humankind’s
oldest dreams of technology to provide a light beam intensive enough \allocation{7}
\underline{\hspace{2cm}} the hardest materials. There are few materials which are not
suited for laser processing, \allocation{8}\underline{\hspace{2cm}} laser treatment of
materials has become an important technique lately. The laser’s most important
potential may be its use in communications. We send and receive the data, video and
other information, using lasers \allocation{9}\underline{\hspace{2cm}} the data at rates 10 to 100
times faster than radio, because lasers can generate a very intense, \allocation{10}
\underline{\hspace{2cm}}, highly parallel and \allocation{11}\underline{\hspace{2cm}}
beam and \allocation{12}\underline{\hspace{2cm}} is a very important property of laser light.

\subsection*{Solution}
In «The War of the Worlds» written before the turn of the last century, H. Wells told
a fantastic story of how Martians almost invaded our Earth. Their \allocation{1}
\textbf{weapon} was a mysterious «sword of heat». Today Wells’ sword of heat has come
to reality in the laser. The name \allocation{2}\textbf{'laser' stands for} light
amplification by stimulated \allocation{3}\textbf{emission} of radiation. Laser, one of
the most \allocation{4}\textbf{sophisticated} inventions of man, produces an intensive
\allocation{5}\textbf{monochromatic} beam of light of a very pure single color. It
\allocation{6}\textbf{represents} the fulfillment of one of humankind’s oldest dreams
of technology to provide a light beam intensive enough \allocation{7}
\textbf{to vaporise} the hardest materials. There are few materials which are not suited
for laser processing, \allocation{8}\textbf{hence} laser treatment of materials has
become an important technique lately. The laser’s most important potential may be its
use in communications. We send and receive data, video, and other information, using
lasers \allocation{9}\textbf{to encode and transmit} the data at rates 10 to 100 times
faster than radio because lasers can generate a very intense, \allocation{10}
\textbf{concentrated}, highly parallel, and \allocation{11}\textbf{coherent} beam, and
\allocation{12}\textbf{coherence} is a very important property of laser light.

\section{Exercise №31}
\subsection*{Work in groups. Choose 5-7 words from Module 10 Word list and prepare a
      short news story to tell your group using these words. Ask your listeners to write
      down the words while they listen to your story. Compare your lists.}

\section{Exercise №32}
\subsection*{Summarise the text in English paying attention to the linking words and
      phrases}

\subsection*{Solution}
The text discusses the invention, properties, and use of lasers. Firstly, it outlines
the history of laser invention, then transitions to its properties. Thirdly, it
discusses the types of existing lasers and concludes by examining practical laser
applications in various fields. While there is no definitive answer to who invented
the laser, several scientists contributed to its creation. Despite initial expectations
of mainly military use, lasers have found widespread application in areas such as
warfare and medicine due to their ability to produce an extremely narrow beam of
light. The text also highlights the role of lasers in computer science. Overall,
lasers have a broad range of applications, from surgical procedures to vehicle speed
control devices. In conclusion, the fact that "death rays" did not become a reality
is ultimately beneficial.

The concept of "photonics" emerged in the late 20th century and has become part of
everyday life, encompassing technologies like fiber optic communication lines,
flat-screen TVs and computer monitors, smartphones, and more. Laser communication
offers high quality, greater bandwidth, and strict confidentiality. Both lasers and
fiber optics have become vital components of many industries, and their combined
potential is rapidly expanding. Semiconductor lasers are used in fiber-optic
telecommunication systems. Research in this field has led to advancements in areas
such as quantum electronics, fiber optics, quantum optics, laser physics, laser
chemistry, and more. The term "photonics" encompasses all these scientific and
technical areas.

\chapter{Семинар №3 22.02.24}

\section{Text 10С}

\subsection*{Photonics}
Photonics is the science and technology of generating, controlling, and detecting
photons, which are particles of light. Photonics underpins technologies of daily life
from smartphones to laptops, medical instruments and lighting technology. The 21st
century will depend as much on photonics as the 20th century depended on electronics.
Photonics is the science of light, it is the technology of generating, controlling, and
detecting light waves and photons, which are \allocation{1.}\underline{\hspace{2cm}}
of light. The characteristics of the waves and photons can be used to \allocation{2.}
\underline{\hspace{2cm}} the universe, to \allocation{3.}\underline{\hspace{2cm}}
diseases, and even to solve crimes. Scientists have been studying light for hundreds
of years. The colors of the rainbow are only a small part of the entire light wave
range, called the electromagnetic spectrum. Photonics explores a wider variety of
\allocation{4.}\underline{\hspace{2cm}}, from gamma rays to radio, including X-rays,
UV and infrared light. It was only in the 17th century that Sir Isaac Newton showed
that white light is made of different colors of light. At the beginning of the 20th
century, Max Planck and later Albert Einstein \allocation{5.}\underline{\hspace{2cm}}
that light was a wave as well as a particle, which was a very \allocation{6.}
\underline{\hspace{1.5cm}} theory at the time. How can light be two completely
different things at the same time? Experimentation later \allocation{7.}
\underline{\hspace{2cm}} this duality in the nature of light. The word Photonics
appeared around 1960, when the laser was invented by Theodore Maiman. Even if we
cannot see the \allocation{8.}\underline{\hspace{1.3cm}} electromagnetic spectrum,
\allocation{9.}\underline{\hspace{2cm}} light waves are a part of our everyday life.
Photonics is everywhere; in consumer electronics (barcode scanners, DVD players, remote
TV control), telecommunications (internet), health (eye surgery, medical instruments),
manufacturing industry (laser cutting and machining), defense and security (infrared
camera, remote sensing), entertainment (holography, laser shows), etc. All around the
world, scientists, engineers and technicians perform \allocation{10.}
\underline{\hspace{2cm}} research surrounding the field of Photonics. The science of
light is also actively taught in classrooms and museums where teachers and educators
share their passion for this field with young people and the general public. Photonics
opens a world of unknown and \allocation{11.}\underline{\hspace{2cm}} possibilities
limited only by a lack of imagination.

\subsection*{How Light Really Works}
Once we understand how atoms take in and give out energy, the science of light
\allocation{12.}\underline{\hspace{1.1cm}} in a very interesting new way. Think about
mirrors, for example. When you look at a mirror and see your face reflected, what's
actually going on? Light (maybe from a window) is hitting your face and \allocation{13.}
\underline{\hspace{2cm}} into the mirror. Inside the mirror, atoms of silver (or another
very reflective metal) are catching the \allocation{14.}\underline{\hspace{2cm}} light
energy and becoming \allocation{15.}\underline{\hspace{2cm}}. That makes them unstable,
so they throw out new \allocation{16.}\underline{\hspace{2cm}} of light that travel back
out of the mirror towards you. In effect, the mirror is playing throw and catch with
you using photons of light as the balls!
The same idea can help us explain things like photocopiers and \allocation{17.}
\underline{\hspace{2cm}} (flat sheets of the chemical element silicon that turn
sunlight into electricity). Have you ever wondered why solar panels look black even
when they're in full sunlight? That's because they're \allocation{18.}
\underline{\hspace{2cm}} back little or none of the light that falls on them and
\allocation{19.}\underline{\hspace{2cm}} all the energy instead. (Things that are black
absorb light, and reflect little or none, while things that are white reflect virtually
all the light that falls on them, and absorb little or none. That's why it's best to
wear white clothes on a hot day.) Where does the energy go in a solar panel if it's not
reflected? If you shine sunlight onto the solar \allocation{20.}\underline{\hspace{2cm}}
in a solar panel, the atoms of silicon in the cells catch the energy from the sunlight.
Then, instead of producing new photons, they produce a flow of electricity instead
through what's known as the \allocation{21.}\underline{\hspace{2cm}} (or photovoltaic)
effect. In other words, the incoming solar energy (from the Sun) is \allocation{22.}
\underline{\hspace{2cm}} to outgoing electricity.

\subsection*{Solution}
\begin{enumerate}
      \item particles
      \item explore
      \item cure
      \item wavelengths
      \item suggested
      \item controversial
      \item confirmed
      \item entire
      \item invisible
      \item cutting-edge
      \item far-reaching
      \item makes sense
      \item bouncing
      \item incoming
      \item excited
      \item photons
      \item cells
      \item reflecting
      \item absorb
      \item cells
      \item photoelectric
      \item converted
\end{enumerate}

\section{Exercise №17}
\subsection*{Read the text again and answer the following questions.}
\begin{enumerate}
      \item What does photonics study?
      \item How could the characteristics of waves and photons be put to practical use?
      \item What kind of waves does photonics explore?
      \item What discoveries did the scientists of the past make while studying light?
      \item What does ‘duality of light’ mean?
      \item Why can we say that photonics is everywhere?
      \item Do you agree with the opinion that photonics is really important today?
      \item What happens when you look at a mirror?
      \item Why do solar panels look black in full sunlight?
      \item Why is it best to wear white clothes on a hot day?
      \item What happens to the solar energy in a solar panel?
      \item Do you think pursuing a career in Photonics could be exciting and rewarding?
\end{enumerate}

\subsection*{Solution}
\begin{enumerate}
      \item Photonics studies "generating, controlling, and detecting photons, which
            are particles of light."
      \item The characteristics of waves and photons can be practically used to
            "explore the universe, to cure diseases, and even to solve crimes."
      \item Photonics explores "a wider variety of wavelengths, from gamma rays to
            radio, including X-rays, UV and infrared light."
      \item Scientists in the past discovered that "white light is made of different
            colors of light" and later confirmed that "light was a wave as well
            as a particle."
      \item The "duality in the nature of light" refers to the fact that "light was
            a wave as well as a particle."
      \item Photonics is everywhere because "light waves are a part of our everyday
            life."
      \item Yes, photonics is important today as it underpins technologies crucial
            to daily life and offers possibilities for advancements in various fields.
      \item When you look at a mirror, "atoms of silver... are catching the incoming
            light energy and becoming excited," which then "throw out new photons of
            light that travel back out of the mirror towards you."
      \item Solar panels look black in full sunlight because they're "reflecting back
            little or none of the light that falls on them and absorbing all the energy
            instead."
      \item It's best to wear white clothes on a hot day because "white reflect
            virtually all the light that falls on them, and absorb little or none."
      \item Solar panels convert sunlight into electricity through "the photoelectric
            (or photovoltaic) effect."
      \item Pursuing a career in Photonics could be exciting and rewarding as it
            involves "cutting-edge research" and offers "far-reaching possibilities
            limited only by a lack of imagination."
\end{enumerate}

\section{Exercise №18}
\subsection*{Listen to a short lecture about lasers and decide which of the following
      points below the speaker talks about.}

\href{https://www.youtube.com/watch?v=oUEbMjtWc-A}{https://www.youtube.com/watch?v=oUEbMjtWc-A}

\begin{itemize}
      \item The unique characteristics of laser light.
      \item How laser light is different from ordinary light
      \item How lasers are used in the military.
      \item How lasers are useful in eye surgery.
      \item How laser was invented.
      \item Different types of lasers.
      \item The operation of a ruby laser.
      \item How electronic transitions create stimulated emission.
      \item How the light becomes intensified and narrowed in wavelength inside a laser
            cavity.
      \item Innovations and improvements in laser technology.
\end{itemize}

\allocation{Useful words:} hallmark - клеймо, проба, признак; range finder - дальномер;
vitreous humor - стекловидное тело; tour de force - проявление таланта, мастерства;
xenon arc - (электрическая) дуга в атмосфере ксенона; flash lamp - импульсная лампа,
the crests and troughs - точки подъёма и спада; resonant cavity - резонансная полость;
avalanche – лавина; decay- распад.

\subsection*{Solution}
\begin{itemize}
      \item The unique characteristics of laser light, enabling technologies like range
            finders, optical communications, bar code scanners, and medical procedures.
      \item The use of lasers in eye surgery, emphasizing the precision and safety of
            green laser light.
      \item The invention of the laser by Ted Maiman in 1960.
      \item The operation of a ruby laser, including stimulated emission and the
            creation of coherent light.
      \item The process of stimulated emission through electronic transitions.
      \item The intensification and narrowing of light within a laser cavity,
            resulting in coherent light with a nearly single wavelength.
      \item Ongoing innovations and improvements in laser technology while maintaining
            fundamental principles.
\end{itemize}

\section{Exercise №19}
\subsection*{Listen to the lecture again, take notes and answer the questions.}

\begin{enumerate}
      \item What examples does the speaker give to prove his point that ‘much of our
            technology today depends on lasers’?
      \item What technology does he say highlights all other applications of lasers?
      \item What are the advantages of a laser scalpel?
      \item What are the three characteristics of laser light that the author calls‘a
            tour de force of engineering’?
      \item How are these three characteristics made?
\end{enumerate}

\subsection*{Solution}
\begin{enumerate}
      \item Examples of laser-dependent technology include range finding devices,
            optical communications, and bar code scanners.
      \item Laser technology's application in eye surgery, such as retinal reattachment,
            showcases its unique characteristics and precision.
      \item Advantages of a laser scalpel include the use of green laser light, which
            passes through the eye's lens and vitreous humor without causing damage,
            allowing for precise treatment of the retina without harming surrounding
            tissues.
      \item The three characteristics of laser light termed 'a tour de force of
            engineering' are coherent light, a narrow beam, and nearly a single
            wavelength.
      \item These characteristics are achieved through stimulated emission within a
            resonant cavity. Electrons returning to the ground state release light,
            initiating an avalanche of identical photons. Inside the cavity, reflection
            and alignment of light rays intensify and narrow the wavelength, resulting
            in coherent light.
\end{enumerate}

\chapter{Домашнее задание №3 29.02.24}

\section{Exercise №32}
\subsection*{Summarise the text in English paying attention to the linking words and
      phrases}

\subsection*{Solution}
The text discusses the invention, properties, and use of lasers. Firstly, it outlines
the history of laser invention, then transitions to its properties. Thirdly, it
discusses the types of existing lasers and concludes by examining practical laser
applications in various fields. While there is no definitive answer to who invented
the laser, several scientists contributed to its creation. Despite initial expectations
of mainly military use, lasers have found widespread application in areas such as
warfare and medicine due to their ability to produce an extremely narrow beam of
light. The text also highlights the role of lasers in computer science. Overall,
lasers have a broad range of applications, from surgical procedures to vehicle speed
control devices. In conclusion, the fact that "death rays" did not become a reality
is ultimately beneficial.

The concept of "photonics" emerged in the late 20th century and has become part of
everyday life, encompassing technologies like fiber optic communication lines,
flat-screen TVs and computer monitors, smartphones, and more. Laser communication
offers high quality, greater bandwidth, and strict confidentiality. Both lasers and
fiber optics have become vital components of many industries, and their combined
potential is rapidly expanding. Semiconductor lasers are used in fiber-optic
telecommunication systems. Research in this field has led to advancements in areas
such as quantum electronics, fiber optics, quantum optics, laser physics, laser
chemistry, and more. The term "photonics" encompasses all these scientific and
technical areas.

\section{Exercise №35}
\subsection*{Look at more examples of Participles from reading and complete the table.
      Try to define the meaning and function of the Participles in these examples.}
\begin{enumerate}
      \item The word "laser" stands for "light amplification by \allocation{stimulated}
            emission of radiation”.
      \item The idea of using lasers as death rays \allocation{employed} by creators of
            such blockbusters as X-Men and Star Wars still remains science fiction.
      \item \allocation{Having been demonstrated} by Theodore Maiman in 1960, the first
            ruby laser was considered the first successful light laser.
      \item \allocation{Having introduced} the concept of stimulated emission, Einstein
            laid the foundation for the laser.
      \item \allocation{Laser-sighting} devices are \allocation{fitted} to military and
            police rifles to help soldiers hit their targets.
      \item \allocation{Being installed} at one end of the laser tube, a mirror keeps
            the photons bouncing back and forth inside the crystal.
      \item The \allocation{escaping} photons form a very \allocation{concentrated}
            beam of powerful laser light.
      \item Light from conventional sources, such as a light bulb or the sun, diverges,
            \allocation{spreading} in all directions.
\end{enumerate}

\begin{small}
      \begin{tabular}{|r|c|c|}
            \hline
                                                             & Active      & Passive          \\
            \hline
            Present Participle (V+ing)                       & doing       & doing            \\
            Perfect Participle\textsuperscript{*}(having+V3) & having done & having been done \\
            Past Participle (V3)                             &             & done             \\
            \hline
      \end{tabular}
\end{small}

* There is also Perfect Continuous Participle form: having +been+ doing which focuses
on the duration of the action as compared to Perfect Participle.

\subsection*{Solution}
\allocation{Stimulated}: Past Participle (Active) - Having introduced the concept of
stimulated emission, Einstein laid the foundation for the laser. In this context,
"stimulated" describes the type of emission that Einstein introduced. \\
\allocation{Employed}: Past Participle (Passive) - The idea of using lasers as death
rays employed by creators of such blockbusters as X-Men and Star Wars still remains
science fiction. Here, "employed" indicates that the idea was utilized by creators. \\
\allocation{Having been demonstrated}: Perfect Participle (Passive) - Having been
demonstrated by Theodore Maiman in 1960, the first ruby laser was considered the first
successful light laser. This indicates that the first ruby laser was demonstrated by
Theodore Maiman before being considered successful. \\
\allocation{Having introduced}: Perfect Participle (Active) - Having introduced the
concept of stimulated emission, Einstein laid the foundation for the laser. This shows
that Einstein introduced the concept before laying the foundation. \\
\allocation{Laser-sighting fitted}: Past Participle (Passive) - Laser-sighting devices
are fitted to military and police rifles to help soldiers hit their targets. Here,
"fitted" is used passively to describe the state of the devices after being installed. \\
\allocation{Being installed}: Present Participle (Passive) - Being installed at one end
of the laser tube, a mirror keeps the photons bouncing back and forth inside the crystal.
This indicates the ongoing action of installing the mirror in a passive voice. \\
\allocation{Escaping concentrated}: Present Participle (Active) - The escaping photons
form a very concentrated beam of powerful laser light. Here, "escaping" and
"concentrated" are used actively to describe the properties of the photons forming the
laser beam.

\section{Exercise №36}
\subsection*{Compare the following pairs of phrases with Participle I and Participle II.
      Translate them into Russian.}
\begin{enumerate}
      \item developing industry - developed industry
      \item changing distances - changed distances
      \item a controlling device - controlled device
      \item an increasing speed - an increased speed
      \item a transmitting signal - a transmitted signal
      \item a reducing noise - reduced noise
      \item a moving object - a moved object
      \item heating parts - heated parts
\end{enumerate}

\subsection*{Solution}
\begin{enumerate}
      \item развивающаяся промышленность - развитая промышленность
      \item изменяющиеся расстояния - изменённые расстояния
      \item управляющее устройство - управляемое устройство
      \item увеличивающаяся скорость - увеличенная скорость
      \item передающий сигнал - переданный сигнал
      \item уменьшающий шум - уменьшенный шум
      \item движущийся объект - перемещённый объект
      \item нагревающиеся детали -  нагретые детали
\end{enumerate}

\section{Exercise №37}
\subsection*{Choose the correct form.}
\begin{enumerate}
      \item \allocation{A:} Have you read that new book yet?
            \allocation{B:} Only some of it. It’s very…\\
            a. bored \quad b. boring

      \item \allocation{A:} Did you enjoy your holiday?
            \allocation{B:} Oh, yes. It was very…\\
            a. relaxed \quad b. relaxing

      \item \allocation{A:} I'm going to a lecture tonight. Do you want to come?
            \allocation{B:} No, thanks. I'm not … in the subject.\\
            a. interested \quad b. interesting

      \item \allocation{A:} Did you hurt yourself when you fell?
            \allocation{B:} No, but it was very …\\
            a. embarrassed \quad b. embarrassing

      \item \allocation{A:} Was mother upset when you broke her vase?
            \allocation{B:} Not really, but she was very….\\
            a. annoyed \quad b. annoying

      \item \allocation{A:} How do you feel today?
            \allocation{B:} I still feel very ….\\
            a. tired \quad b. tiring

      \item \allocation{A:} You look ill. What’s the matter?
            \allocation{B:} I’ve had a very …. day.\\
            a. tired \quad b. tiring

      \item Sit down - I've got some very …. news for you.\\
            a. excited \quad b. exciting

      \item He's got a very …. habit of always interrupting people.\\
            a. annoyed \quad b. annoying

      \item I'm very …. by your behaviour.\\
            a. disappointed \quad b. disappointing
\end{enumerate}

\subsection*{Solution}
\begin{enumerate}
      \item \allocation{A:} Have you read that new book yet?
            \allocation{B:} Only some of it. It’s very…\\
            \textbf{Correct answer:} \textbf{b.} boring

      \item \allocation{A:} Did you enjoy your holiday?
            \allocation{B:} Oh, yes. It was very…\\
            \textbf{Correct answer:} \textbf{a.} relaxed

      \item \allocation{A:} I'm going to a lecture tonight. Do you want to come?
            \allocation{B:} No, thanks. I'm not … in the subject.\\
            \textbf{Correct answer:} \textbf{a.} interested

      \item \allocation{A:} Did you hurt yourself when you fell?
            \allocation{B:} No, but it was very …\\
            \textbf{Correct answer:} \textbf{b.} embarrassing

      \item \allocation{A:} Was mother upset when you broke her vase?
            \allocation{B:} Not really, but she was very….\\
            \textbf{Correct answer:} \textbf{a.} annoyed

      \item \allocation{A:} How do you feel today?
            \allocation{B:} I still feel very ….\\
            \textbf{Correct answer:} \textbf{a.} tired

      \item \allocation{A:} You look ill. What’s the matter?
            \allocation{B:} I’ve had a very …. day.\\
            \textbf{Correct answer:} \textbf{a.} tired

      \item Sit down - I've got some very …. news for you.\\
            \textbf{Correct answer:} \textbf{b.} exciting

      \item He's got a very …. habit of always interrupting people.\\
            \textbf{Correct answer:} \textbf{b.} annoying

      \item I'm very …. by your behaviour.\\
            \textbf{Correct answer:} \textbf{a.} disappointed
\end{enumerate}

\chapter{Семинар №4 29.02.24}

\section{Exercise №38}
\subsection*{Fill in the Perfect Participle, Active or Passive, of the verbs in
      brackets. Explain the meaning of Perfect Participle phrases or translate the
      sentences into Russian.}
\begin{enumerate}
      \item \allocation{(Work)} all day, I was feeling very tired in the evening.
      \item \allocation{(Live)} in an English-speaking country for a few years, she
            spoke English like a native speaker.
      \item \allocation{(Rescue)}, an injured pilot was taken to hospital.
      \item \allocation{(Write)} the test, the students handed in their papers.
      \item \allocation{(Sign)} by the boss, the documents were sent to the customers.
      \item \allocation{(Interrupt)} a few times, he was rather annoyed.
      \item \allocation{(Stop)} the car, the police officer wanted to see the documents.
      \item \allocation{(Arrive)} at the station, we called a taxi.
      \item \allocation{(Check in)} for the flight, they were prepared for the passport
            control.
      \item \allocation{(Buy)} the car, he stopped using public transport.
\end{enumerate}

\subsection*{Solution}
\begin{enumerate}
      \item Having worked all day, I was feeling very tired in the evening.
      \item Having lived in an English-speaking country for a few years, she spoke
            English like a native speaker.
      \item Having been rescued, an injured pilot was taken to the hospital.
      \item Having written the test, the students handed in their papers.
      \item Having been signed by the boss, the documents were sent to the customers.
      \item Having been interrupted a few times, he was rather annoyed.
      \item Having stopped the car, the police officer wanted to see the documents.
      \item Having arrived at the station, we called a taxi.
      \item Having checked in for the flight, they were prepared for passport control.
      \item Having bought the car, he stopped using public transport.
\end{enumerate}

\section{Exercise №39}
\subsection*{Choose the correct form of the Participle. Translate the sentences into
      Russian.}
\begin{enumerate}
      \item They were trying to fix a USB cable \underline{\hspace{2cm}} the
            instructions from a YouTube video.
      \item Serious faults \underline{\hspace{2cm}} in the project had to be corrected
            quickly.
      \item The method \underline{\hspace{2cm}} by the engineers at the moment has
            numerous advantages.
      \item \underline{\hspace{2cm}} no job and no money, he couldn’t pay the rent.
      \item \underline{\hspace{2cm}} a new technique, scientists increased the accuracy
            of the results.
      \item People should be careful, while \underline{\hspace{2cm}} the street.
      \item \underline{\hspace{2cm}} the door, he left the house.
      \item Utilising the principle of feedback, robots can change their operation in
            response to a changing environment.
      \item \underline{\hspace{2cm}} her work, she went home.
      \item \underline{\hspace{2cm}} an expert in the field of computers, he had no
            problem finding a well-paid job.
\end{enumerate}

\subsection*{Solution}
\begin{enumerate}
      \item They were trying to fix a USB cable \allocation{following} the instructions
            from a YouTube video.
      \item Serious faults \allocation{found} in the project had to be corrected quickly.
      \item The method \allocation{being discussed} by the engineers at the moment has
            numerous advantages.
      \item \allocation{Having had} no job and no money, he couldn’t pay the rent.
      \item \allocation{Having applied} a new technique, scientists increased the
            accuracy of the results.
      \item People should be careful, \allocation{while crossing} the street.
      \item \allocation{Having locked} the door, he left the house.
      \item Utilising the principle of feedback, robots can change their operation in
            response to a changing environment.
      \item \allocation{Having completed} her work, she went home.
      \item \allocation{Being} an expert in the field of computers, he had no problem
            finding a well-paid job.
\end{enumerate}

\section{Exercise №41}
\subsection*{Rewrite the following sentences with Participle Constructions according
      to the examples given below and identify the meaning of Participle Constructions.}
\begin{enumerate}
      \item Walking in the woods, I suddenly realised that I had lost my way.
      \item Having spent a lot of time doing my homework, I went to bed very late last
            night.
      \item Given proper care, your car will operate smoothly for years.
      \item Working in a bank, he was familiar with the best ways to invest money.
      \item My sister is the one talking to the professor.
      \item Having collected the data, he began analysing the results.
      \item Having arrived at the site, the scientists discovered many fragments of the
            meteorite.
      \item Being one of the most beautiful Russian monuments, St Basil’s Basilica is a
            World Heritage site.
      \item Trying to sell more goods for cash, the company is losing money.
      \item Karel Capek described a mechanical device that looked like a human but
            lacking human sensibility could perform only automatic, mechanical
            operations.
\end{enumerate}

\subsection*{Solution}
\begin{enumerate}
      \item While walking in the woods, I suddenly realized that I had lost my way.
      \item Because I had spent a lot of time doing my homework, I went to bed very
            late last night.
      \item If given proper care, your car will operate smoothly for years.
      \item Being someone who worked in a bank, he was familiar with the best ways to
            invest money.
      \item The one talking to the professor is my sister.
      \item After having collected the data, he began analyzing the results.
      \item Upon having arrived at the site, the scientists discovered many fragments
            of the meteorite.
      \item St Basil’s Basilica, being one of the most beautiful Russian monuments, is
            a World Heritage site.
      \item In trying to sell more goods for cash, the company is losing money.
      \item Karel Capek described a mechanical device resembling a human but lacking
            human sensibility, capable only of performing automatic, mechanical
            operations.
\end{enumerate}

\section{Exercise №42}
\subsection*{Use a Participle Construction to add the information in italics to the main
      sentence.}
\allocation{Example:} Sam left school early because he felt sick. → Feeling sick, Sam
left school early. They spent all the money. So they couldn’t afford to buy a car. →
Having spent all the money, they couldn’t afford buying a car.

\begin{enumerate}
      \item As she felt tired, Anna went to bed early. \underline{\hspace{2cm}}, Anna
            went to bed early.
      \item After the boss had explained the problem, he told the employee to deal with
            it. \underline{\hspace{2cm}}, the boss told the employee to deal with it.
      \item While he was drinking his coffee, he was thinking about the problem.
            \underline{\hspace{2cm}}, he was thinking about the problem.
      \item If it is looked after carefully, the plant can live through the winter.
            \underline{\hspace{2cm}}, the plant can live through the winter.
      \item We filled up the car and continued our journey. \underline{\hspace{2cm}},
            we continued our journey.
      \item As the manager was impressed by my work, he extended my contract.
            \underline{\hspace{2cm}}, the manager extended my contract.
      \item They have written two tests today and they are too tired to do the third
            one. \underline{\hspace{2cm}}, they are too tired to do the third one.
      \item He was driving home. He had an accident. \underline{\hspace{2cm}}, he had
            an accident.
      \item He was trapped in a dilemma and couldn’t decide what to do.
            \underline{\hspace{2cm}}, he couldn’t decide what to do.
      \item After I dropped him off at the station, I drove straight to the supermarket.
            \underline{\hspace{2cm}}, I drove straight to the supermarket.
      \item The teacher was impressed by Mike’s work, so she gave him the highest mark.
            \underline{\hspace{2cm}}, the teacher gave him the highest mark.
      \item As he had been to England before, he knew where to find a good hotel.
            \underline{\hspace{2cm}}, he knew where to find a good hotel.
\end{enumerate}

\subsection*{Solution}
\begin{enumerate}
      \item Feeling tired, Anna went to bed early.
      \item Having explained the problem, the boss told the employee to deal with it.
      \item Drinking his coffee, he was thinking about the problem.
      \item If carefully looked after, the plant can live through the winter.
      \item Having filled up the car, we continued our journey.
      \item Impressed by my work, the manager extended my contract.
      \item Having written two tests today, they are too tired to do the third one.
      \item Driving home, he had an accident.
      \item Trapped in a dilemma, he couldn’t decide what to do.
      \item After dropping him off at the station, I drove straight to the supermarket.
      \item Impressed by Mike’s work, the teacher gave him the highest mark.
      \item Having been to England before, he knew where to find a good hotel.
\end{enumerate}

\section{Study note}
\subsection*{Participles Overview}
\begin{itemize}
      \item Participles are words derived from verbs that can function as adjectives
            and adverbs or as part of verb phrases to create verb tenses.
      \item The main types of Participles are:
            \begin{itemize}
                  \item Present Participle (Participle I), e.g., \textit{coming}
                  \item Perfect Participle, e.g., \textit{having completed}
                  \item Past Participle (Participle II), e.g., \textit{used}
            \end{itemize}
      \item Participles may also be identified with a particular Voice: active or
            passive.
\end{itemize}

\subsection*{Present and Past Participles as Adjectives}
\begin{itemize}
      \item Present and Past Participles can both be used as adjectives.
      \item The Present Participle describes what someone or something is (What kind?).
      \item The Past Participle describes how somebody feels (How do you feel?).
\end{itemize}

\subsection*{Perfect Participle}
\begin{itemize}
      \item The Perfect Participle (active and passive) is used to emphasize that one
            action happened before another.
      \item Perfect Participles are often used as part of Participle Constructions or
            Clauses that are equivalent to adverbial clauses within complex sentences.
\end{itemize}

\subsection*{Participle Constructions}
\begin{itemize}
      \item Participles are often used as part of Participle Constructions or Clauses,
            enabling a more economical presentation of information compared to complex
            sentences.
      \item Participle Constructions act as adjectives or adverbs within sentences and
            usually are reduced adverbial or relative clauses.
      \item They can be used after various conjunctions such as: when, while, if,
            though, etc.
\end{itemize}

\subsection*{Negative Participle Constructions}
\begin{itemize}
      \item Negative participle constructions are possible, where 'not' normally comes
            before the Participle.
\end{itemize}

\subsection*{Example Sentences}
\begin{enumerate}
      \item A humanoid drawn by Leonardo da Vinci is among the first verifiable
            automation.
      \item Being fitted with vision equipment, robots are able to ‘see’.
      \item I hurt my arm while playing tennis.
      \item Remember to take all your belongings with you when leaving the train.
      \item Not having seen the film, I could not take part in its discussion.
\end{enumerate}

\chapter{Домашнее задание №4 07.03.24}

\section{Exercise №43}
\subsection*{Combine the following sentences into one using a negative Participle Construction:}
\allocation{Example:} I didn’t want to hurt his feelings. I didn’t ask any questions. → Not wanting to hurt his
feelings, I didn’t ask any questions.

\begin{enumerate}
      \item As they haven't received all the applications yet, they are not ready to hire anyone.
      \item I didn't want to lose my passport. I gave it to my father.
      \item I didn't know how to reply. I didn't say a word.
      \item He didn't see the accident ahead of him. He didn't stop his car.
      \item They haven't followed the instructions. They have a problem with the cleaning robot.
      \item They haven't found any flaws in the project. They can start it as soon as possible.
      \item I had no phone. I couldn't call you.
      \item He didn't notice a fuel warning light. He didn't fill up his car in time.
      \item The method wasn't tested. It was not adopted.
      \item He hadn't prepared for the exam. He failed it.
\end{enumerate}

\subsection*{Solution}
\begin{enumerate}
      \item Not having received all the applications yet, they are not ready to hire anyone.
      \item Not wanting to lose my passport, I didn't give it to my father.
      \item Not knowing how to reply, I didn't say a word.
      \item Not seeing the accident ahead of him, he didn't stop his car.
      \item Not following the instructions, they have a problem with the cleaning robot.
      \item Not finding any flaws in the project, they can start it as soon as possible.
      \item Not having no phone, I couldn't call you.
      \item Not noticing a fuel warning light, he didn't fill up his car in time.
      \item Not being tested, the method was not adopted.
      \item Not having prepared for the exam, he failed it.
\end{enumerate}

\section{Exercise №44}
\subsection*{ Put the verbs in brackets into the correct Participle form. (Some examples can have more
      than one correct answer with a difference in meaning). Identify Participle Constructions and
      explain their meaning.}
\begin{enumerate}
      \item \underline{\hspace{1.5cm}} a lecture, the professor was not using any notes.
      \item The problems \underline{\hspace{1.5cm}} at the conference are very important.
      \item The shop \underline{\hspace{1.5cm}} next to our university will open soon.
      \item \underline{\hspace{1.5cm}} a car, she finds it easy to get around.
      \item \underline{\hspace{1.5cm}} the questions, he gave a lot of examples.
      \item \underline{\hspace{1.5cm}} his research, he was ready to write a report.
      \item The equipment \underline{\hspace{1.5cm}} by the company is of the highest standard.
      \item \underline{\hspace{1.5cm}} left or right, indicate it by using a turn signal.
      \item \underline{\hspace{1.5cm}} its maximum intensity, the volcano began to calm down.
      \item \underline{\hspace{1.5cm}} goodbye, he left the office.
      \item \underline{\hspace{1.5cm}} by millions of readers, this book has rightfully become a bestseller.
\end{enumerate}

\subsection*{Solution}
\begin{enumerate}
      \item \textbf{Giving} a lecture, the professor was not using any notes.
      \item The problems \textbf{discussed} at the conference are very important.
      \item The shop \textbf{being built} next to our university will open soon.
      \item \textbf{Having} a car, she finds it easy to get around.
      \item \textbf{Having} answered the questions, he gave a lot of examples.
      \item \textbf{Having} finished his research, he was ready to write a report.
      \item The equipment \textbf{installed} by the company is of the highest standard.
      \item \textbf{Turning} left or right, indicate it by using a turn signal.
      \item \textbf{Having} reached its maximum intensity, the volcano began to calm down.
      \item \textbf{Having} said goodbye, he left the office.
      \item \textbf{Read} by millions of readers, this book has rightfully become a bestseller.
\end{enumerate}

\section{Exercise №45}
\subsection*{Match the sentences with the terms, related to technology. Match the pictures with the
      sentences.}
\begin{enumerate}
      \item James Bond has demonstrated laser cutting in 1964 in Ian Flemings "Goldfinger".
      \item These scanners use a laser beam that is scanned back and forth so rapidly that it appears as a line to the human eye.
      \item These simple, pocket-sized lasers are used to highlight important areas during presentations.
      \item The laser acts as a precise disc-reading mechanism.
      \item The first use of a laser in medicine occurred in the early 1960s, when physicians used a laser on a human for the first time by destroying a retinal eye tumor with a ruby laser.
      \item Both lasers and fiber optics have independently become vital components of many industries.
      \item Laser shows produce visual displays by using beam effects; either by switching a stationary beam on and off or by creating dynamic beam effects.
\end{enumerate}

\subsection*{Options}
\begin{enumerate}
      \item[a.] Laser Pointers
      \item[b.] DVD and CD Players
      \item[c.] Laser Light Shows and Holography
      \item[d.] Surgery
      \item[e.] Fiber Optical Communication
      \item[f.] Barcode Scanner
      \item[g.] Laser Cutting Machines
\end{enumerate}

\subsection*{Solution}
\begin{enumerate}
      \item g
      \item f
      \item a
      \item b
      \item d
      \item e
      \item c
\end{enumerate}

\section{Exercise №46}
\allocation{Устно}

\subsection*{Read and translate the sentences paying attention to the Participle Clauses.}
\begin{enumerate}
      \item When \allocation{completed} in 2010, the Burj Khalifa became the tallest tower in the world and one of the top attractions in Dubai.
      \item Though \allocation{being} a school teacher of mathematics, Tsiolkovsky developed space travel principles that remain in use today.
      \item If \allocation{compared} to today's TV pictures, the first black-and-white images had rather poor quality.
      \item While \allocation{teaching} at school for the deaf, Bell became interested in sound and its transmission.
      \item Though \allocation{discovered}, Newton's mistake had no influence on his theory.
      \item While \allocation{conducting} experiments with communication devices and speech systems, Bell invented the telephone.
      \item If \allocation{cooled} below zero degrees Celsius, water freezes.
      \item While \allocation{working} on the quantum oscillator, Nikolay Basov and Alexander Prokhorov created the first microwave generator, which was the laser’s predecessor.
\end{enumerate}

\section{Exercise №47}
\subsection*{Rewrite the following sentences with Participle Constructions adding an appropriate
      conjunction.}
\begin{enumerate}
      \item Being one of the key issues today, information protection is the centre of attention for today’s computer engineers.
      \item Analysing the information on what is currently being tested, we can imagine what new robots will be like.
      \item Being designed by researchers at the Stanford Research Institute in the late 1960s, an experimental model became one of the first real robots.
      \item Having been added to vehicles, airbags saved numerous lives.
      \item Being known to people from science fiction, robots didn’t materialise until the invention of the computer in the 1940s.
      \item After using a television camera as a visual sensor, the engineers constructed a robot capable of arranging blocks into stacks.
      \item If equipped with microprocessors, computerised robots can handle the data being fed to them by various sensors.
      \item Being fitted with new safety features, robotic vehicles will be much safer than before.
      \item Being one of the main sources of pollution, petrol cars are still widely used today.
      \item Increasing the commercial use of robots, we continue to expand their applications.
\end{enumerate}

\subsection*{Solution}
\begin{enumerate}
      \item While being one of the key issues today, information protection is the center of attention for today’s computer engineers.
      \item While analyzing the information on what is currently being tested, we can imagine what new robots will be like.
      \item Designed by researchers at the Stanford Research Institute in the late 1960s, an experimental model became one of the first real robots.
      \item With airbags having been added to vehicles, numerous lives were saved.
      \item Despite being known to people from science fiction, robots didn’t materialize until the invention of the computer in the 1940s.
      \item Having used a television camera as a visual sensor, the engineers constructed a robot capable of arranging blocks into stacks.
      \item Equipped with microprocessors, computerized robots can handle the data being fed to them by various sensors.
      \item With robotic vehicles being fitted with new safety features, they will be much safer than before.
      \item Despite being one of the main sources of pollution, petrol cars are still widely used today.
      \item By increasing the commercial use of robots, we continue to expand their applications.
\end{enumerate}

\chapter{Семинар №5 07.03.24}

\section{Exercise №30}
\subsection*{Transforming Clauses into Participial Constructions}
\begin{enumerate}
      \item While Boris was driving home, he saw an accident.
      \item After we had talked with Peter, we felt much better.
      \item When John arrived at the station, he saw the train leave.
      \item After he had left the house, he walked to the nearest metro station.
      \item When I looked out of the window, I saw Mary coming.
      \item As we finished our part of the work, we were free to go home.
      \item As Ann had had no time to write us a letter, she sent a telegram.
\end{enumerate}

\subsection*{Solution}
\begin{enumerate}
      \item Driving home, Boris saw an accident.
      \item Having talked with Peter, we felt much better.
      \item Arriving at the station, John saw the train leave.
      \item Having left the house, he walked to the nearest metro station.
      \item Looking out of the window, I saw Mary coming.
      \item Having finished our part of the work, we were free to go home.
      \item Not having had time to write us a letter, Ann sent a telegram.
\end{enumerate}

\section{Exercise №5}
\subsection*{Choose the correct option.}
\begin{enumerate}
      \item I think I know the actor \textbf{played/playing} the main role in this new TV series.
      \item The committee \textbf{believe} the answer \textbf{given/giving} by the politician wasn't the whole truth.
      \item All the games \textbf{played/playing} on the second day of the competition ended in a draw. That had never happened before.
      \item \textbf{Checked in/having checked in}, we unpacked and went to get something to eat.
      \item \textbf{Having watched/after watched} his son win the competition, he was filled with pride.
      \item She suddenly realized that the person \textbf{spoken/speaking} on the phone wasn't her husband but a complete stranger.
      \item He was sitting on the sofa \textbf{doing/having done} a crossword.
      \item \textbf{Having paid/paid} for the meal, we left the restaurant.
      \item \textbf{Being/been} exhausted, he fell asleep on the bus.
      \item While \textbf{watching/being watched} a play, he fell asleep.
\end{enumerate}

\subsection*{Solution}
\begin{enumerate}
      \item I think I know the actor \textbf{playing} the main role in this new TV series.
      \item The committee \textbf{believes} the answer given by the politician wasn't the whole truth.
      \item All the games \textbf{played} on the second day of the competition ended in a draw. That had never happened before.
      \item \textbf{Having checked in}, we unpacked and went to get something to eat.
      \item \textbf{After watching} his son win the competition, he was filled with pride.
      \item She suddenly realized that the person \textbf{speaking} on the phone wasn't her husband but a complete stranger.
      \item He was sitting on the sofa \textbf{doing} a crossword.
      \item \textbf{Having paid} for the meal, we left the restaurant.
      \item \textbf{Being exhausted}, he fell asleep on the bus.
      \item While \textbf{watching} a play, he fell asleep.
\end{enumerate}

\chapter{Исправление домашних заданий №1 21.03.24}

\section{Exercise №41}
\subsection*{Rewrite the following sentences with Participle Constructions according
      to the examples given below and identify the meaning of Participle Constructions.}
\begin{enumerate}
      \item Walking in the woods, I suddenly realised that I had lost my way.
      \item Having spent a lot of time doing my homework, I went to bed very late last
            night.
      \item Given proper care, your car will operate smoothly for years.
      \item Working in a bank, he was familiar with the best ways to invest money.
      \item My sister is the one talking to the professor.
      \item Having collected the data, he began analysing the results.
      \item Having arrived at the site, the scientists discovered many fragments of the
            meteorite.
      \item Being one of the most beautiful Russian monuments, St Basil’s Basilica is a
            World Heritage site.
      \item Trying to sell more goods for cash, the company is losing money.
      \item Karel Capek described a mechanical device that looked like a human but
            lacking human sensibility could perform only automatic, mechanical
            operations.
\end{enumerate}

\subsection*{Old solution}
\begin{enumerate}
      \item While walking in the woods, I suddenly realized that I had lost my way.
      \item Because I had spent a lot of time doing my homework, I went to bed very
            late last night.
      \item If given proper care, your car will operate smoothly for years.
      \item Being someone who worked in a bank, he was familiar with the best ways to
            invest money.
      \item The one talking to the professor is my sister.
      \item After having collected the data, he began analyzing the results.
      \item Upon having arrived at the site, the scientists discovered many fragments
            of the meteorite.
      \item St Basil’s Basilica, being one of the most beautiful Russian monuments, is
            a World Heritage site.
      \item In trying to sell more goods for cash, the company is losing money.
      \item Karel Capek described a mechanical device resembling a human but lacking
            human sensibility, capable only of performing automatic, mechanical
            operations.
\end{enumerate}

\subsection*{New solution}
\begin{enumerate}
      \item While walking in the woods, I suddenly realized that I had lost my way.
      \item Because I had spent a lot of time doing my homework, I went to bed very late last night.
      \item If given proper care, your car will operate smoothly for years.
      \item Being someone who worked in a bank, he was familiar with the best ways to invest money.
      \item The one talking to the professor is my sister.
      \item After having collected the data, he began analyzing the results.
      \item Upon having arrived at the site, the scientists discovered many fragments of the meteorite.
      \item St Basil’s Basilica, being one of the most beautiful Russian monuments, is a World Heritage site.
      \item In trying to sell more goods for cash, the company is losing money.
      \item Karel Capek described a mechanical device resembling a human but lacking human sensibility, capable only of performing automatic, mechanical operations.
\end{enumerate}

\section{Exercise №42}
\subsection*{Use a Participle Construction to add the information in italics to the main
      sentence.}
\allocation{Example:} Sam left school early because he felt sick. → Feeling sick, Sam
left school early. They spent all the money. So they couldn’t afford to buy a car. →
Having spent all the money, they couldn’t afford buying a car.

\begin{enumerate}
      \item As she felt tired, Anna went to bed early. \underline{\hspace{2cm}}, Anna
            went to bed early.
      \item After the boss had explained the problem, he told the employee to deal with
            it. \underline{\hspace{2cm}}, the boss told the employee to deal with it.
      \item While he was drinking his coffee, he was thinking about the problem.
            \underline{\hspace{2cm}}, he was thinking about the problem.
      \item If it is looked after carefully, the plant can live through the winter.
            \underline{\hspace{2cm}}, the plant can live through the winter.
      \item We filled up the car and continued our journey. \underline{\hspace{2cm}},
            we continued our journey.
      \item As the manager was impressed by my work, he extended my contract.
            \underline{\hspace{2cm}}, the manager extended my contract.
      \item They have written two tests today and they are too tired to do the third
            one. \underline{\hspace{2cm}}, they are too tired to do the third one.
      \item He was driving home. He had an accident. \underline{\hspace{2cm}}, he had
            an accident.
      \item He was trapped in a dilemma and couldn’t decide what to do.
            \underline{\hspace{2cm}}, he couldn’t decide what to do.
      \item After I dropped him off at the station, I drove straight to the supermarket.
            \underline{\hspace{2cm}}, I drove straight to the supermarket.
      \item The teacher was impressed by Mike’s work, so she gave him the highest mark.
            \underline{\hspace{2cm}}, the teacher gave him the highest mark.
      \item As he had been to England before, he knew where to find a good hotel.
            \underline{\hspace{2cm}}, he knew where to find a good hotel.
\end{enumerate}

\subsection*{Old solution}
\begin{enumerate}
      \item Feeling tired, Anna went to bed early.
      \item Having explained the problem, the boss told the employee to deal with it.
      \item Drinking his coffee, he was thinking about the problem.
      \item If carefully looked after, the plant can live through the winter.
      \item Having filled up the car, we continued our journey.
      \item Impressed by my work, the manager extended my contract.
      \item Having written two tests today, they are too tired to do the third one.
      \item Driving home, he had an accident.
      \item Trapped in a dilemma, he couldn’t decide what to do.
      \item After dropping him off at the station, I drove straight to the supermarket.
      \item Impressed by Mike’s work, the teacher gave him the highest mark.
      \item Having been to England before, he knew where to find a good hotel.
\end{enumerate}

\subsection*{New solution}
\begin{enumerate}
      \item Feeling tired, Anna went to bed early.
      \item Having explained the problem, the boss told the employee to deal with it.
      \item Drinking his coffee, he was thinking about the problem.
      \item If carefully looked after, the plant can live through the winter.
      \item Having filled up the car, we continued our journey.
      \item Impressed by my work, the manager extended my contract.
      \item Having written two tests today, they are too tired to do the third one.
      \item Driving home, he had an accident.
      \item Trapped in a dilemma, he couldn’t decide what to do.
      \item After dropping him off at the station, I drove straight to the supermarket.
      \item Impressed by Mike’s work, the teacher gave him the highest mark.
      \item Having been to England before, he knew where to find a good hotel.
\end{enumerate}

\section{Exercise №43}
\subsection*{Combine the following sentences into one using a negative Participle Construction:}

\allocation{Example:} I didn’t want to hurt his feelings. I didn’t ask any questions. → Not wanting to hurt his
feelings, I didn’t ask any questions.

\begin{enumerate}
      \item As they haven't received all the applications yet, they are not ready to hire anyone.
      \item I didn't want to lose my passport. I gave it to my father.
      \item I didn't know how to reply. I didn't say a word.
      \item He didn't see the accident ahead of him. He didn't stop his car.
      \item They haven't followed the instructions. They have a problem with the cleaning robot.
      \item They haven't found any flaws in the project. They can start it as soon as possible.
      \item I had no phone. I couldn't call you.
      \item He didn't notice a fuel warning light. He didn't fill up his car in time.
      \item The method wasn't tested. It was not adopted.
      \item He hadn't prepared for the exam. He failed it.
\end{enumerate}

\subsection*{Old solution}
\begin{enumerate}
      \item Not having received all the applications yet, they are not ready to hire anyone.
      \item Not wanting to lose my passport, I didn't give it to my father.
      \item Not knowing how to reply, I didn't say a word.
      \item Not seeing the accident ahead of him, he didn't stop his car.
      \item Not following the instructions, they have a problem with the cleaning robot.
      \item Not finding any flaws in the project, they can start it as soon as possible.
      \item Not having no phone, I couldn't call you.
      \item Not noticing a fuel warning light, he didn't fill up his car in time.
      \item Not being tested, the method was not adopted.
      \item Not having prepared for the exam, he failed it.
\end{enumerate}

\subsection*{New solution}
\begin{enumerate}
      \item Not having received all the applications yet, they are not ready to hire anyone.
      \item Not wanting to lose my passport, I didn't give it to my father.
      \item Not knowing how to reply, I didn't say a word.
      \item Not seeing the accident ahead of him, he didn't stop his car.
      \item Not following the instructions, they have a problem with the cleaning robot.
      \item Not finding any flaws in the project, they can start it as soon as possible.
      \item Not having a phone, I couldn't call you.
      \item Not noticing a fuel warning light, he didn't fill up his car in time.
      \item Not being tested, the method was not adopted.
      \item Not having prepared for the exam, he failed it.
\end{enumerate}

\section{Exercise №44}
\subsection*{ Put the verbs in brackets into the correct Participle form. (Some examples can have more
      than one correct answer with a difference in meaning). Identify Participle Constructions and
      explain their meaning.}
\begin{enumerate}
      \item \underline{\hspace{1.5cm}} a lecture, the professor was not using any notes.
      \item The problems \underline{\hspace{1.5cm}} at the conference are very important.
      \item The shop \underline{\hspace{1.5cm}} next to our university will open soon.
      \item \underline{\hspace{1.5cm}} a car, she finds it easy to get around.
      \item \underline{\hspace{1.5cm}} the questions, he gave a lot of examples.
      \item \underline{\hspace{1.5cm}} his research, he was ready to write a report.
      \item The equipment \underline{\hspace{1.5cm}} by the company is of the highest standard.
      \item \underline{\hspace{1.5cm}} left or right, indicate it by using a turn signal.
      \item \underline{\hspace{1.5cm}} its maximum intensity, the volcano began to calm down.
      \item \underline{\hspace{1.5cm}} goodbye, he left the office.
      \item \underline{\hspace{1.5cm}} by millions of readers, this book has rightfully become a bestseller.
\end{enumerate}

\subsection*{Old solution}
\begin{enumerate}
      \item \textbf{Giving} a lecture, the professor was not using any notes.
      \item The problems \textbf{discussed} at the conference are very important.
      \item The shop \textbf{being built} next to our university will open soon.
      \item \textbf{Having} a car, she finds it easy to get around.
      \item \textbf{Having} answered the questions, he gave a lot of examples.
      \item \textbf{Having} finished his research, he was ready to write a report.
      \item The equipment \textbf{installed} by the company is of the highest standard.
      \item \textbf{Turning} left or right, indicate it by using a turn signal.
      \item \textbf{Having} reached its maximum intensity, the volcano began to calm down.
      \item \textbf{Having} said goodbye, he left the office.
      \item \textbf{Read} by millions of readers, this book has rightfully become a bestseller.
\end{enumerate}

\subsection*{New solution}
\begin{enumerate}
      \item \textbf{Giving} a lecture, the professor was not using any notes.
      \item The problems \textbf{discussed} at the conference are very important.
      \item The shop \textbf{being built} next to our university will open soon.
      \item \textbf{Having} a car, she finds it easy to get around.
      \item \textbf{Having} answered the questions, he gave a lot of examples.
      \item \textbf{Having} finished his research, he was ready to write a report.
      \item The equipment \textbf{installed} by the company is of the highest standard.
      \item \textbf{Turning} left or right, indicate it by using a turn signal.
      \item \textbf{Having} reached its maximum intensity, the volcano began to calm down.
      \item \textbf{Having} said goodbye, he left the office.
      \item \textbf{Read} by millions of readers, this book has rightfully become a bestseller.
\end{enumerate}

\part{Модуль №1}

\chapter{Семинар №6 21.03.24}

\section{Text 11A}

\subsection*{Materials and Materials Science}
\allocation{(1)}We define materials as substances having properties which make them useful in machines,
structures, devices, and other products. It would be no exaggeration to say that human civilization
has been shaped by breakthroughs in materials science. In ancient times the choice of material gave
the name to the era, for example, the Stone Age, the Bronze Age, the Iron Age, etc. So, materials
science is one of the oldest forms of technology and applied science, deriving from manufacture of
ceramics. It is concerned with a wide range of substances, from relatively easily acquired wood or
stone, to modern man made materials such as plastics and glass, or even more advanced smart
materials involving nano- and biotechnology in their manufacture. Part of materials science deals
with classifying materials, which are generally split into four main groups: metals, polymers,
ceramics, and composites.

\allocation{(2)} In the modern world materials are ubiquitous, and so pervasive1 that we often take them for
granted2. Virtually every segment of our daily life is influenced by materials. They have contributed
to the advancement of a number of technologies, including medicine and health, information and
communication, national security and space, transportation, textiles, personal hygiene, agriculture,
food science and the environmental protection. Materials have a generality comparable3 to that of
energy and information, and the three together comprise nearly all technology.

\allocation{(3)} Materials science also covers discovering and designing new materials and analysing their
properties and structure. At present, it is a key discipline in the competitive global economy, a
dynamic and exciting field with many remarkable new materials and discoveries. The power of
materials advances can be illustrated by the following example: to build today’s smartphone in the
1980s would have cost about \$110 million, required nearly 200 kilowatts of energy (compared to
2kW per year today), and the device would have been 14 meters tall. So, materials science has
virtually brought smartphones to the pockets of over 3.5 billion people. Important elements of
modern materials science are products of the space race: the understanding and engineering of
metallic alloys, silica and carbon materials, used in the construction of space vehicles.

\allocation{(4)} Today we are in the midst4 of a materials revolution. Materials are evolving faster today than at
any time in history enabling engineers to improve the performance of existing products and to
develop innovative technologies that will enhance every aspect of our lives. Scientists are using
powerful simulation techniques, as well as sophisticated machine learning algorithms, to propel5
innovations forward at a blazing6 speed and point them toward possibilities they had never
considered. These tools have helped create the metamaterials* used in carbon fiber composites for
lighter-weight vehicles, advanced alloys for more durable7 jet engines, and biomaterials to replace
human joints8. We also see breakthroughs in energy storage and quantum computing. In robotics,
new materials are helping us create the artificial muscles9 needed for humanoid, soft robots.
Advances are currently happening at the macroscale down to microscale10 for metamaterials, and at
the nanoscale with graphene11, carbon nanotubes12, composites, thin metallic- and semiconductor-
based films13.

\allocation{(5)} To sum it up, we need to discover and develop new kinds of materials with the desired properties
and at the relevant cost14 to meet the challenges of the current world. Materials science may be the
most important technology of the next decade. Breakthroughs in materials science are likely to affect
the future of technology significantly. A vast acceleration in our ability to create new, advanced
materials will power industries from energy to manufacturing. Thus, the future is bright for materials
science and engineering.

\section{Exercise №2}

\subsection*{Match the materials (1-7) to the descriptions (a-g).}
\begin{enumerate}
      \item compounds
      \item exotic
      \item ferrous
      \item ceramics
      \item alloy
      \item non-metallic
      \item polymers
\end{enumerate}

\begin{enumerate}
      \item[a.] materials that are not metal
      \item[b.] iron and steel
      \item[c.] combinations of materials
      \item[d.] mixture of metals
      \item[e.] plastic materials
      \item[f.] minerals transformed by heat
      \item[g.] rare or complex
\end{enumerate}

\subsection*{Solution}
\begin{enumerate}
      \item c
      \item g
      \item b
      \item f
      \item d
      \item a
      \item e
\end{enumerate}

\section{Exercise №3}

\subsection*{Find the right answers in column B to finish the sentences in column A. Use a dictionary if
      necessary.}

\allocation{A}
\begin{enumerate}
      \item The first materials people used were
      \item In the past the bricks were made of
      \item The first metals people got were
      \item The materials people learned to produce were
      \item The materials that we grow
      \item The materials that we mine are
      \item The materials made from chemicals that come from oil are
      \item Some new materials are
\end{enumerate}

\allocation{B}
\begin{enumerate}
      \item[a.] coal, oil, and minerals
      \item[b.] gold, bronze, iron
      \item[c.] stones, wood, plants
      \item[d.] sand, hay, clay
      \item[e.] plastics, nylon, acrylic
      \item[f.] silicon, fibreglass, liquid crystals
      \item[g.] glass, metals and concrete
      \item[h.] cotton, wool, paper
\end{enumerate}

\subsection*{Solution}
\begin{enumerate}
      \item The first materials people used were \allocation{c.} \textbf{stones, wood, plants}.
      \item In the past, the bricks were made of \allocation{d.} \textbf{sand, hay, clay}.
      \item The first metals people got were \allocation{b.} \textbf{gold, bronze, iron}.
      \item The materials people learned to produce were \allocation{g.} \textbf{glass, metals and concrete}.
      \item The materials that we grow are \allocation{h.} \textbf{cotton, wool, paper}.
      \item The materials that we mine are \allocation{a.} \textbf{coal, oil, and minerals}.
      \item The materials made from chemicals that come from oil are \allocation{e.} \textbf{plastics, nylon, acrylic}.
      \item Some new materials are \allocation{f.} \textbf{silicon, fibreglass, liquid crystals}.
\end{enumerate}

\section{Exercise №5}

\subsection*{Choose the best option. Read the text and check your answers.}
\begin{enumerate}
      \item \allocation{What are materials?}
            \begin{enumerate}
                  \item useful substances
                  \item natural resources
                  \item liquids and solids
            \end{enumerate}
      \item \allocation{Choose what materials science doesn’t deal with.}
            \begin{enumerate}
                  \item using biological systems and living organisms to create different products
                  \item discovering and designing new materials
                  \item the properties and characteristics of all materials
            \end{enumerate}
      \item \allocation{How does materials science relate to technology?}
            \begin{enumerate}
                  \item All materials are a form of technology.
                  \item It has no relationship with technology.
                  \item It helps to develop technology by choosing the right material for the job. And man-made materials are a form of technology themselves.
            \end{enumerate}
      \item \allocation{Which of the following is not a classification of materials commonly used in \\ materials science?}
            \begin{enumerate}
                  \item ceramics
                  \item polymers
                  \item gemstones
            \end{enumerate}
      \item \allocation{In what way will the breakthroughs in materials science affect the future of technology?}
            \begin{enumerate}
                  \item Materials science won’t play an important role in the future.
                  \item They are likely to affect the future of technology significantly.
                  \item They will revolutionise the diagnosis of diseases caused by genetic factors.
            \end{enumerate}
\end{enumerate}

\subsection*{Solution}
\begin{enumerate}
      \item What are materials? \\
            \textbf{Answer:} \allocation{a.} useful substances
      \item Choose what materials science doesn’t deal with. \\
            \textbf{Answer:} \allocation{a.} using biological systems and living organisms to create different products
      \item How does materials science relate to technology? \\
            \textbf{Answer:} \allocation{c.} It helps to develop technology by choosing the right material for the job. And man-made materials are a form of technology themselves.
      \item Which of the following is not a classification of materials commonly used in materials science? \\
            \textbf{Answer:} \allocation{c.} gemstones
      \item In what way will the breakthroughs in materials science affect the future of technology? \\
            \textbf{Answer:} \allocation{b.} They are likely to affect the future of technology significantly.
\end{enumerate}

\section{Exercise №7}

Answer the following questions using the information from text 11A.
\begin{enumerate}
      \item What are materials?
      \item How important were materials in the past?
      \item How old is materials science?
      \item What does materials science primarily deal with?
      \item What are the main groups of materials?
      \item How do materials affect our daily life?
      \item What can the role of materials in our life be compared with?
      \item What examples of advanced materials does the author give and where are they used?
      \item What tools are modern scientists using to create advanced materials?
      \item In what fields do we see breakthroughs which are achieved due to the development of new materials?
      \item How will materials help us meet the challenges of the current world?
      \item Why can we say that the future is bright for materials science? Do you agree?
\end{enumerate}

\subsection*{Solution}
\begin{enumerate}
      \item What are materials? \\
            \allocation{Materials are substances having properties that make them useful in machines, \\ structures, devices, and other products.}
      \item How important were materials in the past? \\
            \allocation{Materials were crucial in the past, shaping human civilization and even defining \\ eras such as the Stone Age, Bronze Age, and Iron Age.}
      \item How old is materials science? \\
            \allocation{Materials science is one of the oldest forms of technology and applied science, \\ deriving from the manufacture of ceramics.}
      \item What does materials science primarily deal with? \\
            \allocation{Materials science primarily deals with discovering and designing new materials, \\ analyzing their properties and structure, and classifying materials into groups \\ such as metals, polymers, ceramics, and composites.}
      \item What are the main groups of materials? \\
            \allocation{The main groups of materials are metals, polymers, ceramics, and composites.}
      \item How do materials affect our daily life? \\
            \allocation{Materials are ubiquitous in the modern world and influence virtually every segment \\ of daily life, contributing to the advancement of various technologies and \\ sectors including medicine, communication, transportation, agriculture, and more.}
      \item What can the role of materials in our life be compared with? \\
            \allocation{The role of materials in our life can be compared to the generality of energy and \\ information, as all three comprise nearly all technology.}
      \item What examples of advanced materials does the author give and where are they used? \\
            \allocation{Examples of advanced materials include metamaterials used in carbon fiber \\ composites for lighter-weight vehicles, advanced alloys for more durable \\ jet engines, and biomaterials to replace human joints. These materials are used \\ in various industries such as aerospace, healthcare, and \\ robotics.}
      \item What tools are modern scientists using to create advanced materials? \\
            \allocation{Modern scientists are using powerful simulation techniques, sophisticated machine \\ learning algorithms, and advancements in nanotechnology to create advanced \\ materials.}
      \item In what fields do we see breakthroughs which are achieved due to the development of new materials? \\
            \allocation{Breakthroughs are seen in fields such as energy storage, quantum computing, \\ robotics, and aerospace due to the development of new materials.}
      \item How will materials help us meet the challenges of the current world? \\
            \allocation{Materials will help us meet the challenges of the current world by discovering and \\ developing new materials with desired properties to power industries and \\ technologies, thus addressing various global challenges.}
      \item Why can we say that the future is bright for materials science? Do you agree? \\
            \allocation{The future is bright for materials science because breakthroughs in this field are \\ likely to significantly affect the future of technology, power industries, \\ and address global challenges. I agree with this statement considering the ongoing \\ advancements and innovations in materials science and engineering.}
\end{enumerate}

\chapter{Домашнее задание №5 28.03.24}

\section{Exercise №6}

\subsection*{Find the words and phrases in the text which have the following meanings.}
\begin{enumerate}
      \item[§] \allocation{1}
            \begin{enumerate}
                  \item[1.] \textbf{a verb:} a physical material from which something is made, matter
                  \item[2.] \textbf{a noun:} a special quality or characteristic of something
                  \item[3.] \textbf{a noun:} an overstatement of the truth
                  \item[4.] \textbf{a passive verb:} to be formed or created
                  \item[5.] \textbf{a verb:} to have something as a source, to come from something
                  \item[6.] \textbf{a noun:} inorganic and nonmetallic materials that are essential to our daily lives
                  \item[7.] \textbf{a noun:} materials made up of two or more components
            \end{enumerate}
      \item[§] \allocation{2}
            \begin{enumerate}
                  \item[8.] \textbf{an adverb:} almost entirely, nearly
                  \item[9.] \textbf{a passive verb:} to be under the effect of something
                  \item[10.] \textbf{a verb:} to help something to happen or develop
                  \item[11.] \textbf{a noun:} the quality or state of being general
                  \item[12.] \textbf{a verb:} to be made up of something
            \end{enumerate}
      \item[§] \allocation{3}
            \begin{enumerate}
                  \item[13.] \textbf{a verb:} to include or deal with
                  \item[14.] \textbf{an adjective:} related to or based on competition
                  \item[15.] \textbf{a noun:} an ability to act or produce an effect
                  \item[16.] \textbf{a noun:} a substance composed of two or more metals or of a metal and a nonmetal
                  \item[17.] \textbf{a noun:} the dioxide of silicon SiO2
                  \item[18.] \textbf{a noun:} a nonmetallic chemical element with atomic number 6 that readily forms compounds with many other elements
            \end{enumerate}
      \item[§] \allocation{4, 5}
            \begin{enumerate}
                  \item[19.] \textbf{a verb:} to develop by a process of evolution
                  \item[20.] \textbf{a noun:} the act of doing a job, an activity, etc.
                  \item[21.] \textbf{a noun phrase:} the imitative representation of the functioning of a system or a process
                  \item[22.] \textbf{a noun:} something (such as an instrument or apparatus) used in performing an operation
                  \item[23.] \textbf{a verb:} to put something new in the place of something else
                  \item[24.] \textbf{a verb phrase:} to perform so as to succeed
                  \item[25.] \textbf{an adjective:} happening or existing now
                  \item[26.] \textbf{a verb:} to act on and cause a change in (someone or something)
                  \item[27.] \textbf{a noun:} the act or process of moving faster or happening more quickly
            \end{enumerate}
\end{enumerate}

\subsection*{Solution}
\begin{enumerate}
      \item form
      \item characteristic
      \item exaggeration
      \item to be formed or created
      \item derive
      \item minerals
      \item compounds
      \item virtually
      \item to be under the effect of something
      \item facilitate
      \item generality
      \item consist
      \item encompass
      \item competitive
      \item capability
      \item alloy
      \item silica
      \item carbon
      \item evolve
      \item operation
      \item simulation of operation
      \item apparatus
      \item replace
      \item to perform so as to succeed
      \item current
      \item influence
      \item acceleration
\end{enumerate}

\section{Лексика}

\subsection*{Text 11 A}
\begin{enumerate}
      \item acceleration (n)
      \item acquire (v)
      \item alloy (n)
      \item applied science (n)
      \item ceramics (n)
      \item comparable (adj)
      \item composite (n, adj)
      \item compound (n, adj)
      \item comprise (v)
      \item contribution (n)
      \item cover (v)
      \item derive from (v)
      \item enhance (v)
      \item exaggerate (v)
      \item exaggeration (n)
      \item field of science (n)
      \item general (adj)
      \item generality (n)
      \item materials science (n)
      \item performance (n)
      \item property (n)
      \item relevant (adj)
      \item shape (v)
      \item simulation (n)
      \item split (v)
      \item substance (n)
      \item take (v) something for granted
      \item meet (v) the challenges
      \item tool (n)
\end{enumerate}

\subsection*{Text 11 B}
\begin{enumerate}
      \item (electrical) resistance (n)
      \item resistant (adj)
      \item cause (v)
      \item coating (n)
      \item confront (v)
      \item join (v)
      \item keep (v) cool
      \item last (v)
      \item provide (v) an answer
      \item repair (v)
      \item replicate (v)
      \item save (v) money
      \item similar (adj)
      \item transparent (adj)
      \item unlike (prep)
      \item wrap (n, v)
\end{enumerate}

\subsection*{Text 11 C}
\begin{enumerate}
      \item consequence(s) (n)
      \item consequently (adv)
      \item constitute (v)
      \item transition (n)
      \item critical (adj) temperature
      \item encounter (v)
      \item expel (v)
      \item ground-breaking (adj)
      \item levitate (v)
      \item magnetic (adj) field (n)
      \item occur (v)
      \item precise (adj)
      \item reduce (v)
      \item stable (adj)
      \item superconductivity (n)
      \item vary (v)
\end{enumerate}

\section{Exercise №23}

\subsection*{Check how well you remember the words from the list below. Read and translate them or explain their meaning. Try to recall how they were used in text 11A filling the gaps in the sentences.}
\begin{enumerate}
      \item We define materials as \underline{\hspace{3cm}} having \underline{\hspace{3cm}} which make them useful in machines, structures, devices, and other products.
      \item It would be no \underline{\hspace{3cm}} to say that human civilization has been \underline{\hspace{3cm}} by breakthroughs in materials science.
      \item Materials science is one of the oldest forms of technology and an \underline{\hspace{3cm}} science, \underline{\hspace{3cm}} from manufacture of ceramics.
      \item It is concerned with a wide range of substances, from relatively easily \underline{\hspace{3cm}} wood or stone, to modern man-made materials.
      \item In the modern world materials are ubiquitous, and so pervasive that we often take them for \underline{\hspace{3cm}}.
      \item Materials have a \underline{\hspace{3cm}} comparable to that of energy and information, and the three together \underline{\hspace{3cm}} nearly all technology.
      \item Materials science also \underline{\hspace{3cm}} discovering and/or designing new materials and analyzing their properties and structure.
      \item Scientists are using powerful \underline{\hspace{3cm}} techniques, as well as sophisticated machine learning algorithms.
      \item These \underline{\hspace{3cm}} have helped create the metamaterials used in carbon fiber \underline{\hspace{1,5cm}} for lightweight vehicles.
      \item We need to discover and develop new kinds of materials with the desired properties and at the \underline{\hspace{3cm}} cost to \underline{\hspace{3cm}} the challenges of the current world.
      \item Today materials science is a dynamic and exciting \underline{\hspace{3cm}} with many remarkable new materials and discoveries.
      \item Innovative technologies enable engineers to improve the performance of existing products which will \underline{\hspace{3cm}} every aspect of our lives.
\end{enumerate}

\subsection*{Solution}
\begin{enumerate}
      \item We define materials as \textbf{substances} having \textbf{properties} which make them useful in machines, structures, devices, and other products.
      \item It would be no \textbf{exaggeration} to say that human civilization has been \textbf{shaped} by breakthroughs in materials science.
      \item Materials science is one of the oldest forms of technology and an \textbf{applied} science, \textbf{derived} from manufacture of ceramics.
      \item It is concerned with a wide range of substances, from relatively easily \textbf{acquired} wood or stone, to modern man-made materials.
      \item In the modern world materials are ubiquitous, and so pervasive that we often take them for \textbf{granted}.
      \item Materials have a \textbf{generality} comparable to that of energy and information, and the three together \textbf{comprise} nearly all technology.
      \item Materials science also \textbf{covers} discovering and/or designing new materials and analyzing their properties and structure.
      \item Scientists are using powerful \textbf{simulation} techniques, as well as sophisticated machine learning algorithms.
      \item These \textbf{tools} have helped create the metamaterials used in carbon fiber \textbf{composites} for lightweight vehicles.
      \item We need to discover and develop new kinds of materials with the desired properties and at the \textbf{relevant} cost to \textbf{meet} the challenges of the current world.
      \item Today materials science is a dynamic and exciting \textbf{field} with many remarkable new materials and discoveries.
      \item Innovative technologies enable engineers to improve the performance of existing products which will \textbf{enhance} every aspect of our lives.
\end{enumerate}

\section{Exercise №24}

\subsection*{Choose the words from Text 11A vocabulary list. Some phrases can be written as a solid word.}
\textbf{Across:}
\begin{enumerate}
      \item To consist of; be made up of.
      \item To learn or develop (a skill, habit, or quality).
      \item Something such as a piece of equipment or skill that is useful for doing a job.
      \item An imitation of a more practical applications, for example: technology or inventions.
      \item A discipline that is used to apply existing scientific knowledge to develop situation or process.
      \item To give a particular shape or form to something.
      \item Capable of or suitable for comparison.
      \item An increase in speed or rate.
\end{enumerate}

\textbf{Down:}
\begin{enumerate}
      \item To assume that something is true without questioning it.
      \item Physical material from which something is made.
      \item A metal made by combining two or more metallic elements, especially to give greater strength or resistance to corrosion.
      \item Belonging to or existing in the present time.
      \item An attribute, quality, or characteristic of something.
      \item To divide into two or more groups.
      \item The action or process of performing a task or function.
      \item A statement that represents something as better or worse than it really is.
      \item To make something better or improve the condition of something.
\end{enumerate}

\subsection*{Solution}
\begin{enumerate}
      \item Composite
      \item Acquire
      \item Tool
      \item Simulation
      \item Applied
      \item Shape
      \item Comparable
      \item Acceleration
      \item Takeforgranted
      \item Substance
      \item Alloy
      \item Current
      \item Property
      \item Split
      \item Derivefrom
      \item Join
      \item Repair
      \item Exaggeration
\end{enumerate}

\section{Exercise №25}

\subsection*{Read the sentences and choose the right option. Explain your choice or translate the sentences into Russian.}
\begin{enumerate}
      \item If a car turns a corner at a constant speed, it is \textbf{accelerating/assimilating} because its direction is changing.
      \item This view is very pessimistic and tends to \textbf{exaggerate/enhance} the difficulties.
      \item An example of a \textbf{simulation/stimulation} is a fire drill which is used to prepare people for an anticipated event.
      \item Some examples of \textbf{tools/skills} that are often used today are the hammer, saws, shovel, telephone, and the computer.
      \item The old word “abroad” which once meant “out of doors” has \textbf{accepted/acquired} a new meaning and today is used as an adverb meaning “beyond the boundaries of one’s country”.
      \item I go to the gym regularly but I don’t like \textbf{comparative/competitive} sports.
      \item The USA consists/\textbf{comprises} fifty states and one district.
      \item A fire is \textbf{competitive/comparable} with the sun; both give light and heat.
      \item Marie Curie is famous for her \textbf{connection/contribution} to science.
      \item Some types of wood \textbf{shape/split} easily.
      \item Bronze is an \textbf{alloy/compound} of copper and tin.
      \item Water is a chemical \textbf{composite/compound} of hydrogen and oxygen.
      \item I like games in \textbf{general/generality} and especially football.
      \item Strength and wisdom \textbf{derive/drive} from life force.
\end{enumerate}

\subsection*{Solution}
\begin{enumerate}
      \item If a car turns a corner at a constant speed, it is \textbf{accelerating} because its direction is changing.
      \item This view is very pessimistic and tends to \textbf{exaggerate} the difficulties.
      \item An example of a \textbf{simulation} is a fire drill which is used to prepare people for an anticipated event.
      \item Some examples of \textbf{tools} that are often used today are the hammer, saws, shovel, telephone, and the computer.
      \item The old word “abroad” which once meant “out of doors” has \textbf{acquired} a new meaning and today is used as an adverb meaning “beyond the boundaries of one’s country”.
      \item I go to the gym regularly but I don’t like \textbf{competitive} sports.
      \item The USA \textbf{comprises} fifty states and one district.
      \item A fire is \textbf{comparable} with the sun; both give light and heat.
      \item Marie Curie is famous for her \textbf{contribution} to science.
      \item Some types of wood \textbf{split} easily.
      \item Bronze is an \textbf{alloy} of copper and tin.
      \item Water is a chemical \textbf{compound} of hydrogen and oxygen.
      \item I like games in \textbf{general} and especially football.
      \item Strength and wisdom \textbf{derive} from life force.
\end{enumerate}

\chapter{Домашнее задание №6 04.04.24}

\section{Упражнение №2}
\subsection*{Значения терминов:}
\begin{enumerate}
      \item Memory cells: Ячейки памяти - это устройства, используемые для хранения данных в компьютерах. В контексте текста, они изготавливаются из сверхпроводящего материала и могут хранить информацию бесконечно долго.
      \item Windings: Здесь речь идет о "витках" или "обмотках" - это проводник, намотанный в спираль, используемый для проведения электрического тока, например, в трансформаторах.
      \item Coils: Катушки - это также спирально намотанные проводники, которые создают магнитное поле при прохождении электрического тока через них.
\end{enumerate}

\subsection*{Соответствия из колонки В:}
\begin{enumerate}
      \item induce - e. bring about (вызывать)
      \item remove - f. take off, away (удалять)
      \item indefinitely - b. unlimitedly (бесконечно)
      \item memory cell - g. the unit of computer which stores data for future use (ячейка памяти)
      \item retrieve - a. find, get back (извлекать)
      \item winding - d. length of wire wound in a spiral to conduct electric current (обмотка)
      \item coil - c. spiral (катушка)
\end{enumerate}

\subsection*{Синонимы в тексте:}
\begin{enumerate}
      \item Retrieve (извлечь) и get back (вернуть) в контексте извлечения информации из ячеек памяти.
      \item Indefinitely (бесконечно) и unlimitedly (неограниченно) в описании возможности хранения информации в ячейках памяти.
      \item Winding (обмотка) и coil (катушка) в описании частей трансформаторов, которые могут быть охлаждены для создания идеального трансформатора.
\end{enumerate}

\section{Упражнение №15}
\subsection*{Определите, к какой части речи относятся слова.}
\begin{enumerate}
      \item resistant, resist, resistance, resistor, resistivity;
      \item superconductivity, superconductive, superconductor, superconducting;
      \item theory, theorist, theoretical, theorize;
      \item physics, physicist, physical, physically;
      \item explain, explainable, explanation;
      \item store, storage, storable.
\end{enumerate}

\subsection*{Решение}
\begin{enumerate}
      \item \textbf{resistant} - прилагательное \\
            \textbf{resist} - глагол \\
            \textbf{resistance} - существительное \\
            \textbf{resistor} - существительное \\
            \textbf{resistivity} - существительное \\

      \item \textbf{superconductivity} - существительное \\
            \textbf{superconductive} - прилагательное \\
            \textbf{superconductor} - существительное \\
            \textbf{superconducting} - прилагательное \\

      \item \textbf{theory} - существительное \\
            \textbf{theorist} - существительное \\
            \textbf{theoretical} - прилагательное \\
            \textbf{theorize} - глагол \\

      \item \textbf{physics} - существительное \\
            \textbf{physicist} - существительное \\
            \textbf{physical} - прилагательное \\
            \textbf{physically} - наречие \\

      \item \textbf{explain} - глагол \\
            \textbf{explainable} - прилагательное \\
            \textbf{explanation} - существительное \\

      \item \textbf{store} - глагол \\
            \textbf{storage} - существительное \\
            \textbf{storable} - прилагательное \\
\end{enumerate}

\section{Упражнение №16}
\subsection*{Найдите русскому слову соответствующее английское.}

\begin{enumerate}
      \item \textbf{достижение} - achievable, achievement, achieve;
      \item \textbf{электронный} - electronics, electronic, electron;
      \item \textbf{легче} - easily, easy, easier;
      \item \textbf{удовлетворять} - satisfy, satisfactory, satisfaction;
      \item \textbf{действительно} - reality, realise, really.
\end{enumerate}

\subsection*{Solution}
\begin{enumerate}
      \item достижение - achievement
      \item электронный - electronic
      \item легче - easier
      \item удовлетворять - satisfy
      \item действительно - really
\end{enumerate}

\begin{enumerate}
      \item \textbf{достижение} (существительное) - achievable (прилагательное), achievement (существительное), achieve (глагол);
      \item \textbf{электронный} (прилагательное) - electronics (существительное), electronic (прилагательное), electron (существительное);
      \item \textbf{легче} (наречие) - easily (наречие), easy (прилагательное), easier (прилагательное);
      \item \textbf{удовлетворять} (глагол) - satisfy (глагол), satisfactory (прилагательное), satisfaction (существительное);
      \item \textbf{действительно} (наречие) - reality (существительное), realise (глагол), really (наречие).
\end{enumerate}

\section{Упражнение №17}
\subsection*{Переведите слова с суффиксом -ward (-wards), обозначающим направление.}
toward(s), forward(s), backward(s), afterward(s), downward(s),
outward(s), northward(s), southward(s), rearward(s), home-
ward(s), sideward(s), windward(s), upward(s).

\subsection*{Решение}
\begin{itemize}
      \item \textbf{toward(s)} --- к, в направлении
      \item \textbf{forward(s)} --- вперед, впереди
      \item \textbf{backward(s)} --- назад, позади
      \item \textbf{afterward(s)} --- после этого, впоследствии
      \item \textbf{downward(s)} --- вниз, внизу
      \item \textbf{outward(s)} --- наружу, вовне
      \item \textbf{northward(s)} --- на север, севернее
      \item \textbf{southward(s)} --- на юг, южнее
      \item \textbf{rearward(s)} --- назад, задне
      \item \textbf{homeward(s)} --- к дому, домой
      \item \textbf{sideward(s)} --- вбок, боком
      \item \textbf{windward(s)} --- навстречу ветру
      \item \textbf{upward(s)} --- вверх, вверху
\end{itemize}

\section{Упражнение №18}
\subsection*{Найдите слова с нестандартным образованием множественного числа.}
There are a few words taken over from Latin and Greek that
still retain their original plurals in English. In some cases we can
use either. Formulas is seen more often than formulae. Antenna -
antennae (pl ). Many think that media, strata and phenomena are
all singular. They aren't. Data, a plural, is used both ways.

Here are some foreign singular and plural forms of words often
used in English. Latin: medium (a means of mass communication)
- media, nucleus (ядро atoмa) - nuclei; Greek: analysis
analyses; axis - axes; crisis - crises; hypothesis - hypotheses;
phenomenon - phenomena.

\subsection*{Решение}
\begin{itemize}
      \item \textbf{formula} (обычно множественное число - formulas, хотя иногда также используется formulae) - [\textipa{ˈfɔːmjʊlə}] ([\textipa{ˈfɔːmjʊləz}], [\textipa{ˈfɔːmjʊliː}])
      \item \textbf{antenna} (множественное число - antennae) - [\textipa{ænˈtɛnə}] ([\textipa{ænˈtɛniː}])
      \item \textbf{medium} (множественное число - media) - [\textipa{ˈmiːdɪəm}] ([\textipa{ˈmiːdiə}])
      \item \textbf{nucleus} (множественное число - nuclei) - [\textipa{ˈnjuːklɪəs}] ([\textipa{ˈnjuːkliːaɪ}])
      \item \textbf{analysis} (множественное число - analyses) - [\textipa{əˈnælɪsɪs}] ([\textipa{əˈnælɪsiːz}])
      \item \textbf{crisis} (множественное число - crises) - [\textipa{ˈkraɪsɪs}] ([\textipa{ˈkraɪsiːz}])
      \item \textbf{hypothesis} (множественное число - hypotheses) - [\textipa{haɪˈpɒθɪsɪs}] ([\textipa{haɪˈpɒθɪsiːz}])
      \item \textbf{phenomenon} (множественное число - phenomena) - [\textipa{fɪˈnɒmɪnən}] ([\textipa{fɪˈnɒmɪnə}])
\end{itemize}

\section{Упражнение №19}
\subsection*{Найдите синонимы и антонимы.}
below - above; useful - useless; easy - difficult; field -
sphere; to meet demands - to meet requirements (needs); full -
complete; to use - to apply; to get - to obtain; moreover - be-
sides; sufficient - enough; likely - unlikely; to continue - to dis-
continue; conductivity - non conductivity; to vary - to change; to
lead to - to result in; recent - latest; advantage - disadvantage;
low - high; believable - unbelievable; to lose - to find; tiny -
huge; liquid - solid; unexpected - expected; common - ordinary.

\subsection*{Решение}
\textbf{Синонимы:}
\begin{itemize}
      \item below - above
      \item to meet demands - to meet requirements (needs)
      \item to use - to apply
      \item to get - to obtain
      \item sufficient - enough
      \item to vary - to change
      \item to lead to - to result in
      \item recent - latest
      \item low - high
      \item to lose - to find
      \item liquid - solid
\end{itemize}

\textbf{Антонимы:}
\begin{itemize}
      \item useful - useless
      \item easy - difficult
      \item field - sphere
      \item full - complete
      \item moreover - besides
      \item likely - unlikely
      \item to continue - to discontinue
      \item conductivity - non conductivity
      \item advantage - disadvantage
      \item believable - unbelievable
      \item tiny - huge
      \item unexpected - expected
      \item common - ordinary
\end{itemize}

\section{Write 3-5 sentences using these word combinations}
\begin{itemize}
      \item the physics discoveries
      \item discoveries that led to
      \item the scientific advantage
      \item advantage could well come to nation
      \item to bring the mankind to
      \item mercury wire
      \item unexpected phenomenon
      \item to return to normal state
      \item by passing electric current
      \item by applying magnetic field
      \item to make a great contribution
      \item they introduced a model
      \item a model proved to be useful
      \item a theory won for them the Nobel Prize
      \item research in superconductivity
      \item research became especially active
      \item the achieved record of 23 K.
\end{itemize}

\subsection*{Solution}
The recent physics discoveries have sparked interest worldwide. These discoveries, which led to groundbreaking advancements, provide a scientific advantage in various fields. Such an advantage could well come to our nation, bringing mankind closer to understanding complex phenomena. Researchers observed an unexpected phenomenon involving a mercury wire, which returned to its normal state when electric current passed through it or a magnetic field was applied. Their research in superconductivity became especially active, culminating in the achieved record of 23 K, making a great contribution to the field.

\section{Табличка на словообразование - D}
\subsection*{Заполните таблицу на словообразование.}
\begin{center}
      \begin{tabular}{|c|c|c|}
            \hline
            Verb & Noun         & Adjective \\
            \hline
            ...  & retrieval    & ...       \\
            ...  & removal      & ...       \\
            ...  & definition   & ...       \\
            ...  & resistance   & ...       \\
            ...  & applicant    & ...       \\
            ...  & conduction   & ...       \\
            ...  & presence     & ...       \\
            ...  & explanation  & ...       \\
            ...  & belief       & ...       \\
            ...  & introduction & ...       \\
            \hline
      \end{tabular}
\end{center}

\subsection*{Solution}
\begin{center}
      \begin{tabular}{|c|c|c|}
            \hline
            Verb      & Noun         & Adjective    \\
            \hline
            Retrieve  & retrieval    & retrievable  \\
            Remove    & removal      & removable    \\
            Define    & definition   & definable    \\
            Resist    & resistance   & resistant    \\
            Apply     & applicant    & applicable   \\
            Conduct   & conduction   & conductive   \\
            Present   & presence     & presentable  \\
            Explain   & explanation  & explanatory  \\
            Believe   & belief       & believable   \\
            Introduce & introduction & introductory \\
            \hline
      \end{tabular}
\end{center}

\chapter{Семинар №7 04.04.24}

\section{Exercise №12}
\subsection*{Match the words in column B with the correct definitions in column A:}

\begin{center}
      \begin{tabular}{|c|c|}
            \hline
            \textbf{A}               & \textbf{B}                                    \\
            \hline
            1. thick                 & c (the opposite of thin)                      \\
            2. expensive             & d (costing a lot of money)                    \\
            3. super-thin            & e (extremely thin)                            \\
            4. keep something (cool) & b (make or become less hot / colder)          \\
            5. unlike                & a (different from)                            \\
            6. cool down             & f (cause to continue in a specific condition) \\
            \hline
      \end{tabular}
\end{center}

\begin{center}
      \begin{tabular}{|c|c|}
            \hline
            \textbf{A}                & \textbf{B}                                            \\
            \hline
            1. cotton, linen and wool & c (natural fabric(s))                                 \\
            2. similar                & f (being almost the same)                             \\
            3. a coating              & a (a substance that covers a surface)                 \\
            4. to save (e.g. money)   & e (to keep for use in the future)                     \\
            5. to repair              & g (to put something damaged back into good condition) \\
            6. elastic                & b (able to stretch and return to its original shape)  \\
            7. to join                & h (to connect things together)                        \\
            8. to last                & d (to continue to exist)                              \\
            \hline
      \end{tabular}
\end{center}

\begin{center}
      \begin{tabular}{|c|c|}
            \hline
            \textbf{A}           & \textbf{B}                                                        \\
            \hline
            1. to cause          & e (to make something happen)                                      \\
            2. wrap              & c (material that is used to cover or protect objects)             \\
            3. resistance        & d (a force that stops the progress of something)                  \\
            4. to attract        & h (to pull or draw something towards them)                        \\
            5. to prevent        & g (to keep from happening or existing)                            \\
            6. to cut incidences & a (to make the events less frequent)                              \\
            7. to replicate      & b (to do something again in exactly the same way)                 \\
            8. transparent       & f (allowing light through so that objects can be seen through it) \\
            \hline
      \end{tabular}
\end{center}

\subsection*{Solution}

\allocation{1.}
\begin{enumerate}
      \item \textbf{Thick:} The opposite of thin.
      \item \textbf{Expensive:} Costing a lot of money.
      \item \textbf{Super-thin:} Extremely thin.
      \item \textbf{Keep something (cool):} Make or become less hot/colder.
      \item \textbf{Unlike:} Different from.
      \item \textbf{Cool down:} Cause to continue in a specific condition.
\end{enumerate}

\allocation{2.}
\begin{enumerate}
      \item \textbf{Cotton, linen and wool:} Natural fabric(s).
      \item \textbf{Similar:} Being almost the same.
      \item \textbf{A coating:} A substance that covers a surface.
      \item \textbf{To save (e.g. money):} Keep for use in the future.
      \item \textbf{To repair:} Put something damaged back into good condition.
      \item \textbf{Elastic:} Able to stretch and return to its original shape.
      \item \textbf{To join:} To connect things together.
      \item \textbf{To last:} To continue to exist.
\end{enumerate}

\allocation{3.}
\begin{enumerate}
      \item \textbf{To cause:} To make something happen.
      \item \textbf{Wrap:} Material that is used to cover or protect objects.
      \item \textbf{Resistance:} A force that stops the progress of something.
      \item \textbf{To attract:} To pull or draw something towards them.
      \item \textbf{To prevent:} To keep from happening or existing.
      \item \textbf{To cut incidences:} To make the events less frequent.
      \item \textbf{To replicate:} To do something again in exactly the same way.
      \item \textbf{Transparent:} Allowing light through so that objects can be seen through it (e.g. glass).
\end{enumerate}

\section{Text 11 B}
\subsection*{New Materials That Could Change Our Lives}

\allocation{(1)} New, super-thin material cools buildings. A team of engineers has created a 3. super-thin material that could help 4. keep buildings cool even under direct sunlight. The engineers say the new material could provide an answer to air conditioners, which are 2. expensive to run and need a lot of water. The material is 5. unlike anything found in nature. It is a glass-polymer hybrid that is just fifty micrometers 1. thick. That's slightly thicker than the aluminium foil we use for cooking. The key advantage of this technology is that it works 24/7 with no electricity or water usage and can cool objects even under direct sunlight. Just ten to twenty square metres of this material on the rooftop could nicely 4. keep a house in summer.

\allocation{(2)} Scientists make self-repairing clothes. Humans have learnt many things from nature. These things have helped us in our daily life. The latest thing is self-repairing clothing. Scientists have developed a special way for clothing 8. to repair rips and tears by itself, without the need for sewing. It works with materials such as cotton, 1. linen , and wool. Scientists looked at how squid can cling on to things so well. The research team found a protein in the rings of teeth that cover the suckers on a squid. The protein is 2. similar to the one found in the silk that spiders use to make spider webs. It is very strong and 6. elastic. The new protein has been developed as part of a 3. coating , which is put on clothes. When the coating is dipped in water, the area around the rip or tear 7. joins together in less than a minute. This could help clothes 8. last longer and 4. save us money. It could also be useful for military and survival clothes.

\allocation{(3)} Self-Cleaning Plastic. A revolutionary new plastic could help 5. prevent bacteria and superbugs 4. to attract disease and illness. Scientists have developed a new kind of A. wrap, plastic wonder-wrap. They say it will drastically 6. cut incidences of microbe transfer in hospitals, restaurants, kitchens, bathrooms, and other places where bugs lie in wait. The material is like a conventional transparent B. transparent used to cover food. It can be shrink-wrapped to protect places that 4. attract bacteria, like worktops, door handles, taps, hospital equipment, and food containers. The researchers said the inspiration for their new material came from the lotus plant. They attempted 7. to replicate the method in which the leaves of this plant repelled water. As the world confronts the crisis of anti-microbial 3. resistance, new material will become an important part of the anti-bacterial toolbox.

\section{Exercise №14}
\subsection*{Read text 11 B again and answer the questions below choosing the best option according
      to the information given in the text. Compare your answers in pairs or groups.}

\allocation{1.}
\begin{enumerate}
      \item \textbf{Who created the super-thin material?}
            \begin{enumerate}
                  \item[a)] computer scientists
                  \item[b)] a team of engineers
                  \item[c)] Microsoft
            \end{enumerate}
      \item \textbf{According to the text, what is not required for the cooling material to function?}
            \begin{enumerate}
                  \item[a)] sunlight
                  \item[b)] human labour
                  \item[c)] energy and water
            \end{enumerate}
      \item \textbf{What concerns could the new material provide a solution to?}
            \begin{enumerate}
                  \item[a)] the meaning of life
                  \item[b)] air conditioners
                  \item[c)] shortage of water
            \end{enumerate}
      \item \textbf{How thick is the new material?}
            \begin{enumerate}
                  \item[a)] 15 micrometres
                  \item[b)] 50 millimetres
                  \item[c)] 50 micrometres
            \end{enumerate}
      \item \textbf{What is the new material slightly thicker than?}
            \begin{enumerate}
                  \item[a)] aluminium foil
                  \item[b)] cardboard
                  \item[c)] a piece of paper
            \end{enumerate}
      \item \textbf{How often will this new material work?}
            \begin{enumerate}
                  \item[a)] five days a week
                  \item[b)] during daylight hours
                  \item[c)] twenty four hours, seven days a week
            \end{enumerate}
      \item \textbf{How much of the material could cool down a house in summer?}
            \begin{enumerate}
                  \item[a)] enough to cover a football field
                  \item[b)] 10 to 20 square metres
                  \item[c)] 24/7
            \end{enumerate}
\end{enumerate}

\allocation{2.}
\begin{enumerate}
      \item \textbf{From what have humans gained a lot of knowledge?}
            \begin{enumerate}
                  \item[a)] books
                  \item[b)] nature
                  \item[c)] experiments
            \end{enumerate}
      \item \textbf{What is needed for the self-repairing clothes to fix themselves?}
            \begin{enumerate}
                  \item[a)] water
                  \item[b)] sewing
                  \item[c)] glue
            \end{enumerate}
      \item \textbf{According to the text, what natural material is commonly used for the clothes besides cotton and linen?}
            \begin{enumerate}
                  \item[a)] nylon
                  \item[b)] viscose
                  \item[c)] wool
            \end{enumerate}
      \item \textbf{From which sea creature did scientists take inspiration for the concept of self-repairing clothes?}
            \begin{enumerate}
                  \item[a)] whale
                  \item[b)] squid
                  \item[c)] jellyfish
            \end{enumerate}
      \item \textbf{What type of silk contains the protein similar to the one found in a squid?}
            \begin{enumerate}
                  \item[a)] spider web silk
                  \item[b)] Japanese silk
                  \item[c)] synthetic silk
            \end{enumerate}
      \item \textbf{How long does it take for a rip or tear to join together?}
            \begin{enumerate}
                  \item[a)] an hour or so
                  \item[b)] a few minutes
                  \item[c)] less than a minute
            \end{enumerate}
      \item \textbf{What potential benefits could the new invention offer us?}
            \begin{enumerate}
                  \item[a)] time
                  \item[b)] money
                  \item[c)] material
            \end{enumerate}
\end{enumerate}

\allocation{3.}
\begin{enumerate}
      \item \textbf{What could the new plastic prevent?}
            \begin{enumerate}
                  \item[a)] pandemics
                  \item[b)] illnesses
                  \item[c)] bacteria
            \end{enumerate}
      \item \textbf{How much transparency does the revolutionary new plastic have?}
            \begin{enumerate}
                  \item[a)] completely non-transparent
                  \item[b)] somewhat transparent
                  \item[c)] fully transparent
            \end{enumerate}
      \item \textbf{By how much did the article say the plastic would cut microbe transfer?}
            \begin{enumerate}
                  \item[a)] drastically
                  \item[b)] a little
                  \item[c)] totally
            \end{enumerate}
      \item \textbf{What is done to the plastic before it is applied to surfaces?}
            \begin{enumerate}
                  \item[a)] it is sprayed with water
                  \item[b)] it is shrunk
                  \item[c)] it is folded
            \end{enumerate}
      \item \textbf{What types of equipment did the article mention that the plastic could be used to cover?}
            \begin{enumerate}
                  \item[a)] hospital equipment
                  \item[b)] computer equipment
                  \item[c)] sports equipment
            \end{enumerate}
      \item \textbf{What served as the inspiration for the development of this plastic?}
            \begin{enumerate}
                  \item[a)] the water lily
                  \item[b)] the sunflower
                  \item[c)] the lotus plant
            \end{enumerate}
      \item \textbf{In what will the plastic play a significant role?}
            \begin{enumerate}
                  \item[a)] restaurant hygiene
                  \item[b)] medicine
                  \item[c)] an anti-bacterial toolbox
            \end{enumerate}
\end{enumerate}

\subsection*{Solution}

\begin{enumerate}
      \item[b)] a team of engineers
      \item[c)] energy and water
      \item[b)] air conditioners
      \item[c)] 50 micrometres
      \item[a)] aluminium foil
      \item[c)] twenty four hours, seven days a week
      \item[b)] 10 to 20 square metres
\end{enumerate}

\begin{enumerate}
      \item[b)] nature
      \item[a)] water
      \item[c)] wool
      \item[b)] squid
      \item[a)] spider web silk
      \item[c)] less than a minute
      \item[b)] money
\end{enumerate}

\begin{enumerate}
      \item[c)] bacteria
      \item[c)] fully transparent
      \item[a)] drastically
      \item[b)] it is shrunk
      \item[a)] hospital equipment
      \item[c)] the lotus plant
      \item[c)] an anti-bacterial toolbox
\end{enumerate}

\section{Exercise №27}
\subsection*{Replace the words in bold with their synonyms using the words given below. Translate the sentences into Russian.}
\begin{enumerate}
      \item This car has \textbf{comparable} features to the other one, but it is much cheaper.
      \item If you \textbf{connect} the dots on the paper, you'll get a picture.
      \item He is taking the car to the garage to have it \textbf{serviced} this afternoon.
      \item All the information you need is \textbf{stored} in this folder.
      \item She's very \textbf{different} from her sister.
      \item Researchers tried many times to \textbf{repeat} the original experiment.
      \item Make sure that shirt isn't \textbf{see-through} when it gets wet.
      \item Many illnesses are \textbf{brought about} by poor diet and lack of exercise.
      \item My friend found a well-paid job but asked me not to \textbf{tell anybody about it yet}.
      \item When they arrived the meeting had been \textbf{going on} for two hours already.
\end{enumerate}

\subsection*{Solution}
\begin{enumerate}
      \item Эта машина обладает \textbf{сопоставимыми} характеристиками с другой, но она намного дешевле.
      \item Если вы \textbf{соедините} точки на бумаге, вы получите картину.
      \item Он отвозит машину в гараж, чтобы \textbf{провести} техобслуживание этим вечером.
      \item Вся информация, которая вам нужна, \textbf{хранится} в этой папке.
      \item Она очень \textbf{отличается} от своей сестры.
      \item Исследователи много раз пытались \textbf{воспроизвести} оригинальный эксперимент.
      \item Убедитесь, что рубашка не \textbf{прозрачна}, когда она мокрая.
      \item Многие заболевания \textbf{возникают} из-за неправильного питания и недостатка физических упражнений.
      \item Мой друг нашел \textbf{хорошо оплачиваемую} работу, но попросил меня пока никому не рассказывать об этом.
      \item Когда они прибыли, собрание уже \textbf{продолжалось} два часа.
\end{enumerate}

\section{Exercise №32}
\subsection*{Summarise in English using some key words from the vocabulary section.}

\begin{enumerate}
      \item Американские и шведские инженеры создали мягкую роботизированную ткань, которая
            может запоминать и воспроизводить движения владельца. Ткань состоит из полимерной
            оболочки, обладающей весьма сложной структурой и слоя мягкого материала, для которого
            растяжение равносильно изменению электрического сопротивления. Также в материал
            встроена эластомерная трубка. Ткань получила название Omni Fiber. Ее волокна отличаются
            особой тонкостью и гибкостью. Ожидается, что ткань будут применять для создания оде��ды
            для спортсменов или певцов.

      \item Ученые придумали новое революционное применение древесины. Они придумали способ
            сделать её прозрачной. Это изобретение может полностью изменить то, как используются и
            производятся многие вещи в нашей жизни. Прозрачная древесина может однажды заменить
            стекло и использоваться в изготовлении окон и столов, для экранов телефонов и целого ряда
            других строительных материалов. Инновация возникла, когда исследователи
            экспериментировали с различными способами извлечения химических веществ из
            древесины, придающих ей цвет. Они были очень удивлены тем, насколько прозрачной может
            быть древесина. Они считают, что эта «новая» древесина потенциально может заменить
            стекло и некоторые оптические материалы. Прозрачное дерево намного прочнее и менее
\end{enumerate}

\subsection*{Solution}
American and Swedish engineers developed a soft robotic fabric called Omni Fiber, composed of a polymer shell with a complex structure and a layer of stretchable material. It also incorporates an elastomeric tube. The fabric's fibers are exceptionally thin and flexible, expected to be used in sportswear or performance clothing.

Scientists devised a revolutionary application for wood, making it transparent, potentially replacing glass in windows, screens, and various construction materials. The innovation arose from experiments extracting chemicals from wood, making it surprisingly transparent. Transparent wood, stronger and safer than glass, provides better insulation and is biodegradable. This discovery is still in its early stages.

\textbf{Conditional Sentences:}

Study the examples of four types of Conditional sentences below and answer the questions.

\begin{enumerate}
      \item If we created more advanced materials, they would accelerate the development of industries from energy to manufacturing.
      \item If substances are used to make products, they are called materials.
      \item If new materials hadn’t been discovered, a lot of products and technologies wouldn’t have been developed.
      \item If new tools to create materials continue to be used, we will soon see more breakthroughs in different fields.
\end{enumerate}

\allocation{Questions:}
\begin{itemize}
      \item What do they all have in common?
      \item How are they different?
      \item Which example expresses
            \begin{itemize}
                  \item General truth;
                  \item Imagined future situation which is quite likely;
                  \item Hypothetical situation which is unlikely;
                  \item Hypothetical outcome.
            \end{itemize}
\end{itemize}

\allocation{Answers:}
\begin{enumerate}
      \item All examples express conditional statements using "if" clauses, with each portraying a different type of condition and outcome.
      \item
            \begin{itemize}
                  \item Example 1: Imagined future situation which is quite likely.
                  \item Example 2: General truth.
                  \item Example 3: Hypothetical situation which is unlikely.
                  \item Example 4: Hypothetical outcome.
            \end{itemize}
\end{enumerate}

\chapter{Домашнее задание №7 11.04.24}

\section{Exercise №34}
\subsection*{Look at the examples of different types of Conditionals and fill in the blanks. Add 2-3
      examples of your own of each type of Conditionals.}
\begin{enumerate}
      \item We use Type \underline{\hspace{1cm}} to talk about future situations when we believe it is quite likely. (probable future)
      \item We use Type \underline{\hspace{1cm}} to talk about past situations that didn't happen. (unreal for the past)
      \item We use Type \underline{\hspace{1cm}} to talk about things that are true, that have happened, or are very likely to happen.
      \item We use Type \underline{\hspace{1cm}} to talk about the possible result of an imagined situation in the present or future. (unreal for the present or future)
\end{enumerate}

\subsection*{Solution}
\begin{enumerate}
      \item We use Type \textbf{0} to talk about things that are true, that have happened, or are very likely to happen.
      \item We use Type \textbf{3} to talk about past situations that didn't happen. (unreal for the past)
      \item We use Type \textbf{1} to talk about future situations when we believe it is quite likely. (probable future)
      \item We use Type \textbf{4} to talk about the possible result of an imagined situation in the present or future. (unreal for the present or future)
\end{enumerate}



\section{Exercise №35}
\subsection*{Focus on the verb forms in different types of Conditionals in the examples below and fill
      in the table.}
\begin{enumerate}
      \item If we cool certain materials to absolute zero, they become superconductors. (Zero conditional)
      \item If the economy is growing by 6\%, then it is growing too fast. (Zero conditional)
      \item If new technologies don’t guarantee safety, they won’t be adopted. (Type 1)
      \item If car makers solve some technical problems, electric cars will soon replace petrol cars. (Type 1)
      \item If science fiction became science fact immediately, we would be living in an age of flying cars. (Type 2)
      \item If robots could think or decide to do things differently, they would replace humans. (Type 2)
      \item If self-cleaning plastic was widely used, it would drastically cut the incidences of microbe transfer in hospitals and restaurants. (Type 2)
      \item If robots hadn’t been fitted with vision equipment, they wouldn’t have been able to see. (Type 3)
      \item If a laser had been used for the operation, it would have caused less harm. (Type 3)
      \item If internal combustion engine had not been invented, electric cars would have dominated our roads. (Type 3)
\end{enumerate}

\subsection*{Solution}
\begin{table}[htbp]
      \centering
      \begin{tabular}{|c|c|c|c|}
            \hline
            Example & Type of Conditional & IF CLAUSE Verb Form & RESULT (MAIN) CLAUSE Verb Form \\
            \hline
            1       & Zero                & present simple      & present simple                 \\
            2       & Zero                & present simple      & present simple                 \\
            3       & Type 1              & present simple      & future simple                  \\
            4       & Type 1              & present simple      & future simple                  \\
            5       & Type 2              & past simple         & would + base form              \\
            6       & Type 2              & past simple         & would + base form              \\
            7       & Type 2              & past simple         & would + base form              \\
            8       & Type 3              & past perfect        & would have + past participle   \\
            9       & Type 3              & past perfect        & would have + past participle   \\
            10      & Type 3              & past perfect        & would have + past participle   \\
            \hline
      \end{tabular}
\end{table}

\section{Exercise №36}
\subsection*{Look at more examples of Conditionals noticing the verb forms. Identify their types and decide whether the action expressed in the sentence is:}
\begin{enumerate}
      \item[a)] likely/possible
      \item[b)] less likely/less possible
      \item[c)] impossible
\end{enumerate}

\begin{enumerate}
      \item If scientists use modern tools to propel innovations forward, they will lead us to more remarkable possibilities.
      \item If a new super-thin material was used to cool our buildings, we would be able to save a significant amount of energy and money.
      \item If advanced materials hadn’t been created, smartphones wouldn’t have become accessible to billions of people.
      \item Materials science wouldn’t have played a key role in shaping human civilization if many remarkable new materials hadn’t been developed.
      \item If we had self-repairing clothes, there would be no need for sewing.
      \item If a self-cleaning plastic were applied in hospitals, it could help prevent bacteria and superbugs from causing diseases.
      \item If technology weren’t advancing so rapidly, we wouldn’t struggle so much to keep pace with it.
      \item If a new technique for printing organic tissue was created, scientists would be able to reproduce the body’s organs via the use of 3D printing.
      \item If our climate keeps getting warmer, we will soon require new technologies to cool our buildings.
      \item If humans had not learned from nature, they would not have invented so many incredible technologies.
      \item If the problem of global warming is not dealt with, our world will be a much more dangerous and difficult place to live in.
      \item If electric cars become dominant, our cities will turn into much cleaner and quieter places.
\end{enumerate}

\subsection*{Solution}
\begin{enumerate}
      \item[a)] If scientists use modern tools to propel innovations forward, they will lead us to more remarkable possibilities.\\
            \textit{Translation: Если ученые используют современные инструменты для продвижения инноваций вперед, они приведут нас к более замечательным возможностям.}
      \item[b)] If a new super-thin material was used to cool our buildings, we would be able to save a significant amount of energy and money.\\
            \textit{Translation: Если бы для охлаждения наших зданий использовался новый сверхтонкий материал, мы могли бы сэкономить значительное количество энергии и денег.}
      \item[c)] If advanced materials hadn’t been created, smartphones wouldn’t have become accessible to billions of people.\\
            \textit{Translation: Если бы передовые материалы не были созданы, смартфоны не стали бы доступны миллиардам людей.}
      \item[c)] Materials science wouldn’t have played a key role in shaping human civilization if many remarkable new materials hadn’t been developed.\\
            \textit{Translation: Наука о материалах не сыграла бы ключевую роль в формировании человеческой цивилизации, если бы не было разработано много замечательных новых материалов.}
      \item[b)] If we had self-repairing clothes, there would be no need for sewing.\\
            \textit{Translation: Если бы у нас была одежда с возможностью самостоятельного ремонта, не было бы необходимости шить.}
      \item[b)] If a self-cleaning plastic were applied in hospitals, it could help prevent bacteria and superbugs from causing diseases.\\
            \textit{Translation: Если бы в больницах применялся самоочищающийся пластик, это могло бы помочь предотвратить бактерии и супербактерии, вызывающие заболевания.}
      \item[b)] If technology weren’t advancing so rapidly, we wouldn’t struggle so much to keep pace with it.\\
            \textit{Translation: Если бы технологии не развивались так быстро, нам не пришлось бы так сильно бороться, чтобы быть в тренде.}
      \item[b)] If a new technique for printing organic tissue was created, scientists would be able to reproduce the body’s organs via the use of 3D printing.\\
            \textit{Translation: Если бы была создана новая техника для печати органической ткани, ученые смогли бы воспроизводить органы тела с помощью 3D-печати.}
      \item[a)] If our climate keeps getting warmer, we will soon require new technologies to cool our buildings.\\
            \textit{Translation: Если наш климат будет продолжать нагреваться, нам скоро понадобятся новые технологии для охлаждения наших зданий.}
      \item[b)] If humans had not learned from nature, they would not have invented so many incredible technologies.\\
            \textit{Translation: Если бы люди не учились у природы, они не изобрели бы так много невероятных технологий.}
      \item[a)] If the problem of global warming is not dealt with, our world will be a much more dangerous and difficult place to live in.\\
            \textit{Translation: Если проблема глобального потепления не будет решена, наш мир станет гораздо более опасным и сложным местом для жизни.}
      \item[a)] If electric cars become dominant, our cities will turn into much cleaner and quieter places.\\
            \textit{Translation: Если электромобили станут доминирующим��, наши города станут намного чище и тише.}
\end{enumerate}

\section{Exercise №37}
\subsection*{Match the clauses below. Identify the types of Conditionals and explain their meaning.}

\subsection*{A. Match the Clauses}
\begin{enumerate}
      \item If we had more students …
      \item My teacher wouldn’t have been angry with me …
      \item If I have lots of money in the future …
      \item She wouldn’t have been given the current position in the company …
      \item If you heat water to 100°C …
      \item If the weather is fine …
      \item If she doesn’t get a good night’s sleep …
      \item If I still feel awful tomorrow …
      \item If people didn’t drive so fast on this road …
\end{enumerate}

\begin{enumerate}
      \item[a.] I’ll take a trip around the world.
      \item[b.] we would run the course.
      \item[c.] if I had come to my class on time.
      \item[d.] we can go to the country tomorrow.
      \item[e.] she’s always tired in the mornings.
      \item[f.] there wouldn’t be so many accidents.
      \item[g.] if she had been lazy and talentless.
      \item[h.] I’ll take the day off work.
      \item[i.] it boils.
\end{enumerate}

\subsection*{B. Identify the Types of Conditionals}

\begin{enumerate}
      \item If she doesn’t pass the exam this year …
      \item If I had the time …
      \item If he hadn’t done engineering …
      \item I would never have bought this car …
      \item If you don’t book now…
      \item If the rent had been lower…
      \item If the flight is late…
      \item If it snows…
      \item If you take another week off work …
\end{enumerate}

\begin{enumerate}
      \item[a.] if I’d known how much petrol it uses.
      \item[b.] I’d love to learn to play tennis.
      \item[c.] what would he have studied?
      \item[d.] you won’t get good tickets.
      \item[e.] I would have taken the flat.
      \item[f.] we’ll miss our connection.
      \item[g.] we get our skis out.
      \item[h.] she can try again next year.
      \item[i.] the boss will definitely fire you.
\end{enumerate}

\subsection*{Solution}

\allocation{A. Matching the clauses:}
\begin{enumerate}
      \item If we had more students … (\textit{f. there wouldn’t be so many accidents.})
            \textbf{Type:} Second Conditional \\
            \textbf{Meaning:} This is a second conditional statement, indicating a hypothetical situation in the present or future and its probable result.

      \item My teacher wouldn’t have been angry with me … (\textit{c. if I had come to my class on time.})
            \textbf{Type:} Third Conditional \\
            \textbf{Meaning:} This is a third conditional statement, expressing a situation that didn't happen in the past and its imaginary result.

      \item If I have lots of money in the future … (\textit{a. I`ll take a trip around the world.})
            \textbf{Type:} First Conditional \\
            \textbf{Meaning:} This is a first conditional statement, presenting a likely future situation and its potential outcome.

      \item She wouldn't have been given the current position in the company … (\textit{g. if she had been lazy and talentless.})
            \textbf{Type:} Third Conditional \\
            \textbf{Meaning:} This is a third conditional statement, indicating a situation in the past that didn't occur and its imaginary result.

      \item If you heat water to 100°C … (\textit{i. it boils.})
            \textbf{Type:} Zero Conditional \\
            \textbf{Meaning:} This is a zero conditional statement, stating a general truth or scientific fact.

      \item If the weather is fine … (\textit{d. we can go to the country tomorrow.})
            \textbf{Type:} First Conditional \\
            \textbf{Meaning:} This is a first conditional statement, presenting a likely future situation and its potential outcome.

      \item If she doesn’t get a good night’s sleep … (\textit{e. she’s always tired in the mornings.})
            \textbf{Type:} First Conditional \\
            \textbf{Meaning:} This is a first conditional statement, presenting a likely future situation and its potential outcome.

      \item If I still feel awful tomorrow … (\textit{h. I’ll take the day off work.})
            \textbf{Type:} First Conditional \\
            \textbf{Meaning:} This is a first conditional statement, presenting a likely future situation and its potential outcome.

      \item If people didn’t drive so fast on this road … (\textit{b. we would run the course.})
            \textbf{Type:} Second Conditional \\
            \textbf{Meaning:} This is a second conditional statement, indicating a hypothetical situation in the present or future and its probable result.
\end{enumerate}

\allocation{B. Matching the clauses:}
\begin{enumerate}
      \item If she doesn’t pass the exam this year … (\textit{h. she can try again next year.})
            \textbf{Type:} First Conditional \\
            \textbf{Meaning:} This is a first conditional statement, presenting a likely future situation and its potential outcome.

      \item If I had the time … (\textit{b. I’d love to learn to play tennis.})
            \textbf{Type:} Second Conditional \\
            \textbf{Meaning:} This is a second conditional statement, indicating a hypothetical situation in the present or future and its probable result.

      \item If he hadn’t done engineering … (\textit{c. what would he have studied?})
            \textbf{Type:} Third Conditional \\
            \textbf{Meaning:} This is a third conditional statement, expressing a situation that didn't happen in the past and its imaginary result.

      \item I would never have bought this car … (\textit{a. if I’d known how much petrol it uses.})
            \textbf{Type:} Third Conditional \\
            \textbf{Meaning:} This is a third conditional statement, expressing a situation that didn't happen in the past and its imaginary result.

      \item If you don’t book now… (\textit{d. you won’t get good tickets.})
            \textbf{Type:} First Conditional \\
            \textbf{Meaning:} This is a first conditional statement, presenting a likely future situation and its potential outcome.

      \item If the rent had been lower… (\textit{e. I would have taken the flat.})
            \textbf{Type:} Third Conditional \\
            \textbf{Meaning:} This is a third conditional statement, expressing a situation that didn't happen in the past and its imaginary result.

      \item If the flight is late… (\textit{f. we’ll miss our connection.})
            \textbf{Type:} First Conditional \\
            \textbf{Meaning:} This is a first conditional statement, presenting a likely future situation and its potential outcome.

      \item If it snows… (\textit{g. we get our skis out.})
            \textbf{Type:} Zero Conditional \\
            \textbf{Meaning:} This is a zero conditional statement, stating a general truth or scientific fact.

      \item If you take another week off work … (\textit{i. the boss will definitely fire you.})
            \textbf{Type:} First Conditional \\
            \textbf{Meaning:} This is a first conditional statement, presenting a likely future situation and its potential outcome.
\end{enumerate}

\chapter{Семинар №8 11.04.24}

\section{Exercise №38}
\subsection*{Put the verbs in the correct tense. Translate the sentences into Russian.}

\allocation{Conditional 1:}
\begin{enumerate}
      \item If I \underline{\hspace{2cm}} (finish) early, I will call you.
      \item I \underline{\hspace{2cm}} (catch) the 9:00 train, if I hurry up.
      \item She will get the answer, if she \underline{\hspace{2cm}} (try) to understand.
      \item If you \underline{\hspace{2cm}} (be) free earlier, we can go for a walk.
      \item If you are hungry, I \underline{\hspace{2cm}} (make) some sandwiches.
      \item If he \underline{\hspace{2cm}} (study) hard, he’ll do well in the exam.
      \item If you \underline{\hspace{2cm}} (not be) back by 5pm, we’ll leave without you.
\end{enumerate}

\allocation{Conditional 2:}
\begin{enumerate}
      \item If I \underline{\hspace{2cm}} (be) a star, I would help those in need.
      \item He \underline{\hspace{2cm}} (buy) a house if he had a job.
      \item She \underline{\hspace{2cm}} (be) happy, if she married him.
      \item If I \underline{\hspace{2cm}} (be) you, I would ask for help.
      \item If I had more time, I \underline{\hspace{2cm}} (go) to the gym.
      \item I \underline{\hspace{2cm}} (not/have to walk) everywhere, if I bought a car.
      \item If people used bikes instead of cars, there \underline{\hspace{2cm}} (not/be) so much pollution.
\end{enumerate}

\allocation{Conditional 3:}
\begin{enumerate}
      \item If he \underline{\hspace{2cm}} (be) more careful, he would not have had that terrible accident.
      \item I \underline{\hspace{2cm}} (pass) the exam if I had worked harder.
      \item If he \underline{\hspace{2cm}} (not learn) to play the guitar, he wouldn’t have joined the band.
      \item If the government \underline{\hspace{2cm}} (spend) all the money given, all the roads \underline{\hspace{2cm}} (be paved).
      \item We wouldn't have been able to answer your questions if we \underline{\hspace{2cm}} (read/not) the book.
      \item If he had left earlier, he \underline{\hspace{2cm}} (arrive) on time.
      \item If they \underline{\hspace{2cm}} (book) earlier, they could have found better seats.
\end{enumerate}

\subsection*{Solution}
\allocation{Conditional 1:}
\begin{enumerate}
      \item \textbf{If I finish early, I will call you.} - \textit{Если я закончу рано, я позвоню тебе.}
      \item \textbf{I will catch the 9:00 train, if I hurry up.} - \textit{Я поймаю поезд в 9:00, если я потороплюсь.}
      \item \textbf{She will get the answer, if she tries to understand.} - \textit{Она получит ответ, если попытается понять.}
      \item \textbf{If you are free earlier, we can go for a walk.} - \textit{Если ты освободишься раньше, мы сможем пойти погулять.}
      \item \textbf{If you are hungry, I will make some sandwiches.} - \textit{Если ты голоден, я приготовлю немного бутербродов.}
      \item \textbf{If he studies hard, he’ll do well in the exam.} - \textit{Если он усердно учится, он хорошо справится с экзаменом.}
      \item \textbf{If you are not back by 5pm, we’ll leave without you.} - \textit{Если ты не вернёшься к 5 вечера, мы уйдем без тебя.}
\end{enumerate}

\allocation{Conditional 2:}
\begin{enumerate}
      \item \textbf{If I were a star, I would help those in need.} - \textit{Если бы я был звездой, я бы помогал нуждающимся.}
      \item \textbf{He would buy a house if he had a job.} - \textit{Он бы купил дом, если бы у него была работа.}
      \item \textbf{She would be happy if she married him.} - \textit{Она была бы счастлива, если бы вышла за него замуж.}
      \item \textbf{If I were you, I would ask for help.} - \textit{Если бы я был на твоем месте, я бы попросил помощи.}
      \item \textbf{If I had more time, I would go to the gym.} - \textit{Если бы у меня было больше времени, я бы ходил в спортзал.}
      \item \textbf{I wouldn’t have to walk everywhere if I bought a car.} - \textit{Мне бы не пришлось ходить пешком везде, если бы я купил машину.}
      \item \textbf{If people used bikes instead of cars, there wouldn’t be so much pollution.} - \textit{Если бы люди использовали велосипеды вместо машин, не было бы такого загрязнения.}
\end{enumerate}

\allocation{Conditional 3:}
\begin{enumerate}
      \item \textbf{If he had been more careful, he would not have had that terrible accident.} - \textit{Если бы он был осторожнее, у него не было бы этой ужасной аварии.}
      \item \textbf{I would have passed the exam if I had worked harder.} - \textit{Я бы сдал экзамен, если бы усерднее готовился.}
      \item \textbf{If he hadn’t learned to play the guitar, he wouldn’t have joined the band.} - \textit{Если бы он не научился играть на гитаре, он бы не вступил в группу.}
      \item \textbf{If the government had spent all the money given, all the roads would have been paved.} - \textit{Если бы правительство потратило все выделенные средства, все дороги были бы вымощены.}
      \item \textbf{We wouldn’t have been able to answer your questions if we hadn’t read the book.} - \textit{Мы бы не смогли ответить на ваши вопросы, если бы не прочли книгу.}
      \item \textbf{If he had left earlier, he would have arrived on time.} - \textit{Если бы он ушел раньше, он бы прибыл вовремя.}
      \item \textbf{If they had booked earlier, they could have found better seats.} - \textit{Если бы они забронировали билеты раньше, они могли бы найти лучшие места.}
\end{enumerate}

\section{Exercise №39}
\subsection*{Decide what contraction ’d stands for: would or had.}
\allocation{Example:}If I'd known you were in hospital, I'd have visited you. → If I had known you were in hospital, I would have visited you. I'd have bought you a present if I'd known it was your birthday. → I would have bought you a present if I had known it was your birthday.

\begin{enumerate}
      \item If you'd given me your e-mail, I'd have written to you.
      \item If you’d asked me, I’d have phoned the customers to let them know.
      \item If I was rich, I’d spend all my time travelling.
      \item You could have changed your opinion if you’d stayed longer.
      \item I’d help you if I knew how.
      \item If he’d listened to what his friends had been telling him he wouldn’t have lost so much money.
      \item They’d have got the job done more quickly if they’d had more people working on it.
      \item If I saw a snake, I’d be terrified.
\end{enumerate}

\subsection*{Solution}
\begin{enumerate}
      \item If you had given me your e-mail, I would have written to you.
      \item If you had asked me, I would have phoned the customers to let them know.
      \item Incorrect usage. It should be "If I were rich, I would spend all my time traveling."
      \item You could have changed your opinion if you had stayed longer.
      \item Incorrect usage. It should be "I would help you if I knew how."
      \item If he had listened to what his friends had been telling him, he wouldn’t have lost so much money.
      \item They would have got the job done more quickly if they had had more people working on it.
      \item Incorrect usage. It should be "If I see a snake, I am terrified."
\end{enumerate}

\section{Exercise №42}
\subsection*{Read the following examples paying attention to the synonyms of ‘if’. Explain their meaning or translate the sentences into Russian.}
\begin{enumerate}
      \item Provided that there are enough seats, anyone can go on the trip.
      \item Provided that the plane takes off on time, we should reach Irkutsk by morning.
      \item So long as a tiger stands still, it is invisible in the jungle.
      \item The bank lent the company 100,000 pounds on condition that they would repay the money within six months.
      \item You can get a senior citizen’s concession providing you’ve got an ID card.
      \item Supposing I don’t arrive till after midnight, will the hotel still be open?
      \item They may do whatever they like provided that it is legal.
      \item Supposing you lost your passport while travelling, you’d have to go to the embassy, wouldn’t you?
      \item In case I forget later, here are the keys to the garage.
      \item Let’s take our swimming costumes in case there’s a pool at the hotel.
\end{enumerate}

\subsection*{Solution}
\begin{enumerate}
      \item \textbf{Provided that there are enough seats, anyone can go on the trip.}
            \textit{Meaning:} If there are sufficient seats available, everyone is allowed to go on the trip.
            \textit{Translation:} При условии, что есть достаточно мест, любой может поехать в поездку.

      \item \textbf{Provided that the plane takes off on time, we should reach Irkutsk by morning.} \\
            \textit{Meaning:} If the plane departs punctually, we expect to arrive in Irkutsk by morning. \\
            \textit{Translation:} При условии, что самолет взлетит вовремя, мы должны добраться до Иркутска к утру.

      \item \textbf{So long as a tiger stands still, it is invisible in the jungle.} \\
            \textit{Meaning:} If a tiger remains stationary, it is difficult to spot it amidst the jungle foliage. \\
            \textit{Translation:} Пока тигр стоит неподвижно, он невидим в джунглях.

      \item \textbf{The bank lent the company 100,000 pounds on condition that they would repay the money within six months.} \\
            \textit{Meaning:} The bank provided a loan of 100,000 pounds with the stipulation that the company must repay it within six months.  \\
            \textit{Translation:} Банк выдал компании кредит на 100 000 фунтов при условии, что они вернут деньги в течение шести месяцев.

      \item \textbf{You can get a senior citizen’s concession providing you’ve got an ID card.} \\
            \textit{Meaning:} If you possess an ID card, you are eligible for a discount as a senior citizen. \\
            \textit{Translation:} Вы можете получить льготу для пенсионеров, при условии, что у вас есть удостоверение личности.

      \item \textbf{Supposing I don’t arrive till after midnight, will the hotel still be open?} \\
            \textit{Meaning:} If I arrive after midnight, is it likely that the hotel will still be open? \\
            \textit{Translation:} Предположим, что я приеду после полуночи, отель все еще будет открыт?

      \item \textbf{They may do whatever they like provided that it is legal.} \\
            \textit{Meaning:} They are permitted to engage in any activity as long as it conforms to the law. \\
            \textit{Translation:} Они могут делать что угодно, при условии, что это законно.

      \item \textbf{Supposing you lost your passport while traveling, you’d have to go to the embassy, wouldn’t you?} \\
            \textit{Meaning:} If you were to lose your passport during travel, you would need to visit the embassy, correct? \\
            \textit{Translation:} Предположим, что вы потеряли свой паспорт во время путешествия, вам придется пойти в посольство, верно?

      \item \textbf{In case I forget later, here are the keys to the garage.} \\
            \textit{Meaning:} If I happen to forget later, the keys to the garage are provided now. \\
            \textit{Translation:} На случай, если я забуду позже, вот ключи от гаража.

      \item \textbf{Let’s take our swimming costumes in case there’s a pool at the hotel.} \\
            \textit{Meaning:} We should bring our swimsuits just in case there is a pool available at the hotel. \\
            \textit{Translation:} Давай возьмем купальные костюмы на всякий случай, если в отеле будет бассейн.

\end{enumerate}

\section{Exercise 44}
\subsection*{Use your own ideas to complete the sentences. Think of your own examples with different types of Conditionals.}
\begin{enumerate}
      \item I like hot weather provided \underline{\hspace{6cm}}
      \item I’d walk to university unless \underline{\hspace{6cm}}
      \item You can borrow the money provided \underline{\hspace{6cm}}
      \item You won’t get a good job, unless \underline{\hspace{6cm}}
      \item I could go out tonight if \underline{\hspace{6cm}}
      \item If I was free now \underline{\hspace{6cm}}
      \item If I saved a large sum of money \underline{\hspace{6cm}}
      \item If I had never studied English \underline{\hspace{6cm}}
      \item If I had not come to this university \underline{\hspace{6cm}}
      \item If I had been born 60 years ago \underline{\hspace{6cm}}
      \item If there was a power cut in this building \underline{\hspace{6cm}}
      \item If I found myself alone on a desert island \underline{\hspace{6cm}}
\end{enumerate}

\subsection*{Solution}
\begin{enumerate}
      \item I like hot weather provided there's a cool breeze to balance it out.
      \item I'd walk to university unless it's pouring rain outside.
      \item You can borrow the money provided you pay it back by the end of the month.
      \item You won't get a good job unless you have relevant experience or qualifications.
      \item I could go out tonight if I finish my work early.
      \item If I was free now, I'd probably catch up on some reading.
      \item If I saved a large sum of money, I'd invest it in real estate.
      \item If I had never studied English, I wouldn't be able to communicate effectively with people from around the world.
      \item If I had not come to this university, I might have pursued a different career path.
      \item If I had been born 60 years ago, I would have experienced a vastly different world without modern technology.
      \item If there was a power cut in this building, we'd have to rely on flashlights and candles.
      \item If I found myself alone on a desert island, I'd try to build a shelter and find a source of fresh water.
\end{enumerate}

\section*{Exercise №45}
\subsection*{Paraphrase the following sentences according to the model.}
\allocation{Example:}If he calls, give him all the necessary details. → Should he call, give him all the necessary details.

\begin{enumerate}
      \item If I had known about the meeting, I would have attended.
      \item If he had been there, he could have helped them.
      \item If she had not applied early, she wouldn’t have been accepted.
      \item If you had not left an hour early, you would have been late for the meeting.
      \item If I see him, I’ll pass on the message to him.
      \item If you need anything, please call me.
      \item If he calls you, I also want to speak to him.
      \item If the test drive of an Uber car had been successful, self-driving vehicles would have developed more rapidly.
\end{enumerate}

\subsection*{Solution}
\begin{enumerate}
      \item Had I known about the meeting, I would have attended.
      \item Had he been there, he could have helped them.
      \item Had she not applied early, she wouldn’t have been accepted.
      \item Had you not left an hour early, you would have been late for the meeting.
      \item Should I see him, I’ll pass on the message to him.
      \item Should you need anything, please call me.
      \item Should he call you, I also want to speak to him.
      \item Had the test drive of an Uber car been successful, self-driving vehicles would have developed more rapidly.
\end{enumerate}

\chapter{Домашнее задание №8 18.04.24}

\section{Exercise №40}
\subsection*{Put the verbs in brackets into the correct form and explain their meaning or translate the
      sentences into Russian.}
\allocation{Example:}
\begin{itemize}
      \item If you (\textit{to park}) here, your car (\textit{to get}) towed. (Zero conditional) $\rightarrow$ If you park here, your car gets towed.
      \item If you (\textit{to catch}) the fast train, you (\textit{to get}) home early. (Type 2) $\rightarrow$ If you catch the fast train, you will get home early.
      \item If we asked him, he would help us. (Type 2) $\rightarrow$ If we asked him, he would help us.
      \item If I (\textit{to know}) you were coming, I (\textit{to buy}) a cake. (Type 3) $\rightarrow$ If I had known you were coming, I would have bought a cake.
\end{itemize}

\begin{enumerate}
      \item If materials (\textit{to be}) not so pervasive, we (\textit{to be taking}) them for granted. (Type 2)\\
            Если материалы не были бы такими всеобъемлющими, мы бы относились к ним более равнодушно.

      \item If materials science (\textit{to be developing}) so dynamically, it (\textit{to become}) a key discipline. (Type 3)\\
            Если бы наука о материалах развивалась так динамично, она стала бы ключевой дисциплиной.

      \item If we (\textit{to continue}) to develop materials with desired properties, materials science (\textit{to be}) the most important technology of the next decade. (Type 1)\\
            Если мы продолжим развивать материалы с желаемыми свойствами, наука о материалах будет самой важной технологией следующего десятилетия.

      \item If clear safety rules for self-driving cars (\textit{to be created}), they (\textit{to keep}) our roads safe. (Type 1)\\
            Если будут созданы четкие правила безопасности для автономных автомобилей, они будут сохранять безопасность на наших дорогах.

      \item The accident (\textit{not/to happen}) if you (\textit{to test}) your brakes. (Type 3)\\
            Авария не произошла бы, если бы вы проверили свои тормоза.

      \item If computers (\textit{not/to be invented}), lots of new jobs (\textit{not/to appear}). (Type 3)\\
            Если бы компьютеры не были изобретены, множество новых рабочих мест не появилось бы.

      \item When the sun (\textit{to go}) down, it (\textit{to get}) dark. (Type 0)\\
            Когда солнце заходит, становится темно.

      \item What (\textit{to happen}) if the Internet (\textit{to be invented}) 100 years ago? (Type 3)\\
            Что бы произошло, если бы интернет был изобретен 100 лет назад?

      \item If unprecedented developments in AI technology (\textit{to continue}), smart machines (\textit{to take}) over millions of our jobs in the near future. (Type 1)\\
            Если безпрецедентные достижения в технологии искусственного интеллекта продолжатся, умные машины захватят миллионы наших рабочих мест в ближайшем будущем.

      \item If he (\textit{not/to run}) a red light, the accident (\textit{to happen}). (Type 3)\\
            Если бы он не проехал на красный свет, авария не произошла бы.
\end{enumerate}

\begin{itemize}
      \item If materials were not so pervasive, we would be taking them for granted. (Type 2)
      \item If materials science had been developing so dynamically, it would have become a key discipline. (Type 3)
      \item If we continue to develop materials with desired properties, materials science will be the most important technology of the next decade. (Type 1)
      \item If clear safety rules for self-driving cars are created, they will keep our roads safe. (Type 1)
      \item The accident would not have happened if you had tested your brakes. (Type 3)
      \item If computers had not been invented, lots of new jobs would not have appeared. (Type 3)
      \item When the sun goes down, it gets dark. (Type 0)
      \item What would happen if the Internet had been invented 100 years ago? (Type 3)
      \item If unprecedented developments in AI technology continue, smart machines will take over millions of our jobs in the near future. (Type 1)
      \item If he had not run a red light, the accident would not have happened. (Type 3)
\end{itemize}

\section{Exercise №41}
\subsection*{Write Conditional sentences of type 2 or 3 using the sentences given below according to the example. Answers may vary.}
\allocation{Example:} The weather is bad. The flight is cancelled. $\rightarrow$ If the weather was not bad, the flight wouldn't be cancelled. There was no lifeboat. Sailors couldn’t keep afloat. $\rightarrow$ The sailors could keep afloat if there was a lifeboat.

\begin{enumerate}
      \item I didn’t prepare for the seminar. I couldn’t answer the teacher’s questions.
      \item The inaccurate values were used. The result was an error.
      \item There aren’t many currents in this part of the ocean. Organic material isn’t pulled down into the trenches.
      \item It is much easier to compute satellite orbits. The Earth is perfectly spherical and has no atmosphere.
      \item The research team used a free-falling autonomous camera system. Many new species of animals were documented.
      \item No satellites were launched. The transmissions of microwaves across the oceans were impossible.
      \item The Earth doesn't stay in one place in its orbit. Day and night change in length.
      \item The technology is developing fast. It is hard to keep up with technology these days.
      \item The vehicle was built with a new type of alloy. It wasn’t badly damaged in a car crash.
      \item Many ships were lost at sea. Their sailors didn’t know how to find out their location.
\end{enumerate}

\subsection*{Solution}
\begin{enumerate}
      \item If I had prepared for the seminar (\textit{Type 3}), I could have answered the teacher’s questions (\textit{Type 2}).
      \item If the accurate values had been used (\textit{Type 3}), the result wouldn't have been an error (\textit{Type 2}).
      \item If there were many currents in this part of the ocean (\textit{Type 2}), organic material would be pulled down into the trenches (\textit{Type 2}).
      \item If it weren't much easier to compute satellite orbits (\textit{Type 2}), the Earth wouldn't be perfectly spherical and have no atmosphere (\textit{Type 2}).
      \item If the research team hadn't used a free-falling autonomous camera system (\textit{Type 3}), many new species of animals wouldn't have been documented (\textit{Type 2}).
      \item If satellites had been launched (\textit{Type 3}), the transmissions of microwaves across the oceans wouldn't have been impossible (\textit{Type 2}).
      \item If the Earth stayed in one place in its orbit (\textit{Type 2}), day and night wouldn't change in length (\textit{Type 2}).
      \item If technology weren't developing fast (\textit{Type 2}), it wouldn't be hard to keep up with technology these days (\textit{Type 2}).
      \item If the vehicle hadn't been built with a new type of alloy (\textit{Type 3}), it would have been badly damaged in a car crash (\textit{Type 2}).
      \item If the sailors had known how to find out their location (\textit{Type 3}), many ships wouldn't have been lost at sea (\textit{Type 2}).
\end{enumerate}

\section{Exercise №43}
\subsection*{Change the following sentences so that each sentence contains the word ‘unless’.}
\allocation{Example:}You’ll catch a cold if you don’t wear warm clothes. → Unless you wear warm clothes,
you’ll catch a cold.
\begin{enumerate}
      \item You won’t get in if you don’t have a ticket.
      \item The match will be called off if the weather doesn’t clear up.
      \item I wouldn’t get the job if I didn’t pass my driving test.
      \item If your English doesn’t improve, you’ll fail the exam.
      \item If you don’t slow down, you will have an accident.
      \item If it doesn’t rain soon, all the plants are going to die.
      \item If you don’t ask questions, you won’t learn to think critically.
      \item If we hadn’t made a booking weeks in advance, we wouldn’t have been able to get a flight.
      \item We might need to cancel the show if we don’t sell more tickets at the last minute.
      \item If he hadn’t recognised us, he might never have spoken to us.
\end{enumerate}

\subsection*{Solution}
\begin{enumerate}
      \item Unless you have a ticket, you won’t get in.
      \item The match will be called off unless the weather clears up.
      \item Unless I pass my driving test, I wouldn’t get the job.
      \item Unless your English improves, you’ll fail the exam.
      \item Unless you slow down, you will have an accident.
      \item Unless it rains soon, all the plants are going to die.
      \item Unless you ask questions, you won’t learn to think critically.
      \item Unless we made a booking weeks in advance, we wouldn’t have been able to get a flight.
      \item Unless we sell more tickets at the last minute, we might need to cancel the show.
      \item Unless he recognized us, he might never have spoken to us.
\end{enumerate}

\section{Exercise №59}
\begin{enumerate}
      \item If we had invested in new technology,
      \item If people didn’t suffer from range anxiety,
      \item If new materials hadn’t been discovered,
      \item Unless so many different means of transport had been developed,
      \item If they had told me about the problem earlier,
      \item If I had studied English at school,
      \item If you really wanted to go there,
      \item If I had taken his advice,
      \item If they had caught an earlier train,
      \item If the Tu-144 hadn’t been withdrawn,
\end{enumerate}

\begin{itemize}
      \item[a.] people wouldn’t be travelling so extensively today.
      \item[b.] I wouldn’t be in this mess now.
      \item[c.] we would still be competitive.
      \item[d.] everything would be all right now.
      \item[e.] they would be here now.
      \item[f.] more electric cars would have been produced.
      \item[g.] I wouldn’t find it so difficult to learn now.
      \item[h.] our flights to distant parts of the country would be much shorter today.
      \item[i.] materials science wouldn’t be such an exciting and dynamic field as it is today.
      \item[j.] you would have booked the trip long ago.
\end{itemize}

\subsection*{Solution}
\begin{enumerate}
      \item If we had invested in new technology, we would still be competitive.
      \item If people didn’t suffer from range anxiety, more electric cars would have been produced.
      \item If new materials hadn’t been discovered, materials science wouldn’t be such an exciting and dynamic field as it is today.
      \item Unless so many different means of transport had been developed, our flights to distant parts of the country would be much shorter today.
      \item If they had told me about the problem earlier, I wouldn’t be in this mess now.
      \item If I had studied English at school, I wouldn’t find it so difficult to learn now.
      \item If you really wanted to go there, you would have booked the trip long ago.
      \item If I had taken his advice, everything would be all right now.
      \item If they had caught an earlier train, they would be here now.
      \item If the Tu-144 hadn’t been withdrawn, people wouldn’t be travelling so extensively today.
\end{enumerate}

\section{Check yourself}

\subsection*{1. What type of material is it?}
\begin{enumerate}
      \item \textbf{Polymer}: A chemical compound or mixture of compounds formed by polymerization and consisting essentially of repeating structural units. They include plastic and rubber materials.
      \item \textbf{Ceramic}: Compounds made up of either metallic or nonmetallic elements, such as earthenware, porcelain, or brick, that have been shaped and then hardened by heating to high temperatures.
      \item \textbf{Composite}: Composed of two or more individual materials which allow them to achieve a combination of properties that is not displayed by any single material.
      \item \textbf{Concrete}: A building material made from a mixture of broken stone or gravel, sand, cement, and water, which can be spread or poured into molds and forms a mass resembling stone on hardening.
      \item \textbf{Alloy}: A material that is made up of at least two different chemical elements, one of which is a metal, e.g., steel.
\end{enumerate}

\subsection*{Solution}
\begin{enumerate}
      \item \textbf{Polymer}: A chemical compound or mixture of compounds formed by polymerization and consisting essentially of repeating structural units. They include plastic and rubber materials.
      \item \textbf{Ceramic}: Compounds made up of either metallic or nonmetallic elements, such as earthenware, porcelain, or brick, that have been shaped and then hardened by heating to high temperatures.
      \item \textbf{Composite}: Composed of two or more individual materials which allow them to achieve a combination of properties that is not displayed by any single material.
      \item \textbf{Concrete}: A building material made from a mixture of broken stone or gravel, sand, cement, and water, which can be spread or poured into molds and forms a mass resembling stone on hardening.
      \item \textbf{Alloy}: A material that is made up of at least two different chemical elements, one of which is a metal, e.g., steel.
\end{enumerate}

\subsection*{3. True or False?}
\begin{enumerate}
      \item \textbf{Materials science} is a discipline that studies the properties of matter and its applications.
      \item This science \textbf{does not} study the relationships between the structure of materials at atomic or molecular scales.
      \item \textbf{Materials science} is studied at many universities and has become part of forensic engineering or failure analysis.
      \item \textbf{Materials science} is \textbf{not} the most important technology today.
      \item \textbf{Smart materials} are materials that are used for manufacturing smart devices.
      \item \textbf{Smart materials} can be also called \textit{shape memory materials} because they react to changes in their environment.
      \item \textbf{Materials science} can be divided into different disciplines that study different materials and their properties: \textit{metallurgy, biomaterials}, etc.
      \item Breakthroughs in \textbf{materials science} are likely to affect the future of technology significantly.
      \item The examples of revolutionary materials include \textbf{graphene, composites, and wood}.
      \item \textbf{Composites} exist in nature, for example, a piece of wood is a composite.
\end{enumerate}

\subsection*{Solution}
\begin{itemize}
      \item True
      \item False
      \item True
      \item False
      \item True
      \item True
      \item True
      \item True
      \item True
      \item True
\end{itemize}

\subsection*{4. Choose ten materials and find products made of these materials. Complete the table.}
Compare your examples.

\subsection*{Solution}
\begin{table}[htbp]
      \centering
      \begin{tabular}{|c|c|p{6cm}|}
            \hline
            \textbf{Material} & \textbf{Product} & \textbf{Use}                                                                          \\
            \hline
            Wood              & Furniture        & Used for sitting, sleeping, storage, and decoration purposes in homes and offices.    \\
            Steel             & Automobiles      & Used for constructing car bodies, chassis, and various components for transportation. \\
            Glass             & Windows          & Provides visibility while insulating against weather and allowing natural light in.   \\
            Plastic           & Bottles          & Used for containing and transporting liquids such as water, beverages, and chemicals. \\
            Ceramic           & Tiles            & Used for covering floors, walls, and other surfaces in buildings for decoration.      \\
            Leather           & Shoes            & Worn for protection and comfort, primarily on feet, and as a fashion accessory.       \\
            Fabric            & Clothing         & Worn for protection, modesty, and fashion, including shirts, pants, dresses, etc.     \\
            Aluminum          & Cans             & Used for packaging beverages such as sodas, beers, and canned foods for preservation. \\
            Rubber            & Tires            & Used for providing traction, support, and cushioning for vehicles on roads.           \\
            Concrete          & Buildings        & Used as a primary construction material for building foundations, walls, and floors.  \\
            \hline
      \end{tabular} 
\end{table}

\newpage

\subsection*{7. Complete sentence b in each pair so that it has a similar meaning to sentence a.}
\allocation{Example:} Bus lanes were introduced. Travelling by public transport was made easier. → Unless
bus lanes had been introduced, travelling by public transport wouldn’t have been made easier.

\begin{enumerate}
      \item a) It’s likely that there is life on other planets. If so, we are not alone. \\
            b) If \underline{\hspace{3cm}} life on other planets, we \underline{\hspace{3cm}} not be alone.

      \item a) The world’s population will probably continue to increase. If so, we will need more food. \\
            b) If the world’s population \underline{\hspace{3cm}} to increase, we \underline{\hspace{3cm}} more food.

      \item a) Other intelligent beings might inhabit the universe. If so, they would be very different from us. \\
            b) If other intelligent beings \underline{\hspace{3cm}} the universe, they \underline{\hspace{3cm}} very different from us.

      \item a) There aren’t many TV programmes about science, some people don’t know much about it. \\
            b) If there \underline{\hspace{3cm}} more TV programmes about science, people \underline{\hspace{3cm}} more about it.

      \item a) We shouldn’t have spent so much money on space research. Instead, we could have solved many other serious problems. \\
            b) If we \underline{\hspace{3cm}} less on space research, we could have solved many other serious problems.

      \item a) Robotic vehicles have been used in dangerous environments for decades. The idea to create self-driving cars appeared. \\
            b) Unless robotic vehicles \underline{\hspace{3cm}} the idea to create self-driving cars \underline{\hspace{3cm}}.

      \item a) A few accidents involving self-driving cars have happened. People decided that they were not safe. \\
            b) Unless a few accidents involving self-driving cars \underline{\hspace{3cm}}, people \underline{\hspace{3cm}} that they were not safe.

      \item a) Neural networks take inspiration from the human brain. AI software is quite good at learning about scenarios it has never faced. \\
            b) If neural networks \underline{\hspace{3cm}} inspiration from the human brain, AI software \underline{\hspace{3cm}} quite good at learning about scenarios it has never faced.

\end{enumerate}

\section*{Solution}
\begin{enumerate}
      \item If there were life on other planets, we would not be alone.
      \item If the world’s population continues to increase, we will need more food.
      \item If other intelligent beings inhabit the universe, they would be very different from us.
      \item If there were more TV programmes about science, people would know more about it.
      \item If we had spent less on space research, we could have solved many other serious problems.
      \item Unless robotic vehicles had been used in dangerous environments for decades, the idea to create self-driving cars would not have appeared.
      \item Unless a few accidents involving self-driving cars had happened, people would not have decided that they were not safe.
      \item If neural networks take inspiration from the human brain, AI software is quite good at learning about scenarios it has never faced.
\end{enumerate}

\section{Progress test}

\subsection*{Vocabulary. Decide which answer a, b or c best fits into each gap.}
Today we are taking \underline{\hspace{2cm}} many thousands of manufactured objects that \underline{\hspace{2cm}} our comfort in everyday life: the vehicles that we travel in; the clothes that we wear; the machines in our homes and offices; the sport and gym equipment we use; the computers and phones that we can’t live without; and more importantly, the medical technology that keeps us alive. Everything we see and use is made from materials \underline{\hspace{2cm}} from the Earth or created by people. These materials can be split into four main groups: metals, polymers, ceramics, and \underline{\hspace{2cm}} . The technological advances that have transformed our world over the last 20 years have been founded on the developments in Materials Science and Engineering. Materials are \underline{\hspace{2cm}} faster today than at any time in history, enabling engineers to improve the \underline{\hspace{2cm}} of existing products and to develop innovative technologies that will \underline{\hspace{2cm}} all the aspects of our lives. Materials Science and Engineering has become a key discipline in the \underline{\hspace{2cm}} global economy and is recognised as one of the technical disciplines. Due to the achievements in materials science we are \underline{\hspace{2cm}} to develop new products and technologies that will make our lives safer, more convenient, more enjoyable and that will allow us to \underline{\hspace{2cm}} the challenges of the future.

\begin{enumerate}
      \item[a.] will…push
      \item[b.] would… push
      \item[c.] had… pushed
\end{enumerate}

\begin{enumerate}
      \item[a.] part in
      \item[b.] advantage
      \item[c.] for granted
\end{enumerate}

\begin{enumerate}
      \item[a.] comprise
      \item[b.] contribute to
      \item[c.] cover
\end{enumerate}

\begin{enumerate}
      \item[a.] derived
      \item[b.] shaped
      \item[c.] split
\end{enumerate}

\begin{enumerate}
      \item[a.] concrete
      \item[b.] composites
      \item[c.] superconductors
\end{enumerate}

\begin{enumerate}
      \item[a.] evolving
      \item[b.] occurring
      \item[c.] involving
\end{enumerate}

\begin{enumerate}
      \item[a.] acceleration
      \item[b.] role
      \item[c.] performance
\end{enumerate}

\begin{enumerate}
      \item[a.] cover
      \item[b.] acquire
      \item[c.] join
\end{enumerate}

\begin{enumerate}
      \item[a.] general
      \item[b.] competitive
      \item[c.] comparable
\end{enumerate}

\begin{enumerate}
      \item[a.] current
      \item[b.] certain
      \item[c.] unlike
\end{enumerate}

\begin{enumerate}
      \item[a.] solve
      \item[b.] avoid
      \item[c.] meet
\end{enumerate}

\subsection*{Solution}
\begin{enumerate}
      \item[c.] for granted
      \item[b.] contribute to
      \item[a.] derived
      \item[b.] composites
      \item[a.] evolving
      \item[c.] performance
      \item[c.] join
      \item[b.] competitive
      \item[a.] current
      \item[c.] meet
\end{enumerate}

\subsection*{Grammar. Decide which answer a, b or c best fits into each gap.}
\begin{enumerate}
      \item If a material that becomes superconducting at much higher temperatures was found that \underline{\hspace{2cm}} almost certainly \underline{\hspace{2cm}} things along.
            \begin{enumerate}
                  \item[a.] will…push
                  \item[b.] would… push
                  \item[c.] had… pushed
            \end{enumerate}

      \item If mercury wire \underline{\hspace{2cm}} to absolute zero, it loses resistivity.
            \begin{enumerate}
                  \item[a.] was cooled
                  \item[b.] is cooled
                  \item[c.] will be cooled
            \end{enumerate}

      \item If we apply a strong magnetic field to a superconducting material, it \underline{\hspace{2cm}} to the normal state.
            \begin{enumerate}
                  \item[a.] returned
                  \item[b.] will be returned
                  \item[c.] would be returned
            \end{enumerate}

      \item If superconductivity occurred at room temperatures, we \underline{\hspace{2cm}} losses in transporting energy.
            \begin{enumerate}
                  \item[a.] could reduce
                  \item[b.] could have reduced
                  \item[c.] can reduce
            \end{enumerate}

      \item Unless the theory of superconductivity \underline{\hspace{2cm}}, we wouldn’t have been able to understand the behavior of superconducting materials.
            \begin{enumerate}
                  \item[a.] hadn’t been created
                  \item[b.] was created
                  \item[c.] had been created
            \end{enumerate}

      \item Unless lasers \underline{\hspace{2cm}} nobody would have believed that science fiction could become science fact.
            \begin{enumerate}
                  \item[a.] had been predicted
                  \item[b.] hadn’t been predicted
                  \item[c.] were predicted
            \end{enumerate}

      \item If new superconducting materials \underline{\hspace{2cm}}, superconductivity wouldn’t have become so important.
            \begin{enumerate}
                  \item[a.] had been discovered
                  \item[b.] hadn’t been discovered
                  \item[c.] weren’t discovered
            \end{enumerate}

      \item If driverless cars became commonplace, it \underline{\hspace{2cm}} fundamentally \underline{\hspace{2cm}} car use and traffic accidents would be prevented.
            \begin{enumerate}
                  \item[a.] will change
                  \item[b.] can change
                  \item[c.] would change
            \end{enumerate}

      \item Provided robots \underline{\hspace{2cm}} emotions, they could become our perfect companions.
            \begin{enumerate}
                  \item[a.] ‘d understood and felt
                  \item[b.] would understand and feel
                  \item[c.] understood and felt
            \end{enumerate}

      \item If Ford had not created an affordable car, they \underline{\hspace{2cm}} mass produced.
            \begin{enumerate}
                  \item[a.] wouldn’t be
                  \item[b.] wouldn’t have been
                  \item[c.] would have been
            \end{enumerate}
\end{enumerate}

\begin{enumerate}
      \item \textbf{b.} would... push (\textit{Second conditional}) \\
            Type: Second conditional

      \item \textbf{a.} was cooled (\textit{Simple past for hypothetical situations}) \\
            Type: Second conditional

      \item \textbf{c.} would be returned (\textit{Future in the past}) \\
            Type: Second conditional

      \item \textbf{a.} could reduce (\textit{Conditional mood for possibility}) \\
            Type: First conditional

      \item \textbf{c.} had been created (\textit{Past perfect for unreal past condition}) \\
            Type: Third conditional
\end{enumerate}