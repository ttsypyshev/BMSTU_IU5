\part{Модуль №1}

\chapter{Семинар №1 08.02.24}

\section{Text 10A}

\subsection*{Light Beam at the Service of Humanity}
\allocation{(1)} Lasers often remind us of science fiction films and novels. Long ago
science fiction writers predicted the appearance of a mysterious fiery sword, which
would become an invincible weapon. The idea of using lasers as death rays has also
been employed by creators of such blockbusters as X-Men and Star Wars. And though the
ray laser gun still remains science fiction, putting a light beam at the service of
humanity is embodied in myriads5 of other uses based on laser technology.

\allocation{(2)} The word "laser" stands for “light amplification by stimulated
emission of radiation”. A laser, an optical device that strengthens light waves and
generates very intense beams of light, represents a powerful light source. The
difference between ordinary light and laser light is like the difference between the
ripples in your bathtub and huge waves on the sea. Until the invention of the laser,
the available light sources were generally neither monochromatic nor coherent and were
of relatively low intensity. The laser produces a well-directed, very intense beam
which is monochromatic, directional and coherent. Monochromatic means that all of the
light produced by the laser is of a single wavelength. Directional means that the beam
of light has a very low divergence. Light from conventional sources, such as a light
bulb or the sun, diverges, spreading in all directions. The intensity may be large at
the source, but it decreases rapidly as the observer moves away from the source. In
contrast, the laser output has a very small divergence and can maintain high beam
intensities over long ranges. Thus, relatively low power lasers are able to project
more energy at a single wavelength within a narrow beam than can be obtained from
much more powerful conventional light sources. Coherent means that the waves of
light are in phase with each other. A light bulb produces many wavelengths, that is
why its light is incoherent.

\allocation{(3)} The first discoveries that eventually brought us lasers were made at
the dawn of the 20th century. In 1917, Einstein laid the foundation for the laser when
he introduced the concept of stimulated emission. In 1954, Russian physicists
Nikolay Basov and Alexander Prokhorov working on the quantum oscillator created
the first microwave generator, laser’s predecessor, and described the theory of
its operation. At the same time, the idea how to generate stimulated emission at
microwave frequencies was also developed independently by American physicist
Charles Townes. He showed how this device, which was named a maser, could work.
A decade later, in 1964, all three were awarded with the Nobel Prize in physics for
their discoveries. In 1960, physicist from California Theodore Maiman demonstrated
the first ruby laser, which was considered the first successful light laser.
Other types of laser quickly followed: a gas laser and a semiconductor injection13
laser.

\allocation{(4)} Due to their remarkable properties lasers turned out to have all sorts
of useful applications in different fields from communications to medicine. In
science they are a great help in spectroscopy. They allow gigabytes of information
to be recorded. They can be used to focus relatively low wattage power to such high
intensity that it can be used to cut, heat or vaporise material. They have numerous
applications aboard spacecraft. Laser beams allow us to measure distances with
much greater accuracy than ever before. Laser-sighting devices are fitted to
military and police rifles to help soldiers hit their targets. Lasers can be used
as a defence against nuclear missiles and they may also be of use in thermonuclear
fusion reactors. Medicine and surgery have been transformed thanks to highly
accurate laser scalpels and laser diagnostics. In the arts, lasers can provide
fantastic displays of light.

\allocation{(5)} We are currently living in an era of intense development of lasers.
New types of lasers(chemical, excimer, semiconductor, free electron) are introduced
almost every year. New applications of lasers are constantly emerging. For example,
not long ago archaeologists uncovered a new vast network of cities and roads in the
thick jungles around the ancient Cambodian temple complex of Angkor Wat, implementing
an aerial survey using Lidar (light detection and ranging). Lidar might also prove
crucial in helping autonomous vehicles navigate. Lasers could have a huge impact
on the world of computing. For example, a silicon laser computer chip promises
faster data transfers. Laser developers say it could enable us to see people behind
walls, detect underground infrastructure without digging holes, and develop
navigation systems that do not rely on GPS.

\section{Exercise №1}
\subsection*{Match the words (1-6) with their definitions (a-f).
      Use a dictionary if necessary.}
\begin{enumerate}
      \item stimulated
      \item radiation
      \item acronym
      \item emission
      \item beam
      \item amplification \\
\end{enumerate}

\begin{enumerate}
      \item[a.] energy in the form of heat or light that you cannot see and
            which can be very harmful
      \item[b.] a word formed from the initial letters of other words
      \item[c.] the increase in volume of a signal
      \item[d.] a line of radiation or particles flowing in one direction
      \item[f.] the act of sending out gases or other substances
      \item[e.] made stronger or more active
\end{enumerate}

\subsection*{Solution}
\begin{enumerate}
      \item f
      \item a
      \item b
      \item e
      \item d
      \item c
\end{enumerate}

\section{Exercise №2}
\subsection*{In groups answer the questions.}
\begin{enumerate}
      \item What is a laser?
            \begin{enumerate}
                  \item a device which produces a very narrow beam of light useful in
                        many technologies
                  \item a process of optical amplification of light based on
                        radiation emission
                  \item both a and b
            \end{enumerate}
      \item What kind of word is the word ‘laser’?
            \begin{enumerate}
                  \item acronym
                  \item shortening
                  \item contraction
            \end{enumerate}
      \item Can you decode the word ‘laser’? (use the words from task 1)
            \begin{enumerate}
                  \item[] L... A... by Stimulated E... of R... .
            \end{enumerate}
\end{enumerate}

\subsection*{Solution}
\begin{enumerate}
      \item a
      \item a
      \item Light Amplification by Stimulated Emission of Radiation
\end{enumerate}

\section{Exercise №3}
\subsection*{Study the pictures below. Which of the following words and phrases
      refer to ordinary light (1) and which to laser light (2)?}
Coherent; its intensity decreases with distance; highly monochromatic;
it is not strictly monochromatic; organized; less intense; travels in one direction;
incoherent; highly intense; concentrated; travels in all directions; disorganized.

\subsection*{Solution}
\allocation{Ordinary light:}
\begin{itemize}
      \item disorganized
      \item its intensity decreases with distance
      \item it is not strictly monochromatic
      \item less intense, incoherent
      \item travels in all directions
\end{itemize}

\allocation{Laser light:}
\begin{itemize}
      \item organized
      \item coherent
      \item highly monochromatic
      \item travels in one direction
      \item highly intense
      \item concentrated
\end{itemize}

\section{Exercise №6}
\allocation{[Устно]}

\subsection*{Read the text again and answer the following questions.}
\begin{enumerate}
      \item Why can we say that lasers were predicted long before their invention?
      \item What is a laser? What does the word ‘laser’ mean?
      \item What kind of beam do lasers have?
      \item What do we mean by the words ‘monochromatic, directional, and coherent’
            when we refer to laser light?
      \item Why is the light from the laser so concentrated?
      \item Who proposed the theoretical possibility of the process that made lasers
            possible?
      \item Who created the first microwave generator?
      \item Who demonstrated the first successful light laser?
      \item What laser types are mentioned in the text?
      \item Do you agree with the author’s opinion that lasers have found myriads
            of useful applications? What examples do you think best prove this point?
      \item While reading this text, which uses of lasers surprised you the most?
      \item Can you think of an example of a laser device or technology that you
            have used or are using?
\end{enumerate}

\section{Exercise №7}
\allocation{[Устно]}

\subsection*{Read the statements and decide which of them are true (T) and
      which are false (F) according to text 10A. Explain why.}
\begin{enumerate}
      \item The word ‘laser’ means microwave amplification by stimulated emission
            of radiation.
      \item Laser was invented at the dawn of the 20th century.
      \item Albert Einstein was the first inventor of a laser.
      \item Laser came into existence only in the second half of the 20-th century.
      \item Unfortunately most of the applications of a laser proved to be unattainable
            in the real world.
      \item The use of lasers in thermonuclear fusion reactors may be the key
            to the future.
      \item Laser weapons are widely used by the military.
      \item In medicine lasers can be used for various surgical procedures.
      \item Very few inventions can match the impact of the laser’s invention.
      \item Laser technology has a promising future.
\end{enumerate}

\chapter{Домашнее задание №1 15.02.24}

\section{Exercise №5}
\subsection*{Find the words and phrases in the text which have the following meanings.}
\begin{enumerate}
      \item[§] \allocation{1}
            \begin{enumerate}
                  \item[1.] a verb: to make someone remember something
                  \item[2.] a verb: to use a particular idea or method
                  \item[3.] a verb: to continue to be in the same state or condition
                  \item[4.] a verb: to express clearly or show the importance of an
                        idea or principle
            \end{enumerate}
      \item[§] \allocation{2}
            \begin{enumerate}
                  \item[5.] a noun: the product of making larger or greater in amount
                        or intensity
                  \item[6.] a noun: the result of sending something out (e.g. gas or
                        heat)
                  \item[7.] a verb: to make stronger
                  \item[8.] a noun: the point from which something begins
                  \item[9.] an adverb: in relation to something else
                  \item[10.] a noun: a shining line of light
                  \item[11.] an -ing form of a verb: covering a large area
                  \item[12.] a verb: to go down to a lower level
                  \item[13.] a phrase used when you are comparing objects or situations
                        and saying that they are completely different
                  \item[14.] the amount of something (energy, work, information)
                        produced by a machine
            \end{enumerate}
      \item[§] \allocation{3}
            \begin{enumerate}
                  \item[15.] an adverb: after a long time
                  \item[16.] a verb phrase: to provide something (idea, principle) from
                        which another thing can develop
                  \item[17.] a verb: to give someone a prize for something they have
                        done
            \end{enumerate}
      \item[§] \allocation{4, 5}
            \begin{enumerate}
                  \item[18.] a prepositional phrase: because of or thanks to
                  \item[19.] an adjective: unusual or surprising and therefore
                        deserving attention
                  \item[20.] a verb: to have a particular result, especially one that
                        you didn’t expect
                  \item[21.] a verb: to write something (e.g. information) down
                  \item[22.] a verb: to change into a vapour
                  \item[23.] a verb: to find the size, length or amount of something
                  \item[24.] a noun: the quality of being correct and true
                  \item[25.] a verb: to carry out
                  \item[26.] a verb phrase: to be of primary importance
            \end{enumerate}
\end{enumerate}

\subsection*{Solution}
\begin{enumerate}
      \item remind
      \item employ
      \item remain
      \item embody
      \item amplification
      \item emission
      \item strengthen
      \item source
      \item ?
      \item laser
      \item spreading
      \item decrease
      \item the difference between
      \item intensity
      \item eventually
      \item -
      \item award
      \item due to
      \item remarkable
      \item -
      \item record
      \item -
      \item measure
      \item accuracy
      \item -
\end{enumerate}

\section{Exercise №8}
\subsection*{Complete the sentences using the information from the text without
      looking into the text.}
\begin{enumerate}
      \item The word laser is an acronym standing for \underline{\hspace{2cm}}.
      \item Laser light differs from ordinary light due to its \underline{\hspace{2cm}}.
      \item Russian physicists Nikolay Basov and Alexander Prokhorov created
            \underline{\hspace{2cm}} while working on \underline{\hspace{2cm}}.
      \item In 1960, physicist from California Theodore Maiman demonstrated
            \underline{\hspace{2cm}}.
      \item Lasers turned out to have myriads of uses, from \underline{\hspace{2cm}}
            to \underline{\hspace{2cm}}.
      \item In science lasers provide great assistance with \underline{\hspace{2cm}}.
      \item Laser-sighting devices are fitted to \underline{\hspace{2cm}} to help
            soldiers \underline{\hspace{2cm}}.
      \item Today new applications of lasers are \underline{\hspace{2cm}}.
      \item Not long ago archaeologists uncovered \underline{\hspace{2cm}} using Lidar.
      \item In computing lasers could have \underline{\hspace{2cm}}.
\end{enumerate}

\subsection*{Solution}
\begin{enumerate}
      \item Laser stands for "light amplification by stimulated emission of radiation." \\
            "The word 'laser' stands for 'light amplification by stimulated emission
            of radiation'." \allocation{(Paragraph 2)}
      \item Laser light is different from ordinary light because it's monochromatic,
            directional, and coherent. \\
            "The laser produces a well-directed, very intense beam which is
            monochromatic, directional and coherent." \allocation{(Paragraph 2)}
      \item Basov and Prokhorov created the precursor to the laser while studying the
            quantum oscillator. \\
            "In 1954, Russian physicists Nikolay Basov and Alexander Prokhorov working
            on the quantum oscillator created the first microwave generator, laser’s
            predecessor." \allocation{(Paragraph 3)}
      \item In 1960, Maiman demonstrated the first ruby laser. \\
            "In 1960, physicist from California Theodore Maiman demonstrated the first
            ruby laser." \allocation{(Paragraph 3)}
      \item Lasers have diverse applications, from medicine to communications. \\
            "Due to their remarkable properties, lasers turned out to have all sorts
            of useful applications in different fields from communications to medicine."
            \allocation{(Paragraph 4)}
      \item Lasers assist greatly in scientific spectroscopy. \\
            "In science they are a great help in spectroscopy."
            \allocation{(Paragraph 4)}
      \item Laser-sighting aid devices soldiers in hitting targets. \\
            "Laser-sighting devices are fitted to military and police rifles to help
            soldiers hit their targets." \allocation{(Paragraph 4)}
      \item New laser applications are continually emerging. \\
            "New applications of lasers are constantly emerging."
            \allocation{(Paragraph 5)}
      \item Archaeologists found ancient structures using Lidar. \\
            "Not long ago archaeologists uncovered a new vast network of cities and
            roads in the thick jungles around the ancient Cambodian temple complex
            of Angkor Wat, implementing an aerial survey using Lidar."
            \allocation{(Paragraph 5)}
      \item Lasers in computing could greatly improve data transfer speeds. \\
            "For example, a silicon laser computer chip promises faster data
            transfers." \allocation{(Paragraph 5)}
\end{enumerate}

\chapter{Семинар №2 15.02.24}

\subsection*{Vocabulary}
\allocation{Text 10 C}

confirm (v)
consumer (n)
controversial (n)
cure (v) diseases
far-reaching (adj)
invisible (adj)
lack (v, n)
make (v) sense (n)
particle (n)
photonics (n)
underpin (v)

\allocation{Выпилнять -}
\begin{enumerate}
      \item carry out
      \item implement
      \item do
      \item make
      \item created
      \item turn out
      \item produce/manufacture
      \item run
      \item execute
\end{enumerate}

\section{Exercise №21}
\subsection*{Fill in the gaps with the words from Exercise 20 in the right form. The
      first letters are given. Translate the sentences into Russian.}
\allocation{Example:}A d\underline{\hspace{2cm}} microphone is the one that picks up
sound from a specific area. → A directional microphone is the one that picks up sound
from a specific area.

1. All our laboratories are f\underline{\hspace{2cm}} with computers and high-speed
internet access. 2. Some people think that electromagnetic r\underline{\hspace{2cm}}
from our mobiles is harmful. 3.Climatologists say that the e\underline{\hspace{2cm}}
of greenhouse gases contributes to global warming. 4. Melatonin, a hormone involved in
controlling our sleep, is s\underline{\hspace{2cm}} by darkness. 5. The sky cleared up
and a b\underline{\hspace{2cm}} of sunlight shone in through the window. 6. If we don’t
modernise, the o\underline{\hspace{2cm}} from the factory will decrease. 7. Today it is
r\underline{\hspace{2cm}} easy to find any information thanks to the Internet. 8. The
20th century was r\underline{\hspace{2cm}} for its inventions. 9. The Nobel Prizes are
a\underline{\hspace{2cm}} annually from a fund created for that purpose by the Swedish
inventor and industrialist Alfred Bernhard Nobel. 10. A school’s success can be
m\underline{\hspace{2cm}} in terms of the number of pupils who got into university.
11. Scientists need to be very careful about the a\underline{\hspace{2cm}} of their
research results. 12. Reforms should be i\underline{\hspace{2cm}} that will allow the
company to stay competitive. 13. Our students’ ideas are e\underline{\hspace{2cm}} in
new classroom rules. 14. Exercising regularly is the best way to s\underline{\hspace{2cm}}
your immune system. 15. D\underline{\hspace{2cm}} to the large volume of letters he is
unable to answer personally. 16. Sometimes things don't t\underline{\hspace{2cm}} out
the way we think they're going to.

\subsection*{Solution}
\begin{enumerate}
      \item All our laboratories are \allocation{fitted} with computers and high-speed
            internet access.
      \item Some people think that electromagnetic \allocation{radiation} from our
            mobiles is harmful.
      \item Climatologists say that the \allocation{emission} of greenhouse gases
            contributes to global warming.
      \item Melatonin, a hormone involved in controlling our sleep, is
            \allocation{stimulated} by darkness.
      \item The sky cleared up and a \allocation{beam} of sunlight shone in through
            the window.
      \item If we don’t modernise, the \allocation{output} from the factory will
            decrease.
      \item Today it is \allocation{relatively} easy to find any information thanks
            to the Internet.
      \item The 20th century was \allocation{remarkable} for its inventions.
      \item The Nobel Prizes are \allocation{awarded} annually from a fund created
            for that purpose by the Swedish inventor and industrialist Alfred Bernhard
            Nobel.
      \item A school’s success can be \allocation{measured} in terms of the number
            of pupils who got into university.
      \item Scientists need to be very careful about the \allocation{accuracy} of
            their research results.
      \item Reforms should be \allocation{implemented} that will allow the company
            to stay competitive.
      \item Our students’ ideas are \allocation{embodied} in new classroom rules.
      \item Exercising regularly is the best way to \allocation{strengthen} your
            immune system.
      \item \allocation{Due} to the large volume of letters he is unable to answer
            personally.
      \item Sometimes things don't \allocation{turn} out the way we think they're
            going to.
\end{enumerate}

\section{Exercise №22}
\subsection*{Guess the word by its definition. Use text 10A word list to help you.}
\begin{enumerate}
      \item If two or more waves have the same phase we call this light
            c\underline{\hspace{2cm}}.
      \item When a liquid changes into gas we can say that it v\underline{\hspace{2cm}}.
      \item M\underline{\hspace{2cm}} colour refers to a colour scheme that is
            comprised of variations of one colour.
      \item If one thing is in c\underline{\hspace{2cm}} to another, it is very
            different from it.
      \item If something e\underline{\hspace{2cm}} heat, light or gas, it produces it
            and sends out by means of a physical or chemical process.
      \item If someone r\underline{\hspace{2cm}} you of a fact or event that you
            already know about, they say something which makes you think about it.
      \item If someone or something r\underline{\hspace{2cm}} in a particular state
            or condition, they stay in that state or condition and do not change.
      \item You use the conjunction n\underline{\hspace{2cm}} n\_\_ when you are
            talking about two or more things that are not true or that do not happen.
      \item Laser light is very d\underline{\hspace{2cm}} which means that it is
            extremely narrow and is emitted in one direction.
      \item A l\underline{\hspace{2cm}} is a device that emits light through a process
            of optical amplification based on the stimulated emission of
            electromagnetic radiation
\end{enumerate}

\subsection*{Solution}
\begin{enumerate}
      \item If two or more waves have the same phase we call this light
            \allocation{coherent}.
      \item When a liquid changes into gas we can say that it \allocation{vaporizes}.
      \item \allocation{Monochromatic} colour refers to a colour scheme that is
            comprised of variations of one colour.
      \item If one thing is in \allocation{contrast} to another, it is very different
            from it.
      \item If something \allocation{emits} heat, light, or gas, it produces it and
            sends out by means of a physical or chemical process.
      \item If someone \allocation{reminds} you of a fact or event that you already
            know about, they say something which makes you think about it.
      \item If someone or something \allocation{remains} in a particular state or
            condition, they stay in that state or condition and do not change.
      \item You use the conjunction \allocation{neither nor} when you are talking about
            two or more things that are not true or that do not happen.
      \item Laser light is very \allocation{directional} which means that it is
            extremely narrow and is emitted in one direction.
      \item A \allocation{laser} is a device that emits light through a process of
            optical amplification based on the stimulated emission of electromagnetic
            radiation.
\end{enumerate}

\section{Exercise №23}
\subsection*{Match the words with numbers (1-10) with the words with letters (a-j) to
      make up word collocations. Explain the meaning of these expressions and try to
      recall how they were used in text 10A.}
\allocation{Example:}to lay + the foundation for something means ‘to provide conditions
that will make something possible’, e.g. Einstein laid the foundation for the laser.

\begin{enumerate}
      \item to lay
      \item to prove
      \item to measure
      \item light
      \item stimulated
      \item to decrease
      \item conventional
      \item to spread
      \item remarkable
      \item to vaporise \\
\end{enumerate}

\begin{enumerate}
      \item[a.] crucial
      \item[b.] amplification
      \item[c.] emission
      \item[d.] source
      \item[e.] properties
      \item[f.] the foundation
      \item[g.] distances
      \item[h.] material
      \item[i.] rapidly
      \item[j.] in all directions
\end{enumerate}

\subsection*{Solution}
\begin{enumerate}
      \item f
      \item a
      \item g
      \item c
      \item b
      \item i
      \item h
      \item j
      \item e
      \item d
\end{enumerate}

\section{Exercise №24}
\subsection*{Complete each sentence with the correct word to make up a word collocation
      from Exercise 23.}
1. Buying the works of his contemporary artists, Pavel Tretiakov laid the
\underline{\hspace{2cm}} for one of the world’s greatest collections of Russian
paintings. 2. Learning the facts about how COVID-19 emerged may \underline{\hspace{2cm}}
crucial for preventing future outbreaks. 3.Before electricity was invented the
\underline{\hspace{2cm}} sources of light were candles or oil lamps. 4. The use of
lasers to \underline{\hspace{2cm}} distances is based on the principle of reflection of
a laser beam. 5. One of the problems the inventors of a laser faced was how to create
conditions for light \underline{\hspace{2cm}}. 6. Stimulated \underline{\hspace{2cm}}
of radiation is the first and necessary condition for laser light generation, but it is
not the only one. 7. Marketers know that the value of data \underline{\hspace{2cm}}
rapidly over time. 8. The fire was spreading out in all \underline{\hspace{2cm}} because
of the hot weather and strong wind. 9.The number of articles about new materials with
some remarkable \underline{\hspace{2cm}} has increased in the last years. 10. Processing
materials with a laser beam allows engineers to cut, drill, weld, and even
\underline{\hspace{2cm}} different materials.

\subsection*{Solution}
\begin{enumerate}
      \item Buying the works of his contemporary artists, Pavel Tretiakov laid the
            \allocation{foundation} for one of the world’s greatest collections of
            Russian paintings.
      \item Learning the facts about how COVID-19 emerged may \allocation{prove} crucial
            for preventing future outbreaks.
      \item Before electricity was invented the \allocation{primary} sources of light
            were candles or oil lamps.
      \item The use of lasers to \allocation{measure} distances is based on the
            principle of reflection of a laser beam.
      \item One of the problems the inventors of a laser faced was how to create
            conditions for light \allocation{amplification}.
      \item Stimulated \allocation{emission} of radiation is the first and necessary
            condition for laser light generation, but it is not the only one.
      \item Marketers know that the value of data \allocation{decreases} rapidly over
            time.
      \item The fire was spreading out in all \allocation{directions} because of the
            hot weather and strong wind.
      \item The number of articles about new materials with some remarkable
            \allocation{properties} has increased in the last years.
      \item Processing materials with a laser beam allows engineers to cut, drill,
            weld, and even \allocation{vaporise} different materials.
\end{enumerate}

\section{Exercise №25}
\subsection*{Match the words with the correct definition or synonym of each word as it
      is used in text 10B.}

\begin{enumerate}
      \item photon
      \item a partial (mirror)
      \item back and forth
      \item power source
      \item to emit
      \item to reflect
      \item to absorb
      \item to bounce
      \item concentrated
      \item hence
      \item to inject \\
\end{enumerate}

\begin{enumerate}
      \item[a.] to introduce (e.g. a fluid) into something forcefully
      \item[b.] a unit of energy that carries light and has zero mass
      \item[c.] the device that supplies energy
      \item[d.] to return or throw back (e.g. light or sound)
      \item[e.] so, thus
      \item[f.] to move away from a surface
      \item[g.] not complete, limited
      \item[h.] to send out (e.g. light or gas)
      \item[i.] to take a liquid in
      \item[j.] focused
      \item[k.] moving first in one direction and then in the opposite one
\end{enumerate}

\subsection*{Solution}
\begin{enumerate}
      \item b
      \item g
      \item k
      \item c
      \item h
      \item d
      \item i
      \item f
      \item j
      \item e
      \item a
\end{enumerate}

\section{Exercise №27}
\subsection*{Find the opposites. Match the words in column A with their opposites in
      column B.}
\allocation{Example:}\allocation{to evolve} is the opposite of \allocation{to decrease,
      worsen}.

\allocation{ A. }
\begin{enumerate}
      \item to increase
      \item to absorb
      \item stimulated emission
      \item inside
      \item output
      \item to get excited
      \item to flash on
      \item to inject (energy)
      \item coherent
      \item organised
      \item to strengthen
      \item to implement \\
\end{enumerate}

\allocation{ B. }
\begin{enumerate}
      \item[a.] input
      \item[b.] to emit (energy)
      \item[c.] disorganised
      \item[d.] to decrease
      \item[e.] outside
      \item[f.] incoherent
      \item[g.] to reflect
      \item[h.] to calm down
      \item[i.] to weaken
      \item[j.] spontaneous emission
      \item[k.] to prevent, delay
      \item[l.] to flash off
\end{enumerate}

\subsection*{Solution}
\begin{enumerate}
      \item d
      \item g
      \item j
      \item e
      \item a
      \item h
      \item l
      \item k
      \item f
      \item c
      \item i
      \item l
\end{enumerate}

\chapter{Домашнее задание №2 22.02.24}

\section{Exercise №28}
\subsection*{Rewrite each sentence replacing the words in italics by their opposites.
      Use the words in brackets so that the new sentence has the meaning opposite to
      the first sentence.}
\allocation{Example:}The production efficiency is the result of good work. (bad). → The
production inefficiency is the result of poor work.

1. Black surfaces absorb more light than other colours. (white) 2. In spring wild birds
increase in number in Moscow region. (in autumn) 3. Spontaneous emission takes place
without interaction with other photons. (when photon emission is triggered by other
photons) 4. It feels really warm inside on a winter morning. (cold) 5. A mouse and a
keyboard are the examples of input devices. (amonitor and a printer) 6. For the system
(such as an atom or a molecule) to calm down, you need to make its energy level lower.
(higher than the ground state). 7. If you want to take a picture when it is dark you
should choose a ‘flash on’ mode. (in daylight) 8. Ordinary light unlike laser light is
incoherent and disorganized. (laser light) 9. The committee agreed that it was necessary
to implement the changes recommended in the report. 10. Our attention is weakened by
stress. (mindfulness)

\subsection*{Solution}
\begin{enumerate}
      \item White surfaces reflect more light than other colors. (black)
      \item In autumn wild birds decrease in number in Moscow region. (in spring)
      \item Spontaneous absorption takes place without interaction with other photons.
            (when photon absorption is triggered by other photons)
      \item It feels really cold inside on a winter morning. (warm)
      \item A monitor and a printer are the examples of output devices. (a mouse and
            a keyboard)
      \item For the system (such as an atom or a molecule) to excite, you need to make
            its energy level higher. (lower than the ground state)
      \item If you want to take a picture when it is bright you should choose a ‘flash
            off’ mode. (in darkness)
      \item Laser light, unlike ordinary light, is coherent and organized. (ordinary
            light)
      \item The committee agreed that it was unnecessary to implement the changes
            recommended in the report.
      \item Our attention is strengthened by mindfulness.
\end{enumerate}

\section{Exercise №29}
\subsection*{Use the word given in brackets to form a word which fits in the gap.}
1. The name ‘laser’ stands for Light \underline{\hspace{2cm}} by stimulated emission of
radiation. (amplify) 2. Many enjoy the mental \underline{\hspace{2cm}} of a challenging
job. (stimulate) 3.Words \underline{\hspace{2cm}} thoughts and feelings.( embodiment)
4. Difficulties \underline{\hspace{2cm}} the mind, as labour does the body. (strong)
5. Laws controlling the \underline{\hspace{2cm}} of greenhouse gases should be
introduced. (emit) 6. Truth is the \underline{\hspace{2cm}} of all knowledge. (found)
7. A cloud is a mass of \underline{\hspace{2cm}} in the sky. (vaporise) 8. A graphical
\underline{\hspace{2cm}} of the experiment results is required. (represent) 9. Do you
think mobile phones emit \underline{\hspace{2cm}} ? (radiate) 10. If a text is
\underline{\hspace{2cm}}, it means that it is well planned, clear and logical.
(coherence)

\subsection*{Solution}
\begin{enumerate}
      \item The name ‘laser’ stands for Light Amplification by stimulated emission of
            radiation.
      \item Many enjoy the mental stimulation of a challenging job.
      \item Words embody thoughts and feelings.
      \item Difficulties strengthen the mind, as labour does the body.
      \item Laws controlling the emission of greenhouse gases should be introduced.
      \item Truth is the foundation of all knowledge.
      \item A cloud is a mass of vapor in the sky.
      \item A graphical representation of the experiment results is required.
      \item Do you think mobile phones emit radiation?
      \item If a text is coherent, it means that it is well planned, clear and logical.
\end{enumerate}

\section{Exercise №30}
\subsection*{Read the text and fill in the gaps with the following words in the
      appropriate form.}
concentrated, coherence, weapon, monochromatic, stands for, emission, beam, to encode
and transmit, sophisticated, represents, hence, to vaporise

In «The War of the Worlds» written before the turn of the last century, H. Wells told
a fantastic story of how Martians almost invaded our Earth. Their \allocation{1}
\underline{\hspace{2cm}} was a mysterious «sword of heat». Today Wells’ sword of heat
has come to reality in the laser. The name \allocation{2}\underline{\hspace{2cm}} light
amplification by stimulated \allocation{3}\underline{\hspace{2cm}} of radiation. Laser,
one of the most \allocation{4} \underline{\hspace{2cm}} inventions of man, produces an
intensive \allocation{5}\underline{\hspace{2cm}} of light of a very pure single colour.
It \allocation{6}\underline{\hspace{2cm}} the fulfillment of one of the humankind’s
oldest dreams of technology to provide a light beam intensive enough \allocation{7}
\underline{\hspace{2cm}} the hardest materials. There are few materials which are not
suited for laser processing, \allocation{8}\underline{\hspace{2cm}} laser treatment of
materials has become an important technique lately. The laser’s most important
potential may be its use in communications. We send and receive the data, video and
other information, using lasers \allocation{9}\underline{\hspace{2cm}} the data at rates 10 to 100
times faster than radio, because lasers can generate a very intense, \allocation{10}
\underline{\hspace{2cm}}, highly parallel and \allocation{11}\underline{\hspace{2cm}}
beam and \allocation{12}\underline{\hspace{2cm}} is a very important property of laser light.

\subsection*{Solution}
In «The War of the Worlds» written before the turn of the last century, H. Wells told
a fantastic story of how Martians almost invaded our Earth. Their \allocation{1}
\textbf{weapon} was a mysterious «sword of heat». Today Wells’ sword of heat has come
to reality in the laser. The name \allocation{2}\textbf{'laser' stands for} light
amplification by stimulated \allocation{3}\textbf{emission} of radiation. Laser, one of
the most \allocation{4}\textbf{sophisticated} inventions of man, produces an intensive
\allocation{5}\textbf{monochromatic} beam of light of a very pure single color. It
\allocation{6}\textbf{represents} the fulfillment of one of humankind’s oldest dreams
of technology to provide a light beam intensive enough \allocation{7}
\textbf{to vaporise} the hardest materials. There are few materials which are not suited
for laser processing, \allocation{8}\textbf{hence} laser treatment of materials has
become an important technique lately. The laser’s most important potential may be its
use in communications. We send and receive data, video, and other information, using
lasers \allocation{9}\textbf{to encode and transmit} the data at rates 10 to 100 times
faster than radio because lasers can generate a very intense, \allocation{10}
\textbf{concentrated}, highly parallel, and \allocation{11}\textbf{coherent} beam, and
\allocation{12}\textbf{coherence} is a very important property of laser light.

\section{Exercise №31}
\subsection*{Work in groups. Choose 5-7 words from Module 10 Word list and prepare a
      short news story to tell your group using these words. Ask your listeners to write
      down the words while they listen to your story. Compare your lists.}

\section{Exercise №32}
\subsection*{Summarise the text in English paying attention to the linking words and
      phrases}

\subsection*{Solution}
The text discusses the invention, properties, and use of lasers. Firstly, it outlines
the history of laser invention, then transitions to its properties. Thirdly, it
discusses the types of existing lasers and concludes by examining practical laser
applications in various fields. While there is no definitive answer to who invented
the laser, several scientists contributed to its creation. Despite initial expectations
of mainly military use, lasers have found widespread application in areas such as
warfare and medicine due to their ability to produce an extremely narrow beam of
light. The text also highlights the role of lasers in computer science. Overall,
lasers have a broad range of applications, from surgical procedures to vehicle speed
control devices. In conclusion, the fact that "death rays" did not become a reality
is ultimately beneficial.

The concept of "photonics" emerged in the late 20th century and has become part of
everyday life, encompassing technologies like fiber optic communication lines,
flat-screen TVs and computer monitors, smartphones, and more. Laser communication
offers high quality, greater bandwidth, and strict confidentiality. Both lasers and
fiber optics have become vital components of many industries, and their combined
potential is rapidly expanding. Semiconductor lasers are used in fiber-optic
telecommunication systems. Research in this field has led to advancements in areas
such as quantum electronics, fiber optics, quantum optics, laser physics, laser
chemistry, and more. The term "photonics" encompasses all these scientific and
technical areas.

\chapter{Семинар №3 22.02.24}

\section{Text 10С}

\subsection*{Photonics}
Photonics is the science and technology of generating, controlling, and detecting
photons, which are particles of light. Photonics underpins technologies of daily life
from smartphones to laptops, medical instruments and lighting technology. The 21st
century will depend as much on photonics as the 20th century depended on electronics.
Photonics is the science of light, it is the technology of generating, controlling, and
detecting light waves and photons, which are \allocation{1.}\underline{\hspace{2cm}}
of light. The characteristics of the waves and photons can be used to \allocation{2.}
\underline{\hspace{2cm}} the universe, to \allocation{3.}\underline{\hspace{2cm}}
diseases, and even to solve crimes. Scientists have been studying light for hundreds
of years. The colors of the rainbow are only a small part of the entire light wave
range, called the electromagnetic spectrum. Photonics explores a wider variety of
\allocation{4.}\underline{\hspace{2cm}}, from gamma rays to radio, including X-rays,
UV and infrared light. It was only in the 17th century that Sir Isaac Newton showed
that white light is made of different colors of light. At the beginning of the 20th
century, Max Planck and later Albert Einstein \allocation{5.}\underline{\hspace{2cm}}
that light was a wave as well as a particle, which was a very \allocation{6.}
\underline{\hspace{1.5cm}} theory at the time. How can light be two completely
different things at the same time? Experimentation later \allocation{7.}
\underline{\hspace{2cm}} this duality in the nature of light. The word Photonics
appeared around 1960, when the laser was invented by Theodore Maiman. Even if we
cannot see the \allocation{8.}\underline{\hspace{1.3cm}} electromagnetic spectrum,
\allocation{9.}\underline{\hspace{2cm}} light waves are a part of our everyday life.
Photonics is everywhere; in consumer electronics (barcode scanners, DVD players, remote
TV control), telecommunications (internet), health (eye surgery, medical instruments),
manufacturing industry (laser cutting and machining), defense and security (infrared
camera, remote sensing), entertainment (holography, laser shows), etc. All around the
world, scientists, engineers and technicians perform \allocation{10.}
\underline{\hspace{2cm}} research surrounding the field of Photonics. The science of
light is also actively taught in classrooms and museums where teachers and educators
share their passion for this field with young people and the general public. Photonics
opens a world of unknown and \allocation{11.}\underline{\hspace{2cm}} possibilities
limited only by a lack of imagination.

\subsection*{How Light Really Works}
Once we understand how atoms take in and give out energy, the science of light
\allocation{12.}\underline{\hspace{1.1cm}} in a very interesting new way. Think about
mirrors, for example. When you look at a mirror and see your face reflected, what's
actually going on? Light (maybe from a window) is hitting your face and \allocation{13.}
\underline{\hspace{2cm}} into the mirror. Inside the mirror, atoms of silver (or another
very reflective metal) are catching the \allocation{14.}\underline{\hspace{2cm}} light
energy and becoming \allocation{15.}\underline{\hspace{2cm}}. That makes them unstable,
so they throw out new \allocation{16.}\underline{\hspace{2cm}} of light that travel back
out of the mirror towards you. In effect, the mirror is playing throw and catch with
you using photons of light as the balls!
The same idea can help us explain things like photocopiers and \allocation{17.}
\underline{\hspace{2cm}} (flat sheets of the chemical element silicon that turn
sunlight into electricity). Have you ever wondered why solar panels look black even
when they're in full sunlight? That's because they're \allocation{18.}
\underline{\hspace{2cm}} back little or none of the light that falls on them and
\allocation{19.}\underline{\hspace{2cm}} all the energy instead. (Things that are black
absorb light, and reflect little or none, while things that are white reflect virtually
all the light that falls on them, and absorb little or none. That's why it's best to
wear white clothes on a hot day.) Where does the energy go in a solar panel if it's not
reflected? If you shine sunlight onto the solar \allocation{20.}\underline{\hspace{2cm}}
in a solar panel, the atoms of silicon in the cells catch the energy from the sunlight.
Then, instead of producing new photons, they produce a flow of electricity instead
through what's known as the \allocation{21.}\underline{\hspace{2cm}} (or photovoltaic)
effect. In other words, the incoming solar energy (from the Sun) is \allocation{22.}
\underline{\hspace{2cm}} to outgoing electricity.

\subsection*{Solution}
\begin{enumerate}
      \item particles
      \item explore
      \item cure
      \item wavelengths
      \item suggested
      \item controversial
      \item confirmed
      \item entire
      \item invisible
      \item cutting-edge
      \item far-reaching
      \item makes sense
      \item bouncing
      \item incoming
      \item excited
      \item photons
      \item cells
      \item reflecting
      \item absorb
      \item cells
      \item photoelectric
      \item converted
\end{enumerate}

\section{Exercise №17}
\subsection*{Read the text again and answer the following questions.}
\begin{enumerate}
      \item What does photonics study?
      \item How could the characteristics of waves and photons be put to practical use?
      \item What kind of waves does photonics explore?
      \item What discoveries did the scientists of the past make while studying light?
      \item What does ‘duality of light’ mean?
      \item Why can we say that photonics is everywhere?
      \item Do you agree with the opinion that photonics is really important today?
      \item What happens when you look at a mirror?
      \item Why do solar panels look black in full sunlight?
      \item Why is it best to wear white clothes on a hot day?
      \item What happens to the solar energy in a solar panel?
      \item Do you think pursuing a career in Photonics could be exciting and rewarding?
\end{enumerate}

\subsection*{Solution}
\begin{enumerate}
      \item Photonics studies "generating, controlling, and detecting photons, which
            are particles of light."
      \item The characteristics of waves and photons can be practically used to
            "explore the universe, to cure diseases, and even to solve crimes."
      \item Photonics explores "a wider variety of wavelengths, from gamma rays to
            radio, including X-rays, UV and infrared light."
      \item Scientists in the past discovered that "white light is made of different
            colors of light" and later confirmed that "light was a wave as well
            as a particle."
      \item The "duality in the nature of light" refers to the fact that "light was
            a wave as well as a particle."
      \item Photonics is everywhere because "light waves are a part of our everyday
            life."
      \item Yes, photonics is important today as it underpins technologies crucial
            to daily life and offers possibilities for advancements in various fields.
      \item When you look at a mirror, "atoms of silver... are catching the incoming
            light energy and becoming excited," which then "throw out new photons of
            light that travel back out of the mirror towards you."
      \item Solar panels look black in full sunlight because they're "reflecting back
            little or none of the light that falls on them and absorbing all the energy
            instead."
      \item It's best to wear white clothes on a hot day because "white reflect
            virtually all the light that falls on them, and absorb little or none."
      \item Solar panels convert sunlight into electricity through "the photoelectric
            (or photovoltaic) effect."
      \item Pursuing a career in Photonics could be exciting and rewarding as it
            involves "cutting-edge research" and offers "far-reaching possibilities
            limited only by a lack of imagination."
\end{enumerate}

\section{Exercise №18}
\subsection*{Listen to a short lecture about lasers and decide which of the following
      points below the speaker talks about.}

\href{https://www.youtube.com/watch?v=oUEbMjtWc-A}{https://www.youtube.com/watch?v=oUEbMjtWc-A}

\begin{itemize}
      \item The unique characteristics of laser light.
      \item How laser light is different from ordinary light
      \item How lasers are used in the military.
      \item How lasers are useful in eye surgery.
      \item How laser was invented.
      \item Different types of lasers.
      \item The operation of a ruby laser.
      \item How electronic transitions create stimulated emission.
      \item How the light becomes intensified and narrowed in wavelength inside a laser
            cavity.
      \item Innovations and improvements in laser technology.
\end{itemize}

\allocation{Useful words:} hallmark - клеймо, проба, признак; range finder - дальномер;
vitreous humor - стекловидное тело; tour de force - проявление таланта, мастерства;
xenon arc - (электрическая) дуга в атмосфере ксенона; flash lamp - импульсная лампа,
the crests and troughs - точки подъёма и спада; resonant cavity - резонансная полость;
avalanche – лавина; decay- распад.

\subsection*{Solution}
\begin{itemize}
      \item The unique characteristics of laser light, enabling technologies like range
            finders, optical communications, bar code scanners, and medical procedures.
      \item The use of lasers in eye surgery, emphasizing the precision and safety of
            green laser light.
      \item The invention of the laser by Ted Maiman in 1960.
      \item The operation of a ruby laser, including stimulated emission and the
            creation of coherent light.
      \item The process of stimulated emission through electronic transitions.
      \item The intensification and narrowing of light within a laser cavity,
            resulting in coherent light with a nearly single wavelength.
      \item Ongoing innovations and improvements in laser technology while maintaining
            fundamental principles.
\end{itemize}

\section{Exercise №19}
\subsection*{Listen to the lecture again, take notes and answer the questions.}

\begin{enumerate}
      \item What examples does the speaker give to prove his point that ‘much of our
            technology today depends on lasers’?
      \item What technology does he say highlights all other applications of lasers?
      \item What are the advantages of a laser scalpel?
      \item What are the three characteristics of laser light that the author calls‘a
            tour de force of engineering’?
      \item How are these three characteristics made?
\end{enumerate}

\subsection*{Solution}
\begin{enumerate}
      \item Examples of laser-dependent technology include range finding devices,
            optical communications, and bar code scanners.
      \item Laser technology's application in eye surgery, such as retinal reattachment,
            showcases its unique characteristics and precision.
      \item Advantages of a laser scalpel include the use of green laser light, which
            passes through the eye's lens and vitreous humor without causing damage,
            allowing for precise treatment of the retina without harming surrounding
            tissues.
      \item The three characteristics of laser light termed 'a tour de force of
            engineering' are coherent light, a narrow beam, and nearly a single
            wavelength.
      \item These characteristics are achieved through stimulated emission within a
            resonant cavity. Electrons returning to the ground state release light,
            initiating an avalanche of identical photons. Inside the cavity, reflection
            and alignment of light rays intensify and narrow the wavelength, resulting
            in coherent light.
\end{enumerate}

\chapter{Домашнее задание №3 29.02.24}

\section{Exercise №32}
\subsection*{Summarise the text in English paying attention to the linking words and
      phrases}

\subsection*{Solution}
The text discusses the invention, properties, and use of lasers. Firstly, it outlines
the history of laser invention, then transitions to its properties. Thirdly, it
discusses the types of existing lasers and concludes by examining practical laser
applications in various fields. While there is no definitive answer to who invented
the laser, several scientists contributed to its creation. Despite initial expectations
of mainly military use, lasers have found widespread application in areas such as
warfare and medicine due to their ability to produce an extremely narrow beam of
light. The text also highlights the role of lasers in computer science. Overall,
lasers have a broad range of applications, from surgical procedures to vehicle speed
control devices. In conclusion, the fact that "death rays" did not become a reality
is ultimately beneficial.

The concept of "photonics" emerged in the late 20th century and has become part of
everyday life, encompassing technologies like fiber optic communication lines,
flat-screen TVs and computer monitors, smartphones, and more. Laser communication
offers high quality, greater bandwidth, and strict confidentiality. Both lasers and
fiber optics have become vital components of many industries, and their combined
potential is rapidly expanding. Semiconductor lasers are used in fiber-optic
telecommunication systems. Research in this field has led to advancements in areas
such as quantum electronics, fiber optics, quantum optics, laser physics, laser
chemistry, and more. The term "photonics" encompasses all these scientific and
technical areas.

\section{Exercise №35}
\subsection*{Look at more examples of Participles from reading and complete the table.
      Try to define the meaning and function of the Participles in these examples.}
\begin{enumerate}
      \item The word "laser" stands for "light amplification by \allocation{stimulated}
            emission of radiation”.
      \item The idea of using lasers as death rays \allocation{employed} by creators of
            such blockbusters as X-Men and Star Wars still remains science fiction.
      \item \allocation{Having been demonstrated} by Theodore Maiman in 1960, the first
            ruby laser was considered the first successful light laser.
      \item \allocation{Having introduced} the concept of stimulated emission, Einstein
            laid the foundation for the laser.
      \item \allocation{Laser-sighting} devices are \allocation{fitted} to military and
            police rifles to help soldiers hit their targets.
      \item \allocation{Being installed} at one end of the laser tube, a mirror keeps
            the photons bouncing back and forth inside the crystal.
      \item The \allocation{escaping} photons form a very \allocation{concentrated}
            beam of powerful laser light.
      \item Light from conventional sources, such as a light bulb or the sun, diverges,
            \allocation{spreading} in all directions.
\end{enumerate}

\begin{small}
      \begin{tabular}{|r|c|c|}
            \hline
                                                             & Active      & Passive          \\
            \hline
            Present Participle (V+ing)                       & doing       & doing            \\
            Perfect Participle\textsuperscript{*}(having+V3) & having done & having been done \\
            Past Participle (V3)                             &             & done             \\
            \hline
      \end{tabular}
\end{small}

* There is also Perfect Continuous Participle form: having +been+ doing which focuses
on the duration of the action as compared to Perfect Participle.

\subsection*{Solution}
\allocation{Stimulated}: Past Participle (Active) - Having introduced the concept of
stimulated emission, Einstein laid the foundation for the laser. In this context,
"stimulated" describes the type of emission that Einstein introduced. \\
\allocation{Employed}: Past Participle (Passive) - The idea of using lasers as death
rays employed by creators of such blockbusters as X-Men and Star Wars still remains
science fiction. Here, "employed" indicates that the idea was utilized by creators. \\
\allocation{Having been demonstrated}: Perfect Participle (Passive) - Having been
demonstrated by Theodore Maiman in 1960, the first ruby laser was considered the first
successful light laser. This indicates that the first ruby laser was demonstrated by
Theodore Maiman before being considered successful. \\
\allocation{Having introduced}: Perfect Participle (Active) - Having introduced the
concept of stimulated emission, Einstein laid the foundation for the laser. This shows
that Einstein introduced the concept before laying the foundation. \\
\allocation{Laser-sighting fitted}: Past Participle (Passive) - Laser-sighting devices
are fitted to military and police rifles to help soldiers hit their targets. Here,
"fitted" is used passively to describe the state of the devices after being installed. \\
\allocation{Being installed}: Present Participle (Passive) - Being installed at one end
of the laser tube, a mirror keeps the photons bouncing back and forth inside the crystal.
This indicates the ongoing action of installing the mirror in a passive voice. \\
\allocation{Escaping concentrated}: Present Participle (Active) - The escaping photons
form a very concentrated beam of powerful laser light. Here, "escaping" and
"concentrated" are used actively to describe the properties of the photons forming the
laser beam.

\section{Exercise №36}
\subsection*{Compare the following pairs of phrases with Participle I and Participle II.
      Translate them into Russian.}
\begin{enumerate}
      \item developing industry - developed industry
      \item changing distances - changed distances
      \item a controlling device - controlled device
      \item an increasing speed - an increased speed
      \item a transmitting signal - a transmitted signal
      \item a reducing noise - reduced noise
      \item a moving object - a moved object
      \item heating parts - heated parts
\end{enumerate}

\subsection*{Solution}
\begin{enumerate}
      \item развивающаяся промышленность - развитая промышленность
      \item изменяющиеся расстояния - изменённые расстояния
      \item управляющее устройство - управляемое устройство
      \item увеличивающаяся скорость - увеличенная скорость
      \item передающий сигнал - переданный сигнал
      \item уменьшающий шум - уменьшенный шум
      \item движущийся объект - перемещённый объект
      \item нагревающиеся детали -  нагретые детали
\end{enumerate}

\section{Exercise №37}
\subsection*{Choose the correct form.}
\begin{enumerate}
      \item \allocation{A:} Have you read that new book yet?
            \allocation{B:} Only some of it. It’s very…\\
            a. bored \quad b. boring

      \item \allocation{A:} Did you enjoy your holiday?
            \allocation{B:} Oh, yes. It was very…\\
            a. relaxed \quad b. relaxing

      \item \allocation{A:} I'm going to a lecture tonight. Do you want to come?
            \allocation{B:} No, thanks. I'm not … in the subject.\\
            a. interested \quad b. interesting

      \item \allocation{A:} Did you hurt yourself when you fell?
            \allocation{B:} No, but it was very …\\
            a. embarrassed \quad b. embarrassing

      \item \allocation{A:} Was mother upset when you broke her vase?
            \allocation{B:} Not really, but she was very….\\
            a. annoyed \quad b. annoying

      \item \allocation{A:} How do you feel today?
            \allocation{B:} I still feel very ….\\
            a. tired \quad b. tiring

      \item \allocation{A:} You look ill. What’s the matter?
            \allocation{B:} I’ve had a very …. day.\\
            a. tired \quad b. tiring

      \item Sit down - I've got some very …. news for you.\\
            a. excited \quad b. exciting

      \item He's got a very …. habit of always interrupting people.\\
            a. annoyed \quad b. annoying

      \item I'm very …. by your behaviour.\\
            a. disappointed \quad b. disappointing
\end{enumerate}

\subsection*{Solution}
\begin{enumerate}
      \item \allocation{A:} Have you read that new book yet?
            \allocation{B:} Only some of it. It’s very…\\
            \textbf{Correct answer:} \textbf{b.} boring

      \item \allocation{A:} Did you enjoy your holiday?
            \allocation{B:} Oh, yes. It was very…\\
            \textbf{Correct answer:} \textbf{a.} relaxed

      \item \allocation{A:} I'm going to a lecture tonight. Do you want to come?
            \allocation{B:} No, thanks. I'm not … in the subject.\\
            \textbf{Correct answer:} \textbf{a.} interested

      \item \allocation{A:} Did you hurt yourself when you fell?
            \allocation{B:} No, but it was very …\\
            \textbf{Correct answer:} \textbf{b.} embarrassing

      \item \allocation{A:} Was mother upset when you broke her vase?
            \allocation{B:} Not really, but she was very….\\
            \textbf{Correct answer:} \textbf{a.} annoyed

      \item \allocation{A:} How do you feel today?
            \allocation{B:} I still feel very ….\\
            \textbf{Correct answer:} \textbf{a.} tired

      \item \allocation{A:} You look ill. What’s the matter?
            \allocation{B:} I’ve had a very …. day.\\
            \textbf{Correct answer:} \textbf{a.} tired

      \item Sit down - I've got some very …. news for you.\\
            \textbf{Correct answer:} \textbf{b.} exciting

      \item He's got a very …. habit of always interrupting people.\\
            \textbf{Correct answer:} \textbf{b.} annoying

      \item I'm very …. by your behaviour.\\
            \textbf{Correct answer:} \textbf{a.} disappointed
\end{enumerate}

\chapter{Семинар № 29.02.24}

\section{Exercise №38}
\subsection*{Fill in the Perfect Participle, Active or Passive, of the verbs in
      brackets. Explain the meaning of Perfect Participle phrases or translate the
      sentences into Russian.}
\begin{enumerate}
      \item \allocation{(Work)} all day, I was feeling very tired in the evening.
      \item \allocation{(Live)} in an English-speaking country for a few years, she
            spoke English like a native speaker.
      \item \allocation{(Rescue)}, an injured pilot was taken to hospital.
      \item \allocation{(Write)} the test, the students handed in their papers.
      \item \allocation{(Sign)} by the boss, the documents were sent to the customers.
      \item \allocation{(Interrupt)} a few times, he was rather annoyed.
      \item \allocation{(Stop)} the car, the police officer wanted to see the documents.
      \item \allocation{(Arrive)} at the station, we called a taxi.
      \item \allocation{(Check in)} for the flight, they were prepared for the passport
            control.
      \item \allocation{(Buy)} the car, he stopped using public transport.
\end{enumerate}

\subsection*{Solution}
\begin{enumerate}
      \item Having worked all day, I was feeling very tired in the evening.
      \item Having lived in an English-speaking country for a few years, she spoke
            English like a native speaker.
      \item Having been rescued, an injured pilot was taken to the hospital.
      \item Having written the test, the students handed in their papers.
      \item Having been signed by the boss, the documents were sent to the customers.
      \item Having been interrupted a few times, he was rather annoyed.
      \item Having stopped the car, the police officer wanted to see the documents.
      \item Having arrived at the station, we called a taxi.
      \item Having checked in for the flight, they were prepared for passport control.
      \item Having bought the car, he stopped using public transport.
\end{enumerate}

\section{Exercise №39}
\subsection*{Choose the correct form of the Participle. Translate the sentences into
      Russian.}
\begin{enumerate}
      \item They were trying to fix a USB cable \underline{\hspace{2cm}} the
            instructions from a YouTube video.
      \item Serious faults \underline{\hspace{2cm}} in the project had to be corrected
            quickly.
      \item The method \underline{\hspace{2cm}} by the engineers at the moment has
            numerous advantages.
      \item \underline{\hspace{2cm}} no job and no money, he couldn’t pay the rent.
      \item \underline{\hspace{2cm}} a new technique, scientists increased the accuracy
            of the results.
      \item People should be careful, while \underline{\hspace{2cm}} the street.
      \item \underline{\hspace{2cm}} the door, he left the house.
      \item Utilising the principle of feedback, robots can change their operation in
            response to a changing environment.
      \item \underline{\hspace{2cm}} her work, she went home.
      \item \underline{\hspace{2cm}} an expert in the field of computers, he had no
            problem finding a well-paid job.
\end{enumerate}

\subsection*{Solution}
\begin{enumerate}
      \item They were trying to fix a USB cable \allocation{following} the instructions
            from a YouTube video.
      \item Serious faults \allocation{found} in the project had to be corrected quickly.
      \item The method \allocation{being discussed} by the engineers at the moment has
            numerous advantages.
      \item \allocation{Having had} no job and no money, he couldn’t pay the rent.
      \item \allocation{Having applied} a new technique, scientists increased the
            accuracy of the results.
      \item People should be careful, \allocation{while crossing} the street.
      \item \allocation{Having locked} the door, he left the house.
      \item Utilising the principle of feedback, robots can change their operation in
            response to a changing environment.
      \item \allocation{Having completed} her work, she went home.
      \item \allocation{Being} an expert in the field of computers, he had no problem
            finding a well-paid job.
\end{enumerate}

\section{Exercise №41}
\subsection*{Rewrite the following sentences with Participle Constructions according
      to the examples given below and identify the meaning of Participle Constructions.}
\begin{enumerate}
      \item Walking in the woods, I suddenly realised that I had lost my way.
      \item Having spent a lot of time doing my homework, I went to bed very late last
            night.
      \item Given proper care, your car will operate smoothly for years.
      \item Working in a bank, he was familiar with the best ways to invest money.
      \item My sister is the one talking to the professor.
      \item Having collected the data, he began analysing the results.
      \item Having arrived at the site, the scientists discovered many fragments of the
            meteorite.
      \item Being one of the most beautiful Russian monuments, St Basil’s Basilica is a
            World Heritage site.
      \item Trying to sell more goods for cash, the company is losing money.
      \item Karel Capek described a mechanical device that looked like a human but
            lacking human sensibility could perform only automatic, mechanical
            operations.
\end{enumerate}

\subsection*{Solution}
\begin{enumerate}
      \item While walking in the woods, I suddenly realized that I had lost my way.
      \item Because I had spent a lot of time doing my homework, I went to bed very
            late last night.
      \item If given proper care, your car will operate smoothly for years.
      \item Being someone who worked in a bank, he was familiar with the best ways to
            invest money.
      \item The one talking to the professor is my sister.
      \item After having collected the data, he began analyzing the results.
      \item Upon having arrived at the site, the scientists discovered many fragments
            of the meteorite.
      \item St Basil’s Basilica, being one of the most beautiful Russian monuments, is
            a World Heritage site.
      \item In trying to sell more goods for cash, the company is losing money.
      \item Karel Capek described a mechanical device resembling a human but lacking
            human sensibility, capable only of performing automatic, mechanical
            operations.
\end{enumerate}

\section{Exercise №42}
\subsection*{Use a Participle Construction to add the information in italics to the main
      sentence.}
\allocation{Example:} Sam left school early because he felt sick. → Feeling sick, Sam
left school early. They spent all the money. So they couldn’t afford to buy a car. →
Having spent all the money, they couldn’t afford buying a car.

\begin{enumerate}
      \item As she felt tired, Anna went to bed early. \underline{\hspace{2cm}}, Anna
            went to bed early.
      \item After the boss had explained the problem, he told the employee to deal with
            it. \underline{\hspace{2cm}}, the boss told the employee to deal with it.
      \item While he was drinking his coffee, he was thinking about the problem.
            \underline{\hspace{2cm}}, he was thinking about the problem.
      \item If it is looked after carefully, the plant can live through the winter.
            \underline{\hspace{2cm}}, the plant can live through the winter.
      \item We filled up the car and continued our journey. \underline{\hspace{2cm}},
            we continued our journey.
      \item As the manager was impressed by my work, he extended my contract.
            \underline{\hspace{2cm}}, the manager extended my contract.
      \item They have written two tests today and they are too tired to do the third
            one. \underline{\hspace{2cm}}, they are too tired to do the third one.
      \item He was driving home. He had an accident. \underline{\hspace{2cm}}, he had
            an accident.
      \item He was trapped in a dilemma and couldn’t decide what to do.
            \underline{\hspace{2cm}}, he couldn’t decide what to do.
      \item After I dropped him off at the station, I drove straight to the supermarket.
            \underline{\hspace{2cm}}, I drove straight to the supermarket.
      \item The teacher was impressed by Mike’s work, so she gave him the highest mark.
            \underline{\hspace{2cm}}, the teacher gave him the highest mark.
      \item As he had been to England before, he knew where to find a good hotel.
            \underline{\hspace{2cm}}, he knew where to find a good hotel.
\end{enumerate}

\subsection*{Solution}
\begin{enumerate}
      \item Feeling tired, Anna went to bed early.
      \item Having explained the problem, the boss told the employee to deal with it.
      \item Drinking his coffee, he was thinking about the problem.
      \item If carefully looked after, the plant can live through the winter.
      \item Having filled up the car, we continued our journey.
      \item Impressed by my work, the manager extended my contract.
      \item Having written two tests today, they are too tired to do the third one.
      \item Driving home, he had an accident.
      \item Trapped in a dilemma, he couldn’t decide what to do.
      \item After dropping him off at the station, I drove straight to the supermarket.
      \item Impressed by Mike’s work, the teacher gave him the highest mark.
      \item Having been to England before, he knew where to find a good hotel.
\end{enumerate}

\section{Study note}
\subsection*{Participles Overview}
\begin{itemize}
      \item Participles are words derived from verbs that can function as adjectives
            and adverbs or as part of verb phrases to create verb tenses.
      \item The main types of Participles are:
            \begin{itemize}
                  \item Present Participle (Participle I), e.g., \textit{coming}
                  \item Perfect Participle, e.g., \textit{having completed}
                  \item Past Participle (Participle II), e.g., \textit{used}
            \end{itemize}
      \item Participles may also be identified with a particular Voice: active or
            passive.
\end{itemize}

\subsection*{Present and Past Participles as Adjectives}
\begin{itemize}
      \item Present and Past Participles can both be used as adjectives.
      \item The Present Participle describes what someone or something is (What kind?).
      \item The Past Participle describes how somebody feels (How do you feel?).
\end{itemize}

\subsection*{Perfect Participle}
\begin{itemize}
      \item The Perfect Participle (active and passive) is used to emphasize that one
            action happened before another.
      \item Perfect Participles are often used as part of Participle Constructions or
            Clauses that are equivalent to adverbial clauses within complex sentences.
\end{itemize}

\subsection*{Participle Constructions}
\begin{itemize}
      \item Participles are often used as part of Participle Constructions or Clauses,
            enabling a more economical presentation of information compared to complex
            sentences.
      \item Participle Constructions act as adjectives or adverbs within sentences and
            usually are reduced adverbial or relative clauses.
      \item They can be used after various conjunctions such as: when, while, if,
            though, etc.
\end{itemize}

\subsection*{Negative Participle Constructions}
\begin{itemize}
      \item Negative participle constructions are possible, where 'not' normally comes
            before the Participle.
\end{itemize}

\subsection*{Example Sentences}
\begin{enumerate}
      \item A humanoid drawn by Leonardo da Vinci is among the first verifiable
            automation.
      \item Being fitted with vision equipment, robots are able to ‘see’.
      \item I hurt my arm while playing tennis.
      \item Remember to take all your belongings with you when leaving the train.
      \item Not having seen the film, I could not take part in its discussion.
\end{enumerate}
