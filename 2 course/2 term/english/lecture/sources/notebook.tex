\documentclass[12pt, a4paper]{report}
\usepackage[left=2cm, right=1.5cm, top=2cm, bottom=2cm, bindingoffset=0cm]{geometry} % поля
\usepackage{indentfirst} % отступ после заголовка
% \setlength{\parindent}{5ex} % отступ в начале абзаца
% \setlength{\parskip}{1em} % отступ между абзацами

\usepackage[T2A]{fontenc}
\usepackage[utf8]{inputenc} % кодировка
\usepackage[english,russian]{babel} % язык
\usepackage{cmap} % поиск в pdf

\usepackage{enumitem} % форматирование списков
\setlist{noitemsep}
% \setlist[1]{\labelindent=\parindent} % < Usually a good idea
\setlist[itemize]{leftmargin=*}
\setlist[itemize,1]{label=$\triangleleft$}
\setlist[enumerate]{labelsep=*, leftmargin=1.6pc}
\setlist[enumerate,1]{label=\arabic*., ref=\arabic*}
\setlist[enumerate,2]{label=\emph{\alph*}.,ref=\theenumi.\emph{\alph*}}
\setlist[enumerate,3]{label=\roman*., ref=\theenumii.\roman*}
\setlist[description]{font=\sffamily\bfseries}

\usepackage{tikz} % выделение текста (` `)
\tikzset{baseline, inner sep=2pt, minimum height=12pt, rounded corners=2pt}
\newcommand{\code}[1]{\mbox{\ttfamily \tikz \node[anchor=base,fill=black!12]{#1};}}

\makeatletter % убираю "Глава №"
\def\@makechapterhead#1{
    {\normalfont\bfseries\begin{center}\lowercase{#1}\end{center}\par\nobreak\vskip 10\p@}}
\makeatother

\usepackage{graphicx} % картинки
\graphicspath{ {images/} } % место по умолчанию

\usepackage{amsmath} % математика

%--------------------------------------------------------------------------------------%

\title{Тетрадь}
\author{Цыпышев Тимофей, ИУ5-41Б \thanks{Семинары ведёт Медведева К.О.}}
\date{8 февраля 2024 г. - \today}

\begin{document}

\maketitle
\tableofcontents

%--------------------------------------------------------------------------------------%

\part{Module 1}

%--------------------------------------------------------------------------------------%

\chapter{Seminar 1}

\section{Exercise №1}

\subsection*{Match the words (1-6) with their definitions (a-f).
    Use a dictionary if necessary.}
\begin{enumerate}
    \item stimulated
    \item radiation
    \item acronym
    \item emission
    \item beam
    \item amplification \\
\end{enumerate}

\begin{enumerate}
    \item[a.] energy in the form of heat or light that you cannot see and
        which can be very harmful
    \item[b.] a word formed from the initial letters of other words
    \item[c.] the increase in volume of a signal
    \item[d.] a line of radiation or particles flowing in one direction
    \item[f.] the act of sending out gases or other substances
    \item[e.] made stronger or more active
\end{enumerate}

\subsection*{Solution}
\begin{enumerate}
    \item f
    \item a
    \item b
    \item e
    \item d
    \item c
\end{enumerate} \leavevmode\newline

\section{Exercise №2}

\subsection*{In groups answer the questions.}
\begin{enumerate}
    \item What is a laser?
          \begin{enumerate}
              \item a device which produces a very narrow beam of light useful in
                    many technologies
              \item a process of optical amplification of light based on
                    radiation emission
              \item both a and b
          \end{enumerate}
    \item What kind of word is the word ‘laser’?
          \begin{enumerate}
              \item acronym
              \item shortening
              \item contraction
          \end{enumerate}
    \item Can you decode the word ‘laser’? (use the words from task 1)
          \begin{enumerate}
              \item[] L... A... by Stimulated E... of R... .
          \end{enumerate}
\end{enumerate}

\subsection*{Solution}
\begin{enumerate}
    \item a
    \item a
    \item Light Amplification by Stimulated Emission of Radiation
\end{enumerate}

\section{Exercise №3}

\subsection*{Study the pictures below. Which of the following words and phrases
    refer to ordinary light (1) and which to laser light (2)?}
Coherent; its intensity decreases with distance; highly monochromatic;
it is not strictly monochromatic; organized; less intense; travels in one direction;
incoherent; highly intense; concentrated; travels in all directions; disorganized.

\subsection*{Solution}
\code{Ordinary light:}
\begin{itemize}
    \item disorganized
    \item its intensity decreases with distance
    \item it is not strictly monochromatic
    \item less intense, incoherent
    \item travels in all directions
\end{itemize}

\code{Laser light:}
\begin{itemize}
    \item organized
    \item coherent
    \item highly monochromatic
    \item travels in one direction
    \item highly intense
    \item concentrated
\end{itemize} \leavevmode\newline

\section{Exercise №6}
\code{[Устно]}

\subsection*{Read the text again and answer the following questions.}
\begin{enumerate}
    \item Why can we say that lasers were predicted long before their invention?
    \item What is a laser? What does the word ‘laser’ mean?
    \item What kind of beam do lasers have?
    \item What do we mean by the words ‘monochromatic, directional, and coherent’
          when we refer to laser light?
    \item Why is the light from the laser so concentrated?
    \item Who proposed the theoretical possibility of the process that made lasers
          possible?
    \item Who created the first microwave generator?
    \item Who demonstrated the first successful light laser?
    \item What laser types are mentioned in the text?
    \item Do you agree with the author’s opinion that lasers have found myriads
          of useful applications? What examples do you think best prove this point?
    \item While reading this text, which uses of lasers surprised you the most?
    \item Can you think of an example of a laser device or technology that you
          have used or are using?
\end{enumerate}

\section{Exercise №7}
\code{[Устно]}

\subsection*{Read the statements and decide which of them are true (T) and
    which are false (F) according to text 10A. Explain why.}
\begin{enumerate}
    \item The word ‘laser’ means microwave amplification by stimulated emission
          of radiation.
    \item Laser was invented at the dawn of the 20th century.
    \item Albert Einstein was the first inventor of a laser.
    \item Laser came into existence only in the second half of the 20-th century.
    \item Unfortunately most of the applications of a laser proved to be unattainable
          in the real world.
    \item The use of lasers in thermonuclear fusion reactors may be the key
          to the future.
    \item Laser weapons are widely used by the military.
    \item In medicine lasers can be used for various surgical procedures.
    \item Very few inventions can match the impact of the laser’s invention.
    \item Laser technology has a promising future.
\end{enumerate} \leavevmode\newline

%--------------------------------------------------------------------------------------%

\chapter{Homework 1}

\section{Exercise №5}

\subsection*{Find the words and phrases in the text which have the following meanings.}
\begin{enumerate}
    \item[§] \code{1}
        \begin{enumerate}
            \item[1.] a verb: to make someone remember something
            \item[2.] a verb: to use a particular idea or method
            \item[3.] a verb: to continue to be in the same state or condition
            \item[4.] a verb: to express clearly or show the importance of an idea
                or principle
        \end{enumerate}
    \item[§] \code{2}
        \begin{enumerate}
            \item[5.] a noun: the product of making larger or greater in amount
                or intensity
            \item[6.] a noun: the result of sending something out (e.g. gas or heat)
            \item[7.] a verb: to make stronger
            \item[8.] a noun: the point from which something begins
            \item[9.] an adverb: in relation to something else
            \item[10.] a noun: a shining line of light
            \item[11.] an -ing form of a verb: covering a large area
            \item[12.] a verb: to go down to a lower level
            \item[13.] a phrase used when you are comparing objects or situations
                and saying that they are completely different
            \item[14.] the amount of something (energy, work, information) produced
                by a machine
        \end{enumerate}
    \item[§] \code{3}
        \begin{enumerate}
            \item[15.] an adverb: after a long time
            \item[16.] a verb phrase: to provide something (idea, principle) from
                which another thing can develop
            \item[17.] a verb: to give someone a prize for something they have done
        \end{enumerate}
    \item[§] \code{4, 5}
        \begin{enumerate}
            \item[18.] a prepositional phrase: because of or thanks to
            \item[19.] an adjective: unusual or surprising and therefore deserving
                attention
            \item[20.] a verb: to have a particular result, especially one that you
                didn’t expect
            \item[21.] a verb: to write something (e.g. information) down
            \item[22.] a verb: to change into a vapour
            \item[23.] a verb: to find the size, length or amount of something
            \item[24.] a noun: the quality of being correct and true
            \item[25.] a verb: to carry out
            \item[26.] a verb phrase: to be of primary importance
        \end{enumerate}
\end{enumerate}

\subsection*{Solution}
\begin{enumerate}
    \item remind
    \item employ
    \item remain
    \item -
    \item amplification
    \item emission
    \item strengthen
    \item source
    \item in contrast ?
    \item beam
    \item -
    \item decrease
    \item the difference between
    \item intensity
    \item eventually
    \item -
    \item award
    \item due to
    \item remarkable
    \item -
    \item record
    \item -
    \item measure
    \item accuracy
    \item -
\end{enumerate} \leavevmode\newline

\section{Exercise №8}

\subsection*{Complete the sentences using the information from the text without
    looking into the text.}
\begin{enumerate}
    \item The word laser is an acronym standing for …
    \item Laser light differs from ordinary light due to its …
    \item Russian physicists Nikolay Basov and Alexander Prokhorov created … while
          working on …
    \item In 1960, physicist from California Theodore Maiman demonstrated …
    \item Lasers turned out to have myriads of uses, from … to …
    \item In science lasers provide great assistance with …
    \item Laser-sighting devices are fitted to … to help soldiers …
    \item Today new applications of lasers are …
    \item Not long ago archaeologists uncovered … using Lidar.
    \item In computing lasers could have …
\end{enumerate}

\subsection*{Solution}
\begin{enumerate}
    \item Laser stands for "light amplification by stimulated emission of radiation." \\
          "The word 'laser' stands for 'light amplification by stimulated emission
          of radiation'." (Paragraph 2)
    \item Laser light is different from ordinary light because it's monochromatic,
          directional, and coherent. \\
          "The laser produces a well-directed, very intense beam which is monochromatic,
          directional and coherent." (Paragraph 2)
    \item Basov and Prokhorov created the precursor to the laser while studying the
          quantum oscillator. \\
          "In 1954, Russian physicists Nikolay Basov and Alexander Prokhorov working
          on the quantum oscillator created the first microwave generator, laser’s
          predecessor." (Paragraph 3)
    \item In 1960, Maiman demonstrated the first ruby laser. \\
          "In 1960, physicist from California Theodore Maiman demonstrated the first
          ruby laser." (Paragraph 3)
    \item Lasers have diverse applications, from medicine to communications. \\
          "Due to their remarkable properties, lasers turned out to have all sorts
          of useful applications in different fields from communications to medicine."
          (Paragraph 4)
    \item Lasers assist greatly in scientific spectroscopy. \\
          "In science they are a great help in spectroscopy." (Paragraph 4)
    \item Laser-sighting aid devices soldiers in hitting targets. \\
          "Laser-sighting devices are fitted to military and police rifles to help
          soldiers hit their targets." (Paragraph 4)
    \item New laser applications are continually emerging. \\
          "New applications of lasers are constantly emerging." (Paragraph 5)
    \item Archaeologists found ancient structures using Lidar. \\
          "Not long ago archaeologists uncovered a new vast network of cities and
          roads in the thick jungles around the ancient Cambodian temple complex
          of Angkor Wat, implementing an aerial survey using Lidar." (Paragraph 5)
    \item Lasers in computing could greatly improve data transfer speeds. \\
          For example, a silicon laser computer chip promises faster data
          transfers." (Paragraph 5)
\end{enumerate} \leavevmode\newline

%--------------------------------------------------------------------------------------%

\end{document}