\part{Лекция №1 07.02.24}

\chapter{Введение в базы данных}

\section{Контрольные мероприятия}

ЛР(6+1) + РК + Курсовая работа(Продолжение макета)

Распределённый экзамен:
60 = 3 (Удов.)
71 = 4 (Хор.)
85 = 5 (Отл.)

\section{Введение}

\textbf{База данных}  - это упорядоченный набор структурированной информации или данных.

Виды баз данных:
\begin{itemize}
    \item Доряляционные
    \item Реляционные
    \item Постреляционные
\end{itemize}

\subsection{Реляционная модель данных}

\textbf{Реляционная база данных} - это составленная по реляционной модели база данных, в которой данные, занесенные в таблицы, имеют изначально заданные отношения. Сами таблицы в такой базе данных также соотносятся друг с другом строго определенным образом. Реляционные базы данных используют целый комплекс инструментов, которые обеспечивают целостность данных, т. е. их точность, полноту и единообразие.

Данные в реляционной базе данных формируют отношения - двумерные таблицы с информацией о сущностях, т.е. объектах. Строка такой таблицы называется кортежем. Кортежи содержат множество атрибутов одной сущности, категории которых задаются в столбцах.

Типы данных:
\begin{itemize}
    \item Реляционные
    \item Постреляционные
\end{itemize}

\subsection{Правильное именование сущностей}

При выборе имени сущности разумно придерживаться таких правил:
\begin{itemize}
    \item Имя должно быть существительным (полным, сокращенным либо аббревиатурой) в единственном числе.
    \item Имя должно быть как можно короче. Оптимально - 2-4 буквы, максимум до 10.
    \item Имя должно быть уникальным в пределах базы данных.
    \item Имя должно быть мнемонически понятным проектантам без заглядывания в словарь (но словарь такой хорошо бы составить).
    \item Желательно, чтобы имена не начинались и не заканчивались на другие имена сущностей.
\end{itemize}

\section{Группы операторов SQL}

Операторы базы SQL подразделяются на несколько основных групп по признаку типа задач, которые можно решить с их помощью.

\subsection{DDL (Data Definition Language)}

Представляют собой группу операторов для определения данных. Они работают с целыми таблицами. Такие операторы SQL используются в тех случаях, когда нужно внести в базу новую таблицу или, напротив, удалить старую. Они включают в себя следующие командные слова:

\begin{itemize}
    \item \allocation{CREATE} - создание нового объекта в существующей базе.
    \item \allocation{ALTER} - изменение существующего объекта.
    \item \allocation{DROP} - удаление объекта из базы.
\end{itemize}

\subsection{DML (Data Manipulation Language)}

Эти операторы языка SQL предназначены для манипуляции данными. С их помощью меняется наполнение таблиц. Они позволяют изменять значение строк, столбцов и прочих атрибутов. Такие операторы SQL, например, позволяют удалить информацию о сотруднике, который больше не работает в компании, или исправить данные действующих специалистов. Эти операторы SQL представлены следующими командными словами:

\begin{itemize}
    \item \allocation{SELECT} - позволяет выбрать данные в соответствии с необходимым условием.
    \item \allocation{INSERT} - осуществляют добавление новых данных.
    \item \allocation{UPDATE} - производит замену существующих данных.
    \item \allocation{DELETE} - удаление информации.
\end{itemize}

\subsection{DCL (Data Control Language)}

Это операторы SQL, предназначенные для определения доступа к данным. С их помощью можно закрыть или открыть для пользователей работу с базой. Такие операторы необходимы, чтобы ограничить кого-либо из сотрудников в доступе к информации или, наоборот, позволить работать с базой новому специалисту.

\begin{itemize}
    \item \allocation{GRANT} - предоставляет доступ к объекту.
    \item \allocation{REVOKE} - аннулирует выданное ранее разрешение на доступ.
    \item \allocation{DENY} - запрет, который прекращает действие разрешения.
\end{itemize}

\subsection{TCL (Transaction Control Language)}

Предназначен для управления транзакциями, то есть таким сочетанием команд, которые выполняются в определённом алгоритме. Транзакция проведена успешно, если все необходимые команды выполнены пошагово. Если же в какой-либо из них произошёл сбой, то вся операция, включая предыдущие команды, отменяется. Простым и понятным примером таких операторов SQL является проведение банковских платежей.

При этом вы сначала вводите сумму, а затем подтверждаете отправку платежа кодом, который вам присылает банк. Если операция не будет подтверждена, то транзакция отменится автоматически.
\begin{itemize}
    \item \allocation{BEGIN TRANSACTION} - начало транзакции.
    \item \allocation{COMMIT TRANSACTION} - изменение команд транзакции.
    \item \allocation{ROLLBACK TRANSACTION} - отказ в транзакции.
    \item \allocation{SAVE TRANSACTION} - формирование промежуточной точки сохранения внутри операции.
\end{itemize}